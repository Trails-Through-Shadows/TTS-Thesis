\chapter{Formáty využívající se v API}
V této kapitole budou popsány především dva základní formáty dat které jsou stále aktivní -- JSON a XML.

\section{JSON}
JavaScript Object Notation neboli JSON je formát, který je odvozen z Javascriptu, nicméně mnoho dnešních jazyků už má serializaci do JSONu zabudovanou interně. Jedná se o textový formát zápisu objektů, který je dobře čitelný člověkem. Ukládá data do párů \textbf{key:value} kdy key je klíč a value hodnota, která je pod ním uložená, obvykle může jít o číslo, textový řetězec, pole nebo i další objekt. Znaky v JSONu musí být v kódování UTF-8, ale formát podporuje i speciální znaky pokud jsou escaped, jako příklad můžeme uvést znaky \verb |\uD83D\uDE10| nebo \textit{neutral-face}. JSON je využíván primárně k výměně dat mezi webovými aplikacemi a servery, ale dá se používat i pro jednoduché databáze. Má prostá ale zato striktní pravidla a tudíž je jednoduché zkontrolovat jeho správnost. Jeden z jeho nedostatků je, že nemá podporu komentářů oproti XML, které tuto podporu má. \textbf{odkaz někde na iso normu https://www.iso.org/standard/71616.html} %TODO bibtech na iso normu plus dát emoji 😐 třeba xetex ale to se pak celé rozbije


\subsection{Pravidla}
\textbf{Key} neboli klíč daného objektu je vždy string a reprezentuje název určitého atributu. \textbf{Value} představuje hodnotu, kterou tento atribut nabývá a může být datového typu text, číslo, logická hodnota, null, další objekt či pole. Jednotlivé atributy jsou vždy odděleny čárkou. Jako příklad je uveden JSON objekt z API modelové hry, který popisuje hráče, jeho vlastnosti a jeho rasu, reprezentovanou jako vnořený objekt.

\begin{listing}[H]
  \inputminted{json}{resources/code/standards/player.json}
  \caption{Příklad JSON objektu}
  \label{code:json_player}
\end{listing}

V tomto JSONu můžeme vidět všechno, s čím se u tohoto datového formátu můžeme setkat. Jako první atribut, tento objekt má \verb|"id":4|, kde id je klíč a 4 je číslo -- to značí, že tento objekt má svůj unikátní identifikátor 4.

Na 6 řádku můžeme vidět atribut \verb|"title":"Elf"|. Value je v tomto případě text, který poznáme podle zaobalení uvozovek. Tento řádek v kontextu celého objektu značí, že se jedná o rasu postavy, která má název \textit{Elf}.

Atribut, který tento text zaobaluje, je \texttt{"race"}. Jedná se o příklad objektu, v tomto případě rasy, která má své vlastnosti reprezentované právě tímto objektem.

V atributu \texttt{"effects"} se nachází pole objektů. V tomto případě to jsou objekty obsahující klíč s id rasy a jejím efektem, ke kterému je přidána také informace o tom, od jaké úrovně je tento efekt zpřístupněn. Podobně tak na řádku 26 je jako value klíče \texttt{inventory} pole textových řetězců znázorňující předměty, které hráč vlastní.

Dále si můžeme povšimnout atributu \texttt{clazz} na řádku 3, který má hodnotu \texttt{null}. To nám značí, že tento konkrétní klíč u objektu zatím nic neobsahuje. Může se stát že jen nebyl přiřazen a v budoucnu nějakou hodnotu dostane.

A nakonec zde máme na řádku 29 atribut \texttt{dead} s hodnotou \texttt{false}. Tato hodnota značí logickou hodnotu, která má datový typ boolean. V našem případě se jedná o vlastnost, která nám říká, zda je hráč mrtvý (\texttt{true}) či nikoli (\texttt{false}).

\subsection{Serializace a deserializace}
%TODO

\section{XML}
Extensible Markup Language je jazyk primárně určený na serializaci a přenášení dat, který je podobně jako JSON čitelný i člověkem. Základní stavební blok je node, atributy jsou uloženy do párů kterým říkáme tagy. Za pomocí své deklarace podporuje určení kódování a různé verze. Díky XML Schema definition také podporuje celou řadu datových typů a oproti JSONu podporuje i komentáře. Standardy spravuje společnost W3C. \textbf{fest fajny odkaz na w3c specifikaci do bibtex} %TODO bibtech na w3c


\subsection{Pravidla}
Na začátku dokumentu je vždycky XML deklarace, která určuje, o jakou verzi se jedná a jaké mají znaky kódování.
Samotná data jsou zaznamenána pomocí párových tagů, dohromady nazývaným \textbf{element}, které se zapisují jako počáteční (\texttt{<character>}) a ukončovací (\texttt{</character>}) tag, kde \textit{character} značí název atributu, podobně jako \textbf{key} u JSONu. Hodnota atributu se píše buď mezi počáteční a ukončovací tag, nebo přímo do tagu samotného (\texttt{<character id=4>}). Tzv. \textit{procesory} analyzují XML dokumenty a posílají dále strukturovaná data aplikaci, která se využívá. Procesory můžou být jak validující tak nevalidující, přičemž validující musí nalezenou chybu nahlásit ale pořád mohou v parsování pokračovat.

\begin{listing}[H]
  \inputminted{xml}{resources/code/standards/player.xml}
  \caption{Příklad XML dokumentu i se schématem}
  \label{code:xml_player}
\end{listing}

%TODO nějaký odkaz pěkný nebo tak něco na 
Tento XML dokument reprezentuje stejnou strukturu jako JSON popsaný výše. Oproti němu je však rozšířen o schéma, ve kterém můžeme určit jeden z mnoha datových typů \ref{fig:xml_datatypes} pro každý element (řádek 3) či atribut (řádek 6 a 7). Když není schéma specifikováno, všechny datové typy jou brány za text. Velká změna oproti JSONu také spočívá v tom, že víme, o jaký objekt se jedná (řádek 23 a 48) -- místo obyčejných složených závorek zde máme element se jménem \textit{character}, což je informace, kterou JSON neposkytuje.

\begin{figure}[H]
  \centering
  \includegraphics[width=0.7\textwidth]{figures/type-hierarchy.png}
  \caption{Hierarchie datových typů v XML schema}%\cite{british_museum_2021}}
  \label{fig:xml_datatypes}
\end{figure}
%TODO odkaz na obrázek všech možných datových typů https://www.w3.org/TR/xmlschema-2/type-hierarchy.gif a přímo ta stránka https://www.w3.org/TR/xmlschema-2/#built-in-datatypes


\endinput
\chapter{Formáty využívající se v API}
V této kapitole budou popsány především dva základní formáty dat které jsou stále aktivní. A to JSON a XML.

\section{JSON}
JavaSctript Object Notation neboli JSON je formát který je odvozen z Javascriptu. Nicméně mnoho dnešních jazyků už má zabudovanou serializaci do JSONU. Jedná se o textový formát zápisu objektů který je dobře čitelný člověkem. Ukládá data do párů \textbf{key:value} kdy value může být i další objekt nebo pole. Je využíván primárně k výměně dat mezi webovými aplikacemi a servery. Má jednoduchá ale zato striktní pravidla a tudíž je jednoduché zkontrolovat jeho správnost. \textbf{odkaz někde na iso normu https://www.iso.org/standard/71616.html} %TODO bibtech na iso normu

\subsection{Pravidla}
\textbf{Key} je vždycky string a reprezentuje atribut. \textbf{Value} je hodnota kterou nabývá key a může být text, číslo, logická hodnota, null, další objekt či pole. Jednotlivé atributy jsou vždy odděleny čárkou. Jako příklad máme JSON objekt z našeho API. Tento objekt popisuje hráče, jeho vlastnosti a navíc jeho rasu.

\newpage

\begin{listing}
  \begin{minted}{json}
{
  "id": 4,
  "clazz": null,
  "race": {
    "id": 4,
    "title": "Elf",
    "effects": [
      {
        "key": {
          "idRace": 4,
          "idEffect": 25
        },
        "levelReq": 1
      },
      {
        "key": {
          "idRace": 4,
          "idEffect": 28
        },
        "levelReq": 1
      }
    ]
  },
  "title": "Legolas",
  "playerName": "Jožko",
  "inventory": ["dagger","robe"],
  "url": "https://api.tts-game.fun/images/characters/elf-bard.png",
  "health": 8,
  "dead": false
}
  \end{minted}
  \caption{ree}
  \label{code:sumaradyasm}
\end{listing}

V tomto JSONu máme všechno s čím se můžeme potkat. Jako první máme \verb|"ID":4|. ID je klíč a 4 je číslo. To značí že tento objekt má svůj unikátní identifikátor 4. Může se ovšem stát že by naše číslo 4 bylo v uvozovkách \verb|"4"|. To by pak bylo ohodnoceno jako text ne jako číslo a kdybychom s tím dále chtěli pracovat jako s číslem tak si ho musíme v programu přetypovat. \textbf{idk jestli to tu tento příklad dávat } S tím se také pojí deserializace když chceme převádět JSON zpátky na objekt tak datové typy v objektu musí sedět s těmi co jsou v JSONu.

Dále zde máme na 6 řádku \verb|"title":"Elf"|. Value je v tomto případě text. Taktéž tento řádek v kontextu objektu nám značí že je to rasa která má název Elf.

O dva řádky výše je \verb|"race":{<vlastnosti>}|. To je příklad objektu. v tomto případě rasy a ta má nějaké své vlastnosti.

V key \texttt{"effects"} se nachází pole objektů. V tomto případě to jsou klíče co určují rasu a její efekt plus od jaké úrovně je tento efekt zpřístupněn. Stejně tak na řádku 26 je jako value pole textových řetězců. V tomto případě se jedná o inventář hráče a jeho předměty.

A v neposlední řadě máme na řádku 3 jako value \texttt{null}. To nám značí že tento konkrétní key u tohoto objektu zatím nic neobsahuje. Může se stát že v budoucnu nějakou value dostane nebo může existovat u jiného objektu.

A nakonec zde máme na řádku 29 jako value \texttt{false}. To značí logickou hodnotu a tato value má datový typ boolean. V našem případě se jedná o vlastnost která nám říká zda je hráč mrtvý (\texttt{false}) či nikoli (\texttt{true}).


\section{XML}
Extensible Markup Language



\endinput
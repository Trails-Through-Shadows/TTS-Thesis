\chapter{Formáty využívající se v API}
V této kapitole budou popsány především dva základní formáty dat které jsou stále aktivní. A to JSON a XML.

\section{JSON}
JavaSctript Object Notation neboli JSON je formát který je odvozen z Javascriptu. Nicméně mnoho dnešních jazyků už má zabudovanou serializaci do JSONU. Jedná se o textový formát zápisu objektů který je čitelný člověkem. Ukládá data do párů key:value kdy value může být i další objekt nebo pole. Je využíván primárně k výměně dat mezi webovými aplikacemi a servery.

\subsection{idk jak to pojmenovat}
\textbf{Key} je vždycky string a reprezentuje atribut. \textbf{Value} je hodnota kterou nabývá key a může být text, číslo další objekt či pole. Jako příklad máme JSON objekt z našeho API. Tento objekt popisuje hráče, jeho vlastnosti a navíc jeho rasu.
\begin{lstlisting}[language=json, caption=Příklad JSON dokumentu]
{
  "id": 4,
  "race": {
    "id": 4,
    "title": "Elf",
    "effects": [
      {
        "key": {
          "idRace": 4,
          "idEffect": 25
        },
        "levelReq": 1
      },
      {
        "key": {
          "idRace": 4,
          "idEffect": 28
        },
        "levelReq": 1
      }
    ]
  },
  "title": "Legolas",
  "playerName": "Jožko",
  "inventory": [],
  "url": "https://api.tts-game.fun/images/characters/elf-bard.png",
  "health": 8
}
\end{lstlisting}

V tomto JSONu máme všechno s čím se můžeme potkat. Jako první máme \verb|"ID":4|. ID je klíč a 4 je číslo. To značí že tento objekt má svůj unikátní identifikátor 4. Může se ovšem stát že by naše číslo 4 bylo v uvozovkách \verb|"4"|. To by pak bylo ohodnoceno jako text ne jako číslo a kdybychom s tím dále chtěli pracovat jako s číslem tak si ho musíme v programu přetypovat. \textbf{idk jestli to tu tento příklad dávat }S tím se také pojí deserializace když chceme převádět JSON zpátky na objekt tak datové typy v objektu musí sedět s těmi co jsou v JSONu.

Dále zde máme \verb|"title":"Elf"|. Value je v tomto případě text. Taktéž tento řádek v kontextu objektu nám značí že je to rasa která má název Elf.

O dva řádky výše je \verb|"race":{<objekt>}|. To je příklad objektu. v tomto případě rasy a ta má nějaké své vlastnosti.

V key \texttt{"effects"}


\endinput
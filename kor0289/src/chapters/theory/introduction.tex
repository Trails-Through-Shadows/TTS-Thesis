\chapter{Úvod}
Na trhu není mnoho hybridních deskových her, za to je zde mnoho velkých čistě deskových výpravných her. Ovšem rozložit tyto hry na stůl a nastudovat pravidla jak se vlastně hrají trvá věčnost. Nemluvě o uklízení a balení. Pár deskových her jsme s ostatními spolupracovníky vyzkoušeli ale bylo zde až moc příliš věcí které jsme museli hlídat a kontrolovat. Proto jsme přišli s nápadem vytvořit hybridní deskovou hru kde všechny ty \textit{otravné} věci co si člověk musel hlídat, případně provádět tahy místo nepřátel, byly nahrazeny počítačem. Taktéž by zde byla možnost zobrazit aktuální pravidla k dané situaci. Takové to optimalizace by zajisté zrychlili prvotní nastavování hry a urychlilo tahy nepřátel. 

Tato práce se zaměří právě na vytvoření \gls{api} pro takovou velkou výpravnou hru. V prvotní části se zaměří na analýzu a vhodný výběr datového formátu pro komunikaci mezi aplikacemi. Dále bude rozebrána vhodná architektura pro implementaci \gls{api} s ohledem na nejpoužívanější architektury v dnešní době. Taktéž zde budou rozebrány možnosti zabezpečení \gls{api} s jakými formáty. A naposledy bude rozebrána analýza i již existujících herních \gls{api}, analýza co by mělo podporovat \gls{api} probírané v této práci a jaký framework či databázi použít. Provede se zde návrh modelu vývoje a obecně spolupráce se samotnou databází která bude použita pro ukládání dat. Bude zde probrána i samotná implementace ve vybraném frameworku a na jaké problémy se narazilo.

Výsledkem této páce bude kompletní zdokumentované \gls{api} pro výpravnou hru, které bude schopné komunikovat jak s administrátorským rozhraním tak se zobrazovací částí. Bude podporovat validace vstupních dat a bude obsahovat koncové body s herní logikou. Taktéž by mělo být schopné zabezpečit komunikaci mezi těmito ostatními stranami. Jako vedlejší produkt bude uživatelská příručka pro správné použití koncových bodů.
% Create the glossary
\newglossary[glignored]{shortcuts}{glsto}{glstr}{Seznam zmíněných zkratek}
\newglossary[glignored]{gameslist}{glsto}{glstr}{Seznam her or smthng}

% Define the style for the glossary
\newglossarystyle{shortcuts}% style name
{% base it on the "list" style
    \setglossarystyle{list}%
    \renewcommand*{\glossentry}[2]{%
        \item[\glsentryitem{##1}%
                    \glstarget{##1}{\glossentryname{##1}}]
        \ifglshaslong{##1}%
        { (\abbrtype{##1}: \glsentrylong{##1})\space}{}%
        \glossentrydesc{##1}\glspostdescription\space ##2}%
}

% Define the style for the glossary
\newglossarystyle{gameslist}
{
    \renewcommand*{\glossaryheader}{}
    \renewcommand*{\glossentry}[2]{
        \item \textbf{\glsentryname{##1}} - \glsentrydesc{##1}
        \\\hspace*{1em}BoardGameGeek: \url{\glsentryuseri{##1}}
        \ifthenelse{\equal{\glsentryuserii{##1}}{}}{}{\\\hspace*{1em}Vydavatel: \url{\glsentryuserii{##1}}} \\
    }
}

% Define the glossary entries
% user1 - bggurl
% user2 - publisherurl (optional)

\newglossaryentry{gamelist}{
    type=gameslist,
    name={BBB},
    description={Description of Game 2},
    user1={https://boardgamegeek.com/boardgame/123456},
    user2={https://www.publisher1.com},
}

\newglossaryentry{gloomhaven}{
    type=shortcuts,
    name={AAA},
    description={Description of Game 1},
    user1={https://boardgamegeek.com/boardgame/123456},
}



\newglossaryentry{api}{
    type=shortcuts,
    name={Application Interface (API)},
    text={API},
    description={Většinou standardizované komunikační rozhraní mezi různými aplikacemi},
    user1={https://boardgamegeek.com/boardgame/123456},
    user2={https://www.publisher1.com},
}


\newglossaryentry{rest}{
    type=shortcuts,
    name={REST},
    description={representational state transfer},
    user1={https://www.google.com},
    user2={https://www.seznam.cz},
}
\documentclass[czech,bachelor]{../../shared/diploma}

\usepackage[autostyle=true,czech=quotes]{csquotes} % korektni sazba uvozovek, podpora pro balik biblatex
\usepackage[backend=biber, style=iso-numeric, alldates=iso]{biblatex} % bibliografie
\usepackage{dcolumn} % sloupce tabulky s ciselnymi hodnotami
\usepackage{subfig} % makra pro "pod-obrázky" a "pod-tabulky"
\usepackage{float} % lepsi umistovani obrazku (H)
\usepackage{xparse} % lepsi commandy s optional params
\usepackage{hyperref} % vkladani hypertextovych odkazu
\usepackage{enumitem} % lepsi enumerate a descrition and so on
\usepackage{wrapfig} % obrazky obtekané textem
\usepackage{latexsym} % symboly
% minted things
\immediate\write18{echo $ROAD > .ROAD.tex}
\immediate\write18{echo $ROAD "tohle je ten nejhorší hack co jsem kdy udělal"} 
\input{.ROAD}
\usepackage[outputdir=\ROAD]{minted} %
%\usepackage{minted}
\setminted{fontsize=\small, baselinestretch=1, frame=lines, framesep=8pt, linenos}
\renewcommand\listingscaption{Výpis}
\renewcommand\listoflistingscaption{Seznam výpisů zdrojového kódu}

\setcounter{tocdepth}{1}

% dark theme enable
%\usepackage{xcolor} \pagecolor[rgb]{0,0,0} \color[rgb]{1,1,1}

% Zadame pozadovane vstupy pro generovani titulnich stran.
\ThesisAuthor{Martin Korotwitschka}
\ThesisSupervisor{Ing. Radoslav Fasuga, Ph.D.}
\CzechThesisTitle{Tvorba API výpravné evoluční hry}
\EnglishThesisTitle{API Design for the Narrative Evolution Game}
\SubmissionYear{2024}
\ThesisAssignmentFileName{../specification.pdf}

% Pokud nechceme nikomu dekovat makro zapoznámkujeme.
\Acknowledgement{V první řadě bych chtěl poděkvoat všem členům týmu za aktivní spolupráci a zkušenosti získané na tomto projektu. Také bych chtěl poděkovat vedoucímu práce Ing. Radoslavu Fasugovi, Ph.D. za jeho vedení a rady. Dále svojí rodině a přátelům, bez kterých by bylo mé studium obtížné.}
\CzechAbstract{Tato práce se zabývá tvorbou aplikačního rozhranní pro hybridní výpravnou hru, které bude poskytovat koncové body pro administrativní rozhranní a pro rozhranní uživatele. Tato práce se nejprve zabývá analýzou možných způsobů řešení a implementace aplikačního rozhranní. Jako finální produkt by mělo vzniknout aplikační rozhranní které bude poskytovat koncové body ostatním spoluřešitelům. Společně s tímto aplikačním rozhranním by měla vzniknout i příručka popisující koncové body.}
\CzechKeywords{API; REST; RESTful; SOAP; GraphQL; JSON; HTTP; Databáze; výpravná hra; vývoj hry; Java; Spring Boot; }
\EnglishAbstract{This thesis deals with the creation of an application interface for a hybrid narrative game that will provide endpoints for the administrative interface and for the user interface. This thesis first discusses the analysis of possible ways of designing and implementing the application interface. The final product should be an application interface that will provide endpoints to other co-researchers. Together with this application interface, a manual describing the endpoints should be produced.

Translated with DeepL.com (free version)}
\EnglishKeywords{API; REST; RESTful; SOAP; GraphQL; JSON; HTTP; Database; adventure game; game development; Java; Spring Boot;}

%bib resource
\addbibresource{resources/sauce.bib}
\addbibresource{resources/impl.bib}


% Commandy pro snadnější psaní
\NewDocumentCommand{\chapterref}{mO{}}{(\hyperref[#1]{Kapitola \ref*{#1}#2})}
\NewDocumentCommand{\sectionref}{mO{}}{(\hyperref[#1]{Sekce \ref*{#1}#2})}
\NewDocumentCommand{\subsectionref}{mO{}}{(\hyperref[#1]{Podsekce \ref*{#1}#2})}
\NewDocumentCommand{\tableref}{mO{}}{(\hyperref[#1]{Tabulka \ref*{#1}#2})}
\NewDocumentCommand{\coderef}{mO{}}{(\hyperref[#1]{Výpis \ref*{#1}#2})}
\NewDocumentCommand{\figureref}{mO{}}{(\hyperref[#1]{Obrázek \ref*{#1}#2})}
\NewDocumentCommand{\listingref}{mO{}}{(\hyperref[#1]{Výpis \ref*{#1}#2})}


% Glossary stuff
\usepackage{glossaries}

\makenoidxglossaries
\loadglsentries{resources/glossary/glosset}

% Zacatek dokumentu
\begin{document}
\pagenumbering{roman}
% Nechame vysazet titulni strany.
\MakeTitlePages
\listoflistings
\clearpage
\listoffigures
\clearpage
\listoftables
\clearpage

% A nasleduje text zaverecne prace.
\pagenumbering{arabic}
\chapter{Úvod}

to je konec tohle
\chapter{Historie a vývoj API}
\cite{apiHistory} Myšlenka API se objevila již ve čtyřicátých letech minulého století, kdy Maurice Wilkes a David Wheeler vytvořili modulární softwarovou knihovnu pro počítač EDSAC, \cite{enwiki:1215865067} k němuž přiložili i dokument, který bychom v dnešní době nazvali API dokumentací. \cite{Wilkes1951} I když se tato dokumentace v mnoha ohledech od těch dnešních liší, tak je považována za jakýsi prototyp API dokumentací.

Poprvé se termín API v technické literatuře vyskytl v roce 1968 v článku \textit{Datové struktury a techniky pro vzdálenou počítačovou grafiku}.\cite{art:fall_joint_computer_conference} Pár let poté se tento termín začal používat i v oblasti databází.\cite{art:comparasion_of_the_application_programming_interfaces} Výsledkem je definice API z roku 1990 vytvořená Carlem Malamudem: \textit{API = soubor služeb, které má programátor k dispozici pro provádění určitých úkolů.}\footnote[1]{Původní anglická definice zní takto: \textit{API = a set of services available to a programmer for performing certain tasks} }

Další fáze vývoje API jakožto rozhraní založeného na internetových protokolech byla disertační práce pana Roy Fieldinga\cite{phd:api_web_services}, kde definoval REST jako tzv. \textit{go-to protokol}, který umožňuje standardizovanou komunikaci mezi zařízeními na internetu. Fielding představil nové pojetí API jakožto Web Programming interface, oproti tradičnímu pojetí, kde bylo založeno na knihovně. Tato disertační práce je základ pro dnešní pojetí webových API.

\endinput
\chapter{Formáty využívající se v API}
V této kapitole budou popsány především dva základní formáty dat které jsou stále aktivní -- JSON a XML.


\subsection*{Serializace a deserializace} %TODO cote https://www.baeldung.com/cs/serialization-deserialization
Serializace a deserializace jsou důležité koncepty v programování. Umožňují ukládat, přenášet a znovu sestavit data. Používají se celou řadu věcí, jako je ukládání objektů do databáze, posílání dat po síti a nebo pro účely mezipaměťi.

Objekt má 3 základní vlastnosti: identitu, stav a chování. Stav reprezentuje jednotlivá data objektu.
\textbf{Serializace} je proces převádění stavu objektů do proudu bytů co mohou být kdekoliv uloženy nebo poslány. Tento proud může být poté zase rekonstruován do původního objektu. Pro serializaci si však muíme vybrat formát dat. Jako třeba JSON nebo XML. Ovšem používá se i binární reprezentace dat. Tato reprezentace se často využívá pro výkonnostní potřeby, protože jsou typicky rychlejší na zápis a čtení. Jejich nevýhoda ale je, že nejsou člověkem čitelné.

\textbf{Deserializace} je opačný proces od serializace. Tedy převedení proudu bytů zpátky do objektu.

Jednou z nevýhod serializace a deserializace jsou vysoké nároky na výkon. Můžeto trvat nezanedbatelné množství času. Zvlášť u velkých objektů. Za zmínku stojí i fakt, že ne všechny objekty mohou být serializovány. Jako třeba sockety nebo file handlery.


\section{JSON}
JavaScript Object Notation neboli JSON je formát, který je odvozen z Javascriptu, nicméně mnoho dnešních jazyků už má serializaci do JSONu zabudovanou interně. Jedná se o textový formát zápisu objektů, který je dobře čitelný člověkem. Ukládá data do párů \textbf{key:value} kdy key je klíč a value hodnota, která je pod ním uložená, obvykle může jít o číslo, textový řetězec, pole nebo i další objekt. Znaky v JSONu musí být v kódování UTF-8, ale formát podporuje i speciální znaky pokud jsou escaped, jako příklad můžeme uvést znaky \verb |\uD83D\uDE10| nebo \textit{neutral-face}. JSON je využíván primárně k výměně dat mezi webovými aplikacemi a servery, ale dá se používat i pro jednoduché databáze. Má prostá ale zato striktní pravidla a tudíž je jednoduché zkontrolovat jeho správnost. Jeden z jeho nedostatků je, že nemá podporu komentářů oproti XML, které tuto podporu má. %\textbf{ TODO odkaz někde na iso normu https://www.iso.org/standard/71616.html} %TODO bibtech na iso normu plus dát emoji 😐 třeba xetex ale to se pak celé rozbije


\subsection{Pravidla}
\textbf{Key} neboli klíč daného objektu je vždy string a reprezentuje název určitého atributu. \textbf{Value} představuje hodnotu, kterou tento atribut nabývá a může být datového typu text, číslo, logická hodnota, null, další objekt či pole. Jednotlivé atributy jsou vždy odděleny čárkou. Jako příklad je uveden JSON objekt z API modelové hry, který popisuje hráče, jeho vlastnosti a jeho rasu, reprezentovanou jako vnořený objekt.

\begin{listing}[H]
  \inputminted{json}{resources/code/standards/player.json}
  \caption{Příklad JSON objektu}
  \label{code:json_player}
\end{listing}

V tomto JSONu můžeme vidět všechno, s čím se u tohoto datového formátu můžeme setkat. Jako první atribut, tento objekt má \verb|"id":4|, kde id je klíč a 4 je číslo -- to značí, že tento objekt má svůj unikátní identifikátor 4.

Na 6 řádku můžeme vidět atribut \verb|"title":"Elf"|. Value je v tomto případě text, který poznáme podle zaobalení uvozovek. Tento řádek v kontextu celého objektu značí, že se jedná o rasu postavy, která má název \textit{Elf}.

Atribut, který tento text zaobaluje, je \texttt{"race"}. Jedná se o příklad objektu, v tomto případě rasy, která má své vlastnosti reprezentované právě tímto objektem.

V atributu \texttt{"effects"} se nachází pole objektů. V tomto případě to jsou objekty obsahující klíč s id rasy a jejím efektem, ke kterému je přidána také informace o tom, od jaké úrovně je tento efekt zpřístupněn. Podobně tak na řádku 26 je jako value klíče \texttt{inventory} pole textových řetězců znázorňující předměty, které hráč vlastní.

Dále si můžeme povšimnout atributu \texttt{clazz} na řádku 3, který má hodnotu \texttt{null}. To nám značí, že tento konkrétní klíč u objektu zatím nic neobsahuje. Může se stát že jen nebyl přiřazen a v budoucnu nějakou hodnotu dostane.

A nakonec zde máme na řádku 29 atribut \texttt{dead} s hodnotou \texttt{false}. Tato hodnota značí logickou hodnotu, která má datový typ boolean. V našem případě se jedná o vlastnost, která nám říká, zda je hráč mrtvý (\texttt{true}) či nikoli (\texttt{false}).


\section{XML}
Extensible Markup Language je jazyk primárně určený na serializaci a přenášení dat, který je podobně jako JSON čitelný i člověkem. Základní stavební blok je node, atributy jsou uloženy do párů kterým říkáme tagy. Za pomocí své deklarace podporuje určení kódování a různé verze. Díky XML Schema definition také podporuje celou řadu datových typů a oproti JSONu podporuje i komentáře. Standardy spravuje společnost W3C. %\textbf{fest fajny odkaz na w3c specifikaci do bibtex} %TODO bibtech na w3c


\subsection{Pravidla}
Na začátku dokumentu je vždycky XML deklarace, která určuje, o jakou verzi se jedná a jaké mají znaky kódování.
Samotná data jsou zaznamenána pomocí párových tagů, dohromady nazývaným \textbf{element}, které se zapisují jako počáteční (\texttt{<character>}) a ukončovací (\texttt{</character>}) tag, kde \textit{character} značí název atributu, podobně jako \textbf{key} u JSONu. Hodnota atributu se píše buď mezi počáteční a ukončovací tag, nebo přímo do tagu samotného (\texttt{<character id=4>}). Tzv. \textit{procesory} analyzují XML dokumenty a posílají dále strukturovaná data aplikaci, která se využívá. Procesory můžou být jak validující tak nevalidující, přičemž validující musí nalezenou chybu nahlásit ale pořád mohou v parsování pokračovat.

\begin{listing}[H]
  \inputminted{xml}{resources/code/standards/player.xml}
  \caption{Příklad XML dokumentu i se schématem}
  \label{code:xml_player}
\end{listing}

\begin{figure}[H]
  \centering
  \includegraphics[width=0.7\textwidth]{figures/type-hierarchy.png}
  \caption{Hierarchie datových typů v XML schema}%\cite{british_museum_2021}}
  \label{fig:xml_datatypes}
\end{figure}
%TODO odkaz na obrázek všech možných datových typů https://www.w3.org/TR/xmlschema-2/type-hierarchy.gif a přímo ta stránka https://www.w3.org/TR/xmlschema-2/#built-in-datatypes


%TODO nějaký odkaz pěkný nebo tak něco na 
Tento XML dokument reprezentuje stejnou strukturu jako JSON popsaný výše. Oproti němu je však rozšířen o schéma, ve kterém můžeme určit jeden z mnoha datových typů \ref{fig:xml_datatypes} pro každý element (řádek 3) či atribut (řádek 6 a 7). Když není schéma specifikováno, všechny datové typy jou brány za text. Velká změna oproti JSONu také spočívá v tom, že víme, o jaký objekt se jedná (řádek 23 a 48) -- místo obyčejných složených závorek zde máme element se jménem \textit{character}, což je informace, kterou JSON neposkytuje.


\section{Shrnutí}
Teď byly představeny případné technologie co bychom mohli použít jako formát pro serializaci a deserializaci. Jak tyto technologie fungují, jak vypadají a jejich výhody a nevýhody.

\begin{table}[h]
  \centering
  \begin{tabular}{|l|c|c|c|c|}
    \hline
           & Čitelnost & Jednoduchost & Rychlost & Skladnost \\
    \hline
    JSON   & 1         & 1            & 2        & 3         \\
    \hline
    XML    & 2         & 3            & 4        & 4         \\
    \hline
    Binary & 5         & 4            & 1        & 1         \\
    \hline
  \end{tabular}
  \caption{Porovnání JSON, XML a Binary }
  \label{tab:formats_comparison}
\end{table}

\tableref{tab:formats_comparison} nám srovnává různé technologie podle našich případných požadavků. \footnote[1]{Známkování jsem prováděl čistě podle vlastního uvážení co bude pro naše použití nejvhodnější} \footnote[2]{Známkování je od 1 do 5 kdy 1 představuje souhlas a 5 nesouhlas} Pro naše účely je potřeba něco jednoduchého a srozumitelného, přiměřeně rychlého. Tedy nejlépe z toho vyplývá JSON. XML je pro naše účely moc komplexní a většinu vlastností nevyužijeme. Binární formát je sice rychlý ale špatně se odhalují chyby a je zde horší universálnost serializeru a deserializeru mezi jazyky.


\endinput
\chapter{Standardy využívané pro tvorbu API}
V této kapitole si porovnáme a představíme různé architektury které se používají pro tvorbu \gls{api}.
Hlavní nástroje co si představíme jsou na trhu a v nějaké míře se používají jsou REST, Webhooky a SOAP

\subsection*{Úvod}
\gls{api} je universální komunikační rozhraní mezi aplikacemi, které můžeme použít pro více druhů koncových aplikací. Vlastně to je soubor definicí, protokolů a občas i nepsaných pravidel.

\section{REST}
\gls{rest} což můžeme přeložit jako reprezentační stavový přenos. Je to nejčastěji používaná architektura API. REST byl původně vytvořen jako vodítko jak komunikovat na velké síti jako je Internet. Popularita REST API je z důvodu jednoduchosti implementace a provádění změn. Taky je velice výkonné a přehledné i na velkých projektech.
Webové služby které využívají architekturu REST pro API jsou nazývána RESTful webové služby. Pojem RESTful API většinou označují RESTful web API. Nicméně tyto dva pojmy se mohou zaměňovat nezávisle na sobě.

Níže si popíšeme základní principy REST API

\subsection{Uniform interface}
Klíčová vlastnost RESTful služeb. Indikuje, že server předává a přijímá data v určitém specificky strukturovaném formátu. Nejčastěji v JSON.

Jednotné rozhraní by mělo dodržovat tyto pravidla\
\begin{enumerate}
    \item Požadavky by měly identifikovat zdroje.
    \item Klient je schopen z přijmutých dat provést úpravu nebo smazání dat.
    \item KLient je schopen přijmout metadata o tom jaký typ zprávy jde a tím pádem jak tuto zprávu zpracovat.
    \item Klient může dostat informace o ostatních datech které se vážou k tomuto požadavku. (HATEOAS)
\end{enumerate}

\subsection{Stateless (bezstavový)}
Komunikační metoda kdy server obslouží každého klienta zvlášť a nezávisle na jeho předchozích akcí. Klienti mohou dávat požadavky nezávisle na pořadí. Server si tudíž neuchovává data o klientech.

\subsection{Layered system}
Styl architektury kdy klient se může připojit i k jiným zprostředkovatelům a pořád dostane odpověď od serveru. Servery taktéž se mohou dotazovat na jiné servery při zpracovávání požadavků od klienta. Tímto způsobem můžeme mít více vrstev jako je bezpečnostní vrstva, aplikační vrstva či business logika.
Tyto vrstvy jou stále pro klienta neviditelné.

\subsection{Cacheability}
Velká výhoda je že RESTful webové služby podporují cache. Vyžaduje aby Server tyto data označil jako cacheable či non-cacheable. Při stejném dotazu klienta vícekrát za nějaký krátký čas se použijí již stáhnutá data. Například pro obrázky. Tím se docílí o hodně rychlejší odezvy.

\subsection{Code on demand}
Dnes již málo používané, REST API může vrátit kód co má klient vykonat. Tím může server rozšířit klienta o funkcionalitu. Např. při validaci formuláře se ihned může zobrazit chybová hláška.



\subsection{Jak funguje RESTful API} % TODO reformat this bcs its not the same as rest
RESTful API funguje na protokolu HTTP/S. Nejdříve klient pošle request serveru podle dokumentace konkrétní API. Server poté zkontroluje zda je klient oprávněn tuto operaci provést. Poté zpracuje požadavek a pošle odpověď s příslušným stavovým kódem.

\subsection{Co obsahuje request na RESTful API} % TODO reformat this bcs its not the same as rest
Požadavky na RESTful API musí obsahovat následující
\subsubsection{URI}
Unique resource identifier je unikátní identifikátor pro určité data. V RESTful API nejčastěji URL, které se taktéž říká endpoint. URL specifikuje cestu k daným datům.
\subsubsection{Metoda}
Jak již bylo předtím zmíněno, RESTful API typicky používá HTTP/S protokol. V tomto protokolu máme několik možností metod.

\begin{description}
    \item[GET] - Požadavek pro získání dat na základě parametrů v URL. Opakované volání vrátí vždy stejný výsledek.
    \item[POST] - Požadavek pro vložení kompletně nových dat. Data pro vložení se zároveň vloží do message body. Zde se s opakovaným voláním vloží zase znovu ta stejná data.
    \item[PUT] - Požadavek pro úpravu dat. Data pro přepsání dat jsou vložena do message body. Opakované volání vrátí vždy stejný výsledek.
    \item[DELETE] - Požadavek pro smazání dat.
    \item[PATCH] - Požadavek pro částečnou úpravu dat. Není třeba posílat celý objekt.
\end{description}

\subsection{Co obsahuje odpověď od serveru}
REST principy vyžadují aby v odpovědi byli obsaženy tyto věci.
\subsubsection{Stavové kódy}
Stavovými kódy lehce zjistíme jak daný požadavek dopadl. Zda-li úspěšně nebo se vyskytla chyba. Proto je důležité je správně používat. Stav je trojmístné číslo a podle počátečního čísla poznáme typ odpovědi. Kód začínající 2xx značí úspěch, 4xx chybu na straně klienta, 5xx chybu na straně serveru a v neposlední řadě 3xx což označuje přesměrování URL.
Nejčastěji využívané kódy jsou tyto:

\begin{description}
    \item[200 OK] - Vše proběhlo v pořádku.
    \item[201 Created] - vše proběhlo v pořádku při požadavku POST (data byla zapsána).
    \item[400 Bad Request] - Server nepřijímá tyto data. Chyba na straně uživatele.
    \item[401 Unauthorized] - Klient nemá potřebná oprávnění pro vykonání této akce.
    \item[404 Not Found] - Adresa či data na které se klient dotazuje neexistují.
    \item[500 Internal Server Error] - Obecná chyba na straně serveru.
\end{description}

\subsubsection{Message body}
Ať už požadavek či odpověď, mohou obsahovat další data. Nikde není specifikováno v jakém formátu budou. Toto rozhodnutí je čistě na programátorech. Ale nejčastěji se používá JSON nebo XML. Jedná se o čistě textovou reprezentaci dat.

Například při požadavku GET pro uživatele s ID 1 server vrátí v message body JSON
\begin{verbatim}
	{"name":"Jožko Mrkvička", "age":30}
\end{verbatim}
Tento objekt reprezentuje uživatele který se jmenuje Jožko Mrkvička a je mu 30 let.

\subsubsection{Headers}
V hlavičkách jsou přídavné informace. Ať už kódování body, datum a čas, typ obsahu v body. Nebo na straně klienta autorizaci jako třeba session key.

\subsection{Shrnutí}
Dnes nejpoužívanější architektura pro API. Je velice flexibilní, jednoduchý a programátor si může sám určit který formát pro přenos dat bude používat. Je také velice intuitivní a samopopisný takže je jeho učící křivka rychlá. Nicméně kvůli textové podobě přenosu dat může být pomalejší než architektury používající binární soubory ale díky podpoře cache se může optimalizovat.

\section{SOAP}
SOAP neboli Simple Object Access Protocol je další z mnoha architektur API. Taktéž využívá aplikační vrstvu HTTP/S ale klidně může využívat jiné aplikační protokoly. Využívá pro přenos dat XML formát.

Mám se tady rozepsat o tom co je to XML? u JSONU jsem to taky neudělal. Možná tu i rozepíšu JSON.

\subsection{Charakteristika}
SOAP má tři základní stavební kameny které definují zprávy.
\begin{description}
    \item[Envelope] Zapouzdřuje celou právu. Určuje jakou strukturu má zpráva mít a jak ji zpracovat
    \item[Header] Obsahuje informace o zprávě. Například autentizační údaje
    \item[Body] Samotná data. Obsahují dotazující informace a informace s odpovědí.
    \item[(Fault message)] Může a nemusí obsahovat. Obsahuje kód chyby, actor, string a detail
\end{description}

Z pohledu klienta je podobné jako RESTful architektura. Klient vygeneruje požadavek ve formátu XML. Poté vygenerovaný SOAP požadavek klient pošle na SOAP server a ten následně vyvolá požadovanou aplikaci běžící na serveru. Odpověď z aplikace s požadovanými daty, parametry a hodnotami přepošle nejdříve SOAP request handleru a následně je odpověď poslána klientovi.

\subsection{Výhody}
\begin{description}
    \item[Nezávislost na platformě] SOAP může běžet na jakémkoli operačním systému či síťovém protokolu. Což umožňuje komunikaci mezi různými jazyky jak na Windows tak i Linux.
    \item[HTTP protokol] Primárně se používá HTTP protokol. To je výhoda že se nemusí upravovat firewall. I když funguje na jakémkoli ale tam by se mohlo stát že by se musela upravovat komunikační infrastruktura.
    \item[Zabezpečení] má vlastní rozšíření Web Services security. Podporuje funkce jako x.509 certifikáty, vlastně-definované tokeny, Kerberos tickety a uživatelské ID/heslo pověření. Taktéž možnost HTTPS přidává vrstvu šifrování.
\end{description}

\subsection{Nevýhody}
\begin{description}
    \item[Rychlost] kvůli vysokému zabudovanému zabezpečení a kvůli serializaci do XML je tento protokol velice pomalý s porovnáním s ostatními
    \item[Složitost] vzhledem k podpoře více protokolů se nemůže využít funkcí jednotlivých protokolů jako třeba u RESTful cachování a Uniform interface
\end{description}

\subsection{Shrnutí}
Vzhledem k masivnosti a rychlosti tohoto protokolu se od něj upouští aby se optimalizovala rychlost tvořených API. To ale neznamená že se už nepoužívá. Díky jeho bezpečnosti je stále využíván bankami, E-komercí, ve zdravotnicví a všude kde je primární apel na bezpečnost.

\section{GraphQL}
GraphQL byl vyvinut Facebookem dnes již Metou. Je to open-source query language a runtime určený pro API. GraphQL poskytuje deklarativní získávání dat kdy si klient přesně určí jaká data potřebuje. Tím pádem je velice šetrný k datům a není třeba mít více endpointů pro různě rozdělená či vyplněná data. Díky tomu že GraphQL může načítat data z různých zdrojů, není závislý na konkrétní databázi či úložišti.

\subsection{Design}
Je postaven kolem modelu \textit{dostanu přesně to o co si řeknu} bez nadměrného nebo nedostatečného načítání. To celkově zrychluje přenos dat. Využívá typový systém pro definici prostředků které nazýváme schéma. při každém požadavku je query zkontrolována oproti schématu a poté vykonána. Server potom vrátí data ve stejném formátu jako byla query typicky jako JSON.

\begin{listing}[h!]
    \inputminted[]{ts}{resources/code/playertype.gql}
    \caption{Příklad schématu v GraphQL}
    \label{code:gql_type}
\end{listing}

\begin{listing}[h!]
    \inputminted[]{graphql}{resources/code/playerquery.gql}
    \caption{Příklad query v GraphQL}
    \label{code:gql_querry}
\end{listing}

\subsection{Datové typy}
aaaaaaaaaaaaaaaaaaaaaaaaaaaaaaaaaaaaaaaaaaaaaaaaaaaaaaaaaaaaaaaaaaaaaaaaaaaaaaaaaaaaaaaaaaaaaaaaaaaaaaaaaaaaaaaaaaaaaaaaaaa
je jich dost a je tu hoafo co popisovat uvidíme jak dlouhé tohle celkově bude


\begin{listing}[h]
    \inputminted[]{graphql}{resources/code/types.example.gql}
    \caption{Příklady datových typů}
    \label{code:gql_datatypes}
\end{listing}

\begin{description}
    \item[Type system] - Základ typového systému je Query. Ta určuje jaké objekty mohou být získány (\coderef{code:gql_type} řádek 29). Vlastnosti ve výchozím stavu mohou nabývat hodnotu \texttt{null}. Z datový typ se může dát vykřičník abychom označili že tato hodnota nikdy nebude \texttt{null}. \textbf{dále datové typy zde}
    \item[Queries] - Query přesně definuje jaká data klient potřebuje. Za pomocí \coderef{code:gql_querry} dostaneme téměř stejný výsledek jako \coderef{code:json_player} akorát celý JSON bude zabalený do \texttt{"characters": {}}
    \item[Mutations] - Dovoluje pozměnění, smazání, nebo přidání dat ze strany klienta. Po provedení GraphQL vrátí upravená data. Taktéž definuje jaký tvar vrácená data budou mít. \coderef{code:gql_datatypes} řádek whatever kolik to bude.
    \item[Subscriptions] - Podpora posílání aktualizací v reálném čase. Dotaz je velice podobný jako u query (\coderef{code:gql_datatypes} řádek subsection)
\end{description}

\section{Porovnání a shrnutí} %TODO předělat tohle jsou jen myšlenky co tu dát
\textbf{předělat tohle jsou jen myšlenky co bych měl říct}

Porovnání dělám na základě vlastních úsudků co bude po naše účely nejlepší styl API. Nepředpokládáme nějaké velké vytížení co se dotazů na API týče. Taktéž potřebujeme něco poměrně jednoduchého a lehce pochopitelného. Zabezpečení není priorita protože bez fyzické hry je API nevyužitelné. Taktéž když budeme potřebovat protivníky nebo hrací plán tak budeme chtít vědět vše o nich a ne jen část dat. Dá se předpokládat že pokud by hra byla v komerci úspěšná tak odhadujeme 1.000.000 kopií. -> 10000per sec? takže CDN a cajk asi idk.


\endinput
\chapter{Zabezpečení a autentizace}
%TODO hoří má panenko kurva
V této kapitole bla bla bla bla si projdeme základní autentizační protokoly a standardy.
Pro jednoduchost se zaměříme na standardy v OpenAPI ve verzi 3.0.
Autorizaci používáme pro prevenci zneužití a kontrolu nad tím kdo jak API používá a jak často.

\section{API key}
%odkaz na api key zdroj https://swagger.io/docs/specification/authentication/api-keys/
API klíč je unikátní token který umožňuje uživateli se autorizovat. Na základě tohoto klíče se přidělí oprávnění. Mohou být poslády jako query param  \texttt{GET /something?key=rea11ysecr5tApIKeY} nebo v hlavičce requestu či jako Cookie \texttt{X-KEY: rea11ysecr5tApIKeY}. API klíč by měl znát jen server a daný klient. Takováto autentizace je brána jako bezpečná pouze při použití dalších bezpečnostních prvků jako je třeba HTTPS/SSL.

Nejčastější použití API klíče je hlavně pro blokování anonymních požadavků. To vyfiltruje případný škodlivý provoz na API. Dále díky jeho jedoduchosti se často používá mezi IoT zařízeními.

\subsection{Popis API klíče}
\begin{listing}[ht]
    \inputminted[]{yaml}{resources/code/security/openapi-key.yml}
    \caption{OpenAPI 3.0 definice}
    \label{code:api_key}
\end{listing}

V tomto příkladě jsme si definovali název API key \texttt{X-API-key} který bude v headeru (\ref{code:api_key} řádek 8). Dále s tím můžeme pracovat jako s \texttt{ApiKeyAuth} a dávat jej do všech ostatních definic endpointů pokud by bylo třeba specifikovat (\ref{code:api_key} řádek 20,21). Jinak za pomocí \ref{code:api_key} řádek 11,12 aplikujeme autorizaci na všechny operace.


Můžeme mít i více API klíčů. Například pro specifikování uživatele a specifikování aplikace.

Při nevalidním nebo chybějícím klíči můžeme vrátit chybový kód 401 který značí neoprávněný přístup. Můžeme si ho definovat takto \ref{code:api_key} řádek 11-16.

\section{OAuth}



\chapter{Analýza}
V této kapitole bude probrána specifikace, požadavky na API a také požadavky na celkové fungování hry. Část je také věnována analýze již existujících herních API.

Cílem této práce je vytvořit zdokumentované jednotné API, které bude používáno jak uživatelským prostředím, tak i administrátorským rozhraním.

\section{Analýza již existujících herních API}

Tato část se věnuje představení již existujících online her s evolucí herních situací, která se ukládá mezi sezeními. Těchto her ovšem není mnoho -- většina webových online her nenutí uživatele k přihlášení a také nepodporuje postup či evoluce situací. Takovéto prémiové hry mívají většinou spíše formu desktopových aplikací a v takovém případě by se bohužel špatně analyzovala odcházející a přicházející komunikace.
Byla mi ovšem doporučena jedna hra, která je vytvořená přímo pro hraní za pomocí API, proto jsem se rozhodl tuto hru analyzovat a zjistit, jaké funkce a požadavky by mělo API pro tuto hru splňovat.

\subsection{SpaceTraders API}\label{sub:SpaceTraders}
SpaceTraders je hra založená na REST API, ve které hráči kontrolují a rozšiřují své flotily vesmírných lodí a za jejich pomocí objevují, obchodují a probíjí si vlastní cestu skrz galaxii. Hra je určena pro nadšence, kteří jsou vybízeni vytvořit si vlastní frontend a případně herní mechaniky automatizovat přes jakýkoli jazyk, což může sloužit i jako příjemný nástoj, jak se naučit práci s API nebo nový programovací jazyk. API je zdokumentováno za pomocí technologií OpenAPI a Stoplight. \cite[]{spacetraders}

Pro dotazování hra využívá jak parametrů dotazu tak proměnných v dotazované URL\@.

Token se vkládá do hlavičky ve formátu \texttt{'Authorization: Bearer INSERT\_TOKEN\_HERE'}. Tento token má formu zakódovaného JWT \sectionref{sec:jwt} objektu pomocí RS256.
Jeho dekódovaný text je zobrazen na obrázku \ref{fig:jwt_spacetraders}.

\begin{figure}[!ht]
    \centering
    \includegraphics[width=0.5\textwidth]{figures/spaceTraders/jwt.png}
    \caption{Dekódovaný token ze hry SpaceTraders \cite[]{jwt_decoder}}
    \label{fig:jwt_spacetraders}
\end{figure}

Hra nejprve vyžaduje registraci přes endpoint \texttt{/v2/register}. Ten vrátí údaje o novém agentovi spolu s autentizačním tokenem \coderef{code:space_login}[, řádek 3], se kterým se uživatel bude dále ověřovat ve všech následujících požadavcích.

\begin{listing}[!ht]
    \inputminted[breaklines]{json}{resources/code/spaceTraders/login.jsonc}
    \caption{Odpověď na požadavek na registraci\protect\footnotemark}
    \label{code:space_login}
\end{listing}
\footnotetext{Velká část dat musela být pro přehlednost smazána}

Jako odpověď se zde používá formát JSON, který obsahuje především objekt \texttt{data} a poté dodatečné objekty jako třeba \texttt{meta}, ve kterých mohou být další informace jako stránkování. % TODO odkaz na stránkování
Status je použit v souladu s klasickým výkladem statusových kódů % TODO odkaz na ty pravidla někde v restku
a dále je obsahu rozšířen o konkrétní popis chyby v daném požadavku. Příklad je vidět na výpisu \ref{code:space_error}, kde je zobrazena chyba při požadavku na odlet na jinou planetu, neboť loď není na orbitě.


\begin{listing}[ht!]
    \inputminted[breaklines]{json}{resources/code/spaceTraders/error_response.jsonc}
    \caption{Výpis chyby při požadavku odletět na jinou planetu}
    \label{code:space_error}
\end{listing}

\section{Specifikace požadavků}
Pomocí výše provedené analýzy API existujících her, vyčlenění požadavků na funkcionality hry ze strany ostatních členů týmu a za pomocí analýzy dostupných nástrojů pro tvorbu API byly vytyčeny požadavky a funkce, které by mělo API modelové hry podporovat. Jedná se především o CRUD operace se základními objekty, jejich filtrování, stránkování a lazy load. API by taktéž mělo podporovat přihlašování a obranu před základními typy útoků jako je SQL injection, DDOS a DOS útok nebo neoprávněný přístup díky chybám v API.

API by taktéž mělo podporovat validaci všech vstupních dat (rozsahy vstupních hodnot, filtrace speciálních znaků, kontrola správného postupu operací při hraní hry) a mělo by mít odpovídající koncové body pro samotné hraní hry.

\subsection{Funkční požadavky}
Nyní si představíme funkční požadavky na API, které předložili ostatní členové týmu. Tyto požadavky se měnily a rozšiřovaly spolu s průběhem návrhu i implementace. Některé požadavky, které vzešly primárně z backoffice se využívají ve frontendu a případně i naopak.

\subsubsection*{Požadavky, které byly vyčleněny primárně ze strany backoffice}

\begin{enumerate}[label=\textbf{F\arabic*}:, leftmargin=*, align=left]
    \item \textbf{CRUD operace} -- Nad základními objekty, se kterými se bude často pracovat, a upravovat pomocí endpointů. Těmito objekty jsou \texttt{akce, efekty, předměty, charaktery a jejich vlastnosti, dobrodružství, kampaň, obchody, nepřátelé, překážky, lokace} a \texttt{části lokace}
    \item \textbf{Filtrování} -- Možnost vyhledat objekty podle vstupních parametrů u koncových bodů, které poskytují seznam objektů.
    \item \textbf{Lazy load} -- Způsob načítání dat, který umožňuje vracet pouze daný objekt bez jeho závislostí, případně vrácení pouze těch závislostí, které se určí. Výsledkem je rychlejší zpracování a menší objem přesunutých dat, když uživatel tyto závislosti nepotřebuje. Namísto objektu se tedy vrátí jen jeho identifikátor.
    \item \textbf{Stránkování} -- Další způsob předávání dat, který umožní jejich postupné zpracování, což vede ke zkrácení času potřebného k vyhodnocení požadavku jak v API tak ve zobrazovací části.
    \item \textbf{Caching} -- Již jednou zpracovaná data z databáze není třeba znovu získávat z databáze, pokud nedošlo ke změně. Tato funkcionalita umožní násobně rychlejší odezvu pro opakované získávání stejných dat.
    \item \textbf{Administrátorská práva} -- Ne všichni mohou mít přístup pro úpravu dat v databázi. Díky administrátorským přihlašovacím údajům a následnému tokenu se budou moci upravovat a vkládat data do databáze pouze s odpovídajícím ověřením.
    \item \textbf{Validace} -- Data vkládaná do databáze budou validována a případně vrátí chybovou hlášku, podle které bude možno snadno identifikovat chybu vstupních dat a následně ji opravit.
\end{enumerate}


\subsubsection*{Požadavky primárně ze strany uživatelského prostředí}

\begin{enumerate}[label=\textbf{F\arabic*}:, leftmargin=*, align=left]
    \item \textbf{Získávání objektů} -- Bude umožněno získávat jakékoliv objekty, které neobsahují herní data či jiné citlivé informace, přímo z databáze.
    \item \textbf{Podpora herního průběhu} -- Uživatel bude moci projít celým soubojem a interagovat s entitami v něm za pomocí řady validovaných a přehledně uspořádaných koncových bodů.
    \item \textbf{Zamezení zneužití} -- Postup operací v herním průběhu bude kontrolován tak, aby se zamezilo případnému zneužití nebo obcházení pravidel hry.
    \item \textbf{Přihlášení} -- Uživatel se bude moci přihlásit a získat token pro ověření v dalších požadavcích.
    \item \textbf{Herní data} -- Uživatel bude mít pod svým účtem uložený postup hry a bude moci pokračovat tam, kde skončil. Dále bude mít možnost vytvářet nové postavy pro kampaně a také nová dobrodružství.
    \item \textbf{Validace} -- Obsah vstupních dat bude validován a případně vrátí smysluplnou chybovou hlášku.
    \item \textbf{Obrázky} -- Bude možné získat obrázek z url adresy přiložené k objektu, případně v požadavku specifikovat jeho velikost.
\end{enumerate}



\subsection{Nefunkční požadavky}
Dále je důležité vyhradit si nefunkční požadavky. Jejich vznik je stejný jako požadavky funkční, byly sestrojovány postupně s vývojem na základě zkušeností a požadavků ostatních členů týmu.


\begin{enumerate}[label=\textbf{F\arabic*}:, leftmargin=*, align=left]
    \item \textbf{Rozdělení API na dvě části} -- Z důvodu spolupráce na API s jinými členy týmu, především herním systémem, který pro svůj chod využívá stejných modelů, bylo rozhodnuto, že herní logika i mapování bude v jednom projektu. Tomu tedy musí být přizpůsobena i spolupráce a podpůrné technologie.
    \item \textbf{Dokumentace} API bude zdokumentováno za pomocí OpenAPI a pro vizuální zobrazení koncových bodů bude použit Swagger, který zároveň poslouží jako skvělé ladící rozhraní.
    \item \textbf{Hosting} API bude stejně jako ostatní části projektu hostováno na veřejných serverech.
    \item \textbf{Přehlednost} Koncové body API by měly být samopopisující a snadno pochopitelné.
    \item \textbf{Standardizovanost} API se bude držet ověřených dobrých praktik z praxe a bude udržovat jednotnost a standardizovanost.
\end{enumerate}


\chapter{Návrh}
Tato kapitola se bude zabývat návrhem sestavy technologií které se budou využívat pro realizaci API.

\section{Model vývoje}
takové to vodopádnový model a tak
dev stack nebo tak něco idk nějaký sum up
\section{Spolupráce}
grafik co kdo dělal
git a issues a tak idk co všechno tu npsat
\section{To co je v DB}
fest fajne grafiky od db možá toolky co se používali pro modelování db




\section{Implementace}
Nyní když máme všechen návrh i analýzu hotovou, můžeme se pustit do implementace. V této kapitole se zaměříme na konkrétní implementaci API pro naši hru.

\subsection{Nastavení projektu}

Pokud máte IntelliJ IDEA ultimate, máte možnost vytvořit nový projekt Spring Boot. Pokud ne, můžete použít Spring Initializer, což je webová aplikace, která vám umožní vytvořit nový projekt Spring Boot.
Pokud nemáte tak můžete navštívit Spring Initializer \url{https://start.spring.io/} a přidejte následující závislosti do projektu:

\begin{itemize}
    \item Spring Web
    \item Spring Data JPA
    \item H2 Database
\end{itemize}

Můžete přidat do projektu i další velice užitečné závislosti.
\begin{itemize}
    \item Lombok - snižuje množství kódu, který musíte napsat
    \item Spring HATEOAS - umožňuje jednodušeji vytvářet url odkazy
\end{itemize}

Místo H2 databáze můžete použít jakoukoli jinou databázi, která je podporována Spring Boot. Ovšem H2 je velice jednoduchá databáze, která je pouze v paměti (H2 in-memory database), což je ideální pro demonstrování. Ovšem to nemění nic na implementaci API, která je stejná pro jakoukoli jinou databázi.

Změňte název projektu a poté vyberte "Generate". Stáhne se .zip soubor, který se musí rozbalit. Uvnitř najdete jednoduchý projekt založený na Maven, včetně souboru pom.xml (Poznámka: Můžete použít i Gradle, ale příklady v tomto příkladu budou založeny na Maven).

Spring Boot funguje s jakýmkoli IDE, můžete použít Eclipse, IntelliJ IDEA, Netbeans, atd.

Pro automaticky generovanou dokumentaci můžete přidat závislost Springdoc Swagger 2 (viz. \ref{code:swagger-dependency}). Tato závislost se přidává do souboru \texttt{pom.xml} v kořeni projektu.
Poté můžete spustit projekt a zobrazí se vám dokumentace na adrese \url{http://localhost:8080/swagger-ui/index.html}.

\begin{listing}[H]
    \begin{minted}{xml}
        <!-- https://mvnrepository.com/artifact/org.springdoc/springdoc-openapi-ui -->
        <dependency>
            <groupId>org.springdoc</groupId>
            <artifactId>springdoc-openapi-starter-webmvc-ui</artifactId>
            <version>2.2.0</version>
        </dependency>
    \end{minted}
    \caption{Přidání závislosti Springdoc Swagger 2}
    \label{code:swagger-dependency}
\end{listing}

\begin{figure}[H]
    \centering
    \includegraphics[width=0.9\textwidth]{../images/implementation/swagger.png}
    \caption{rozhraní Swagger}
    \label{fig:swagger-ui}
\end{figure}
\begin{figure}[H]
    \centering
    \includegraphics[width=0.9\textwidth]{../images/implementation/swaggger endpoint.png}
    \caption{Detail konkrétního endpointu}
    \label{fig:swagger-ui}
\end{figure}


\subsection{Vytvoření kostry}
Následně se podíváme na věci, které je třeba udělat pro úplné minimum funkčnosti API. Vytvoříme entitu, repository a načteme nějaké výchozí data a vytvoříme jednoduchý controller.

První věc, kterou musíme udělat, je vytvořit entitu. Entita je třída, která reprezentuje tabulku v databázi. V našem případě budeme mít entitu pro Akci a entitu pro Efekt akce.

\begin{listing}[H]
    \begin{minted}{java}
@Data
@Entity
@NoArgsConstructor
@AllArgsConstructor
@Builder(toBuilder = true)
public class Action {
    @Id @GeneratedValue(strategy = GenerationType.AUTO)
    private Integer id;

    private String name;

    @NonNull @Column(nullable = false)
    private String description;
}
    \end{minted}
    \caption{Entita Akce}
    \label{code:action-entity}
\end{listing}

\subsection*{Definice Třídy \texttt{Action}}

Třída \texttt{Action} je definována s využitím JPA a Lombok anotací pro efektivní práci s databází a snadnější manipulaci s kódem. Níže je detailní popis každé použité anotace:

\begin{itemize}
    \item \textbf{@Data} - Generuje metody getter, setter, \texttt{toString}, \texttt{equals} a \texttt{hashCode}. To zjednodušuje kód tím, že eliminuje potřebu explicitně je psát.
    \item \textbf{@Entity} - Označuje třídu jako entitu JPA, což znamená, že bude mapována do databázové tabulky. Název tabulky bude ve výchozím nastavení Action, pokud není uvedeno jinak.
    \item \textbf{@NoArgsConstructor} - Další anotace z Lomboku, která generuje prázdný konstruktor (konstruktor bez parametrů). To je často vyžadováno JPA pro interní použití.
    \item \textbf{@AllArgsConstructor} - Lombok anotace, která generuje konstruktor se všemi atributy třídy jako parametry. To usnadňuje vytváření instancí objektu s přednastavenými hodnotami.
    \item \textbf{@Builder} - Lombok anotace, která umožňuje použití návrhového vzoru builder pro tuto třídu. Tento vzor umožňuje sestavit objekt krok za krokem pomocí řetězení metod. toBuilder = true znamená, že lze získat builder z existující instance.
    \item \textbf{@Id} a \textbf{@GeneratedValue(strategy = GenerationType.AUTO)} - Specifikují, že pole \texttt{id} je primární klíč entity a jeho hodnota bude automaticky generována. \texttt{Generation.AUTO} znamená, že konkrétní strategie generování klíče je přenechána persistenčnímu poskytovateli, který může vybírat na základě typu databáze.
    \item \textbf{@NonNull} - Lombok anotace, která označuje, že daný atribut nesmí být null. Lombok vygeneruje kontrolu, která vyvolá výjimku, pokud je do konstruktoru předána null hodnota.
    \item \textbf{@Column(nullable = false)} - Anotace JPA, která říká, že sloupec v databázové tabulce odpovídající atributu description nesmí obsahovat null hodnoty. Tato specifikace je součástí definice schématu databáze. Taktéž pokud necháme pouze notaci \texttt{@Column}, tak to bude mít stejný efekt jako kdyby tam žádná notace nebyla.
\end{itemize}

Sice anotace Lomboku, jako \texttt{@Data}, \texttt{@NoArgsConstructor}, \texttt{@AllArgsConstructor}, a \texttt{@Builder}, v našem kódu přinášejí značnou úsporu času tím, že automaticky generují boilerplate kód, včetně metod gettery/settery a konstruktorů, není jejich použití nezbytné. Můžeme se rozhodnout napsat tyto funkce ručně.


Nyní když máme entitu tak můžeme vytvořit repository pro tuto entitu. Prosté deklarování rozhraní EmployeeRepository \ref{code:action-repository}, které rozšiřuje JpaRepository od Spring Data JPA, nám automaticky umožní:
\begin{itemize}
    \item Vytvořit nový záznam v databázi
    \item Aktualizovat záznam v databázi
    \item Smazat záznam z databáze
    \item Hledat záznamy (jeden, všechny, podle jednodušších či složitějších kritérií)
\end{itemize}

\begin{listing}[H]
    \begin{minted}{java}
    public interface ActionRepository extends JpaRepository<Action,Integer> {
}
    \end{minted}
    \caption{Interface ActionRepository}
    \label{code:action-repository}
\end{listing}


\subsection*{Data}
Nyní můžeme vytvořit nějaká data, která budou v databázi. Vytvoříme třídu DataLoader \ref{code:dataloader}, která bude implementovat CommandLineRunner. Tato třída bude spuštěna při startu aplikace a vytvoří pomocí dříve vytvořeného repository dvě entity.

\begin{listing}[H]
    \begin{minted}{java}
@Configuration
class LoadDatabase {
    private static final Logger log = LoggerFactory.getLogger(LoadDatabase.class);
    @Bean CommandLineRunner initDatabase(ActionRepository repository) {

        return args -> {
            log.info("Preloading " + repository.save(Action.builder().name("Punch").description("Punch the enemy").build()));
            log.info("Preloading " + repository.save(Action.builder().name("Kick").description("Kick the enemy").build()));
        };
    }
}
    \end{minted}
    \caption{DataLoader}
    \label{code:dataloader}
\end{listing}

\subsection{Přístupové body}

HTTP poslouží jako platforma, a abychom náš repozitář obalili webovou vrstvou, využíváme Spring MVC. Díky Spring Bootu můžeme vynechat psaní složitého infrastrukturního kódu, což nám umožňuje zaměřit se přímo na provádění akcí.

Ovšem předem si vytvoříme pomocnou třídu RestException, která nám umožní vytvářet výjimky s HTTP kódem a zprávou.
\subsubsection*{Výjimky}
\begin{listing}[H]
    \begin{minted}{java}
@Getter
@AllArgsConstructor
public class RestException extends RuntimeException {
    private String message;
    private HttpStatus code;

    public static RestException of(HttpStatus code, String message, Object... args) {
        return new RestException(String.format(message, args), code);
    }

    public String getMessage() {
        return "{ \"message\": \"" + message + "\" }";
    }

}
    \end{minted}
    \caption{RestException}
    \label{code:rest-exception}
\end{listing}

Pomocí této vlastní výjimky \ref{code:rest-exception} můžeme nastavit vlastní chybový kód a vlastní hlášku. Metoda of vytváří novou instanci RestException s formátovaným chybovým hlášením a HTTP statusem s JSON formátováním.

Nicméně když je tato výjimka zavolána, tak se použije toto dodatečné nastavení Spring MVC pro vykreslení chybového kódu a zprávy. Zde se ještě nastaví hlavičky HTTP odpovědi, aby byla zpráva ve formátu JSON.

\begin{listing}[H]
    \begin{minted}{java}
@ControllerAdvice
public class RestExceptionAdvice {

    @ExceptionHandler(RestException.class)
    ResponseEntity<String> employeeNotFoundHandler(RestException ex) {
        HttpHeaders headers = new HttpHeaders();
        headers.add("Content-Type", "application/json");
        return ResponseEntity.status(ex.getCode()).headers(headers).body(ex.getMessage());
    }
}
    \end{minted}
    \caption{Přídavné nastavení pro výjimky}
    \label{code:rest-exception-advice}
\end{listing}




\subsubsection*{Samotné přístupové body}

Ve třídě ActionController \ref{code:action-controller} poté vytvoříme metodu pro získání akce podle id.

\begin{listing}[H]
    \begin{minted}[linenos]{java}
@RequiredArgsConstructor
@RestController
public class ActionController {
        private final ActionRepository actionRepository;

        @GetMapping("/actions/{id}")
        ResponseEntity<?> getActionById(@PathVariable Integer id) {
            
            Action action = actionRepository.findById(id)
                .orElseThrow( () -> RestException.of(HttpStatus.NOT_FOUND, "Action with id %d not found", id));
            
            return ResponseEntity.ok(
                EntityModel.of(action,
                    linkTo(methodOn(ActionController.class)
                        .getActionById(id))
                        .withSelfRel()
                )
            );
        }
}
    \end{minted}
    \caption{ActionController}
    \label{code:action-controller}
\end{listing}

Anotace \texttt{@RestController} znamená, že data vrácená každou metodou budou přímo zapsána do těla HTTP odpovědi, namísto generování šablony.

\texttt{ActionRepository} je do kontroleru vkládán prostřednictvím konstruktoru, což umožňuje manipulaci s daty akcí.

V kontroleru je definována cesta pro jednu operaci, která odpovídají HTTP metodě GET, používá se anotace \texttt{@GetMapping}.

\texttt{ResponseEntity} je třída v rámci Spring Frameworku, která umožňuje reprezentovat kompletní HTTP odpověď, včetně obsahu, hlaviček a HTTP status kódu. Správné nastavení těchto status kódů je klíčové, neboť každý kód má specifický význam a ovlivňuje chování klienta.

V tomto příkladu je \texttt{ResponseEntity} využita k odeslání odpovědi, která obsahuje \texttt{EntityModel}. \texttt{EntityModel} je nástroj, který usnadňuje vytváření hypermediálních odkazů, což je základní princip HATEOAS u RESTful API.


ActionNotFoundException je výjimka používaná k signalizaci, že hledaná akce nebyla nalezena.

V této fázi už máme funkční kostru aplikace, která umožňuje získat akci podle id. Nyní přidáme crud operace pro akce.

\begin{listing}[H]
    \begin{minted}{java}
    @GetMapping("/actions")
    ResponseEntity<?> getAllActions() {

        List<EntityModel<Action>> employees = actionRepository.findAll().stream()
                .map(action -> EntityModel.of(action,
                        linkTo(methodOn(ActionController.class).getActionById(action.getId()))
                                .withSelfRel(),
                        linkTo(methodOn(ActionController.class).getAllActions())
                                .withRel("employees")))
                .toList();

        return ResponseEntity.ok(
                CollectionModel.of(employees,
                        linkTo(methodOn(ActionController.class).getAllActions()).withSelfRel()
                )
        );
    }
    \end{minted}
    \caption{Získání všech akcí}
    \label{code:get-all-actions}
\end{listing}

\begin{listing}[H]
    \begin{minted}{java}
    @PutMapping("/actions/{id}")
    ResponseEntity<?> updateAction(@PathVariable Integer id,@RequestBody  Action newAction) {
        Action action = actionRepository.findById(id)
                .map(a -> {
                    a.setName(newAction.getName());
                    a.setDescription(newAction.getDescription());
                    return actionRepository.save(a);
                })
                .orElseThrow(() -> RestException.of(HttpStatus.NOT_FOUND, "Action with id %d not found", id));


        return ResponseEntity.ok(
                EntityModel.of(action,
                        linkTo(methodOn(ActionController.class).getActionById(id)).withSelfRel(),
                        linkTo(methodOn(ActionController.class).getAllActions()).withRel("actions")
                )
        );
    }
    \end{minted}
    \caption{Aktualizace akce}
    \label{code:update-action}
\end{listing}


\begin{listing}[H]
    \begin{minted}{java}
     @PostMapping("/actions")
    ResponseEntity<?> createAction(@RequestBody Action newAction) {
        Action action = actionRepository.save(newAction);
        return ResponseEntity.created(
                    linkTo(methodOn(ActionController.class).getActionById(action.getId())).toUri())
                .body(EntityModel.of(action,
                            linkTo(methodOn(ActionController.class).getActionById(action.getId())).withSelfRel()
                )
        );
    }
    \end{minted}
    \caption{Vytvoření akce}
    \label{code:create-action}
\end{listing}


\begin{listing}[H]
    \begin{minted}{java}
    @DeleteMapping("/actions/{id}")
    ResponseEntity<?> deleteAction(@PathVariable Integer id) {
        actionRepository.deleteById(id);
        return ResponseEntity.noContent().build();
    }
    \end{minted}
    \caption{Smazání akce}
    \label{code:delete-action}
\end{listing}


Tímto jsme vytvořili crud operace. Není zde nic co by již nebylo zmíněno předtím.

\subsection{Přidání efektů}
Nyní přidáme efekty k akcím. Vytvoříme novou entitu Effect stejným způsobem jako předtím entitu Action a nový repositář pro efekty. Poté do dataLoaderu můžeme přidat vytvoření efektů \ref{code:loader-data:edited}. Tímto máme napojené efekty na akce.

Nejdříve tedy přidáme 1:N vztah do Akce:

\begin{listing}[H]
    \begin{minted}{java}
    @ManyToOne(fetch = FetchType.EAGER, cascade = {CascadeType.MERGE, CascadeType.DETACH})
    private Effect effect;
    \end{minted}
    \caption{Přidaný vztah do Akce}
    \label{code:relation:action-effect}
\end{listing}

Anotace @ManyToOne v kódu ukazuje, že existuje vztah "mnoho k jednomu" mezi entitami, kde více instancí \texttt{Action} může být spojeno s jednou instancí třídy Effect \ref{code:relation:action-effect}. Parametr fetch = FetchType.EAGER určuje, že související Effect entita bude načtena okamžitě s načtením hlavní entity, což je opačný přístup k FetchType.LAZY, který načítá data až při jejich explicitním požadavku.

\begin{description}
    \item[CascadeType.MERGE] znamená, že když se provede operace merge na hlavní entitu, tato operace se automaticky propaguje i na entitu Effect. To je užitečné například při aktualizaci dat, kdy chcete, aby se změny v hlavní entitě automaticky projevily i na přidružených entitách.
    \item[CascadeType.DETACH] znamená, že když se hlavní entita odpojí od persistence kontextu (například při ukončení transakce), odpojí se i entita Effect. Toto může být užitečné pro správu životního cyklu entit v kontextu JPA.
\end{description}

A nyní můžeme vložit efekty do databáze a při získání akce se nám zobrazí i efekt. Níže je upravený DataLoader \ref{code:loader-data:edited}.

\begin{listing}[H]
    \begin{minted}{java}
@Bean
CommandLineRunner initDatabase(ActionRepository repository, EffectRepository effectRepository) {

    Effect effect = effectRepository.save(Effect.builder().name("Damage").strength(10).build());
    log.info("Preloading {}", effect);

    return args -> {
        log.info("Preloading {}", repository.save(Action.builder().name("Punch").description("Punch the enemy").effect(effect).build()));
        log.info("Preloading {}", repository.save(Action.builder().name("Kick").description("Kick the enemy").effect(effect).build()));
    };
}
    \end{minted}
    \caption{Upravený loader dat}
    \label{code:loader-data:edited}
\end{listing}



\section{Evoluce a udržitelnost API}

Evoluce REST API přesahuje jen přidání hypermediálních prvků. REST není fixní technologie ani standard, ale sada architektonických pravidel, která zvyšují odolnost aplikací. Klíčem je umožnit aktualizace služeb bez výpadků pro klienty.

V minulosti byly aktualizace známé tím, že často "rozbíjely" klienty, protože aktualizace serveru vyžadovala změny na straně klienta. V dnešní době, kdy i minuty výpadku mohou znamenat milionové ztráty, jsou staré strategie aktualizace neudržitelné.

Představte si, že z nějakého důvodu potřebujete rozdělit jméno Akce na dvě části, například první slovo a zbytek.

Před úpravou třídy Action z jednoho pole name na firstWord a restWords, je důležité zvážit, zda tyto změny nepřeruší funkčnost pro stávající klienty. Proto tedy můžeme upravit třídu Action takto:

\begin{listing}[H]
    \begin{minted}{java}
public class Action {
    @Id
    @GeneratedValue(strategy = GenerationType.AUTO)
    private Integer id;

    private String description;

    @NonNull
    @Column(nullable = false)
    private String firstWord;
    private String restWords;

    public String getName() {
        return firstWord + " " + (restWords == null ? "" : restWords);
    }
    @ManyToOne(fetch = FetchType.EAGER, cascade = { CascadeType.MERGE, CascadeType.DETACH,})
    private Effect effect;

}
    \end{minted}
    \caption{caption}
    \label{code:label}
\end{listing}

V tomto případě jsme přidali dvě nové pole firstWord a restWords a metodu getName, která vrací původní jméno akce. Tímto jsme zachovali zpětnou kompatibilitu s klienty, kteří očekávají jméno akce v jednom poli. Samozřejmě můžeme větší změny provést také tak, že vydáme novou verzi API.

\section*{Závěr}

V tomto handbooku bylo pojednáváno o technikách pro vývoj REST API pomocí Spring. REST API nezahrnuje jen správu URL a výměnu dat ve formátu JSON, ale také zahrnuje strategie, které zabezpečují, že služby jsou kompatibilní s existujícími klienty:

\begin{itemize}
    \item Zachování starých polí - Neodstraňujte existující data, která mohou klienti stále používat.
    \item Relační odkazy - Pomocí hypermediálních odkazů umožňují klientům flexibilně přistupovat k zdrojům bez pevně zakódovaných URL.
    \item Dlouhodobé odkazy - I při změnách udržujte staré odkazy aktivní, aby byl zajištěn přechod na nové funkce.
    \item Řízení stavu pomocí odkazů - Instruujte klienty o dostupných operacích pomocí odkazů namísto dat.
\end{itemize}

V případě potřeby celý demonstrační projekt je k dispozici na \url{https://gitlab.com/rcMarty/api-example}
\chapter{Závěr}
V této práci bylo vytvořeno API pro výpravnou hru. Bylo vytvořeno RESTful API, pro získávání vkláadání a upravování hráčských a administrátorských dat. API bylo vytvořeno v jazyce Java s využitím frameworku Spring Boot. Výsledkem práce je API které slouží jako hlavní komunikační nástroj mezi samotnou hrou tak i administrátorským systémem.
\endinput

% Seznam literatury
\printbibliography[title={Literatura}, heading=bibintoc]

% Seznam zkratek
\printnoidxglossary[type=shortcuts]%, style=shortcuts]
\printnoidxglossary[type=term]%, style=shortcuts]

% Přílohy
\appendix
\input{chapters/appendix/implProblems}
\begin{figure}[h]
    \centering
    \includegraphics[width=0.9\textwidth]{../../shared/diagrams/dbScheme.pdf}
    \caption{ER diagram databáze}
    \label{fig:dix:database_schema}
\end{figure}

\begin{figure}[h] %TODO graf endpointu
    \centering
    \includegraphics[width=\textwidth]{figures/impl/endpointSwagger.png}
    \caption{Swagger dokumentace koncového bodu akce}
    \label{fig:action:endpoint}
\end{figure}

\end{document}

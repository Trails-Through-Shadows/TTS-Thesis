\chapter{Závěr}
\label{ch:conclusion}

V této bakalářské práci jsem se zaměřili na návrh a implementaci administrativního rozhraní pro kompletní správu obsahu výpravně evoluční hry, která kombinuje prvky deskových a digitálních her. Cílem práce bylo vytvořit efektivní nástroj umožňující administrátorům správu herních prvků, jako jsou postavy, předměty, lokace, nepřátele a další. V průběhu práce byla provedena analýza existujících administrativních rozhraní zaměřená na jejich funkčnost a uživatelskou přívětivost, na základě které byla výsledná práce směřována.

Implementace administrativního rozhraní zahrnovala vytvoření funkčního prototypu, který umožňuje provádět operace nad objekty, jako je vytváření, editace, mazání nebo zobrazení detailů. Prototyp byl vyvíjen s obezřetnosti na možnost budoucího rozšíření a modulárnosti. Díky kombinaci moderních technologií bylo možné vytvořit uživatelsky přívětivé rozhraní s intuitivní navigací a bezpečnostními opatřeními.

Za součást této práce se dá považovat i spolupráce s ostatními členy týmu na vývoji herního designu, implementaci API nebo tvoření celkového obsahu hry, který lze vidět v zdrojových kódech přílohy \textit{TTS-Database}.

Kromě technické implementace byla také vytvořena uživatelská příručka, které je samostatně integrovaná do administrativního rozhraní. Uživatelskou příručku lze naleznout v kterékoliv sekci webu pod tlačítkem \textit{Help}. Tento manuál poskytuje dostatečné informace, které uživatelům umožňují efektivně využívat všechny funkce rozhraní.

Celkově lze konstatovat, že výsledný prototyp splňuje veškeré stanovené cíle dle zadání a představuje tak schopný nástroj pro správu obsahu hry. Díky možnostem rozšíření není v budoucnu problém rozhraní rozšířit o další aktualizace. Budoucí práce by mohla zahrnovat rozšíření o další nástroje jako je například statistika, záznam provedených akcí nebo celkově rozšířit možnost efektivněji spravovat obsah.

Ukázka výsledného rozhraní je možné vidět v příloze~\ref{ch:appendix-interface-screenshots}

\endinput
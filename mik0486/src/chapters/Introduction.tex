\chapter{Úvod}
\label{ch:introduction}

Herní průmysl v dnešní době nabízí širokou škálu žánrů, které se od sebe liší především herním stylem. Jedním z těchto žánrů je i výpravně evoluční hra, která je především kombinací deskových a digitálních her. V těchto herních žánrech, kde právě přebývá digitální prvek je důležité mít kvalitní a sofistikovaný nástroj pro správu obsahu hry.

V úvodní kapitole se tato bakalářská práce věnuje návrhu a implementaci administrativního rozhraní pro kompletní správu obsahu výpravné evoluční hry. Cílem této práce je vytvořit robustní, uživatelsky přívětivé rozhraní, které umožní efektivní správu herních prvků, jako jsou rasy, povolání, předměty, akce, lokace a další. Administrativní rozhraní bude sloužit jako základ pro správu obsahu hry a bude umožňovat provádět operace nad objekty, jako je vytváření, editace, mazání nebo zobrazení detailů.

V průběhu práce bude provedena analýza stávajících dostupných hybridních her s důrazem na jejich funkčnost a uživatelskou přívětivost. Na základně této analýzy následně implementujeme administrativní rozhraní, které odpovídá vytyčeným požadavkům. Důraz bude kladen na intuitivní navigaci, bezpečnost, integraci s databázi a schopnost kompletní správy dynamického obsahu hry.

Výsledek práce bude funkční prototyp administrativního rozhraní, který bude sloužit pro naplnění obsahu výpravně evoluční hry. Jako vedlejší produkt práce bude uživatelská příručka zobrazena v administrativním rozhraní, která uživateli dodá dostatečné informace pro správnou a efektivní správu dat.

\endinput
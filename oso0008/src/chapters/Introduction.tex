\chapter{Úvod}
Hry a~zábava jsou součástí lidské civilizace již od pradávných dob. Výjimkou nejsou ani stolní hry. Ta nejstarší, známá jako \textit{Královská hra z~Uru} \cite{royal_game_of_ur}, je datována k~roku 2400 před naším letopočtem do staré Mezopotámie. Od té doby se samozřejmě stolní hry velmi vyvinuly. V~dnešní době se dělí do mnoha kategorií, od jednoduchých karetních, přes složité deskové, až po hry s~miniaturami a~herními plány, které zabírají celé stoly. V~posledních letech se také začaly objevovat hry hybridní, které mimo fyzické prvky využívají také digitální technologie či jiná elektronická zařízení.

Cílem této práce je vytvořit uživatelské rozhraní pro hybridní stolní hru, která kombinuje fyzické a~digitální prvky. Projekt je vyvíjen v~týmu čtyř lidí, přičemž každý z~členů má na starost jinou část. Výsledkem je příběhově založená RPG hra, která poskytuje hráčům jedinečný zážitek. Hra obsahuje prvky stolních her, jako jsou karty, kostky a~herní plán, ale také digitální prvky, které zajišťují například vyprávění příběhu, správu nepřátel a~dalších herních komponent.

Práce je rozdělena do několika částí. V~první části je popsána historie hybridních stolních her a~jejich vývoje od prvních pokusů o~propojení s~elektronickými zařízeními až po moderní hry, které využívají vlastní aplikace či specializované elektronické zařízení. Dále je pojednáno o~problematice vývoje grafických rozhraní a~o~aplikacích vybraných titulů stolních her, které využívají digitální technologie k~zlepšení herního zážitku. V~poslední části práce je vysvětlen návrh fyzických a~digitálních prvků a~je popsána implementace prototypu uživatelského prostředí.
\chapter{Úvod}
Hry a~zábava jsou součástí lidské civilizace již od pradávných dob. Výjimkou nejsou ani stolní hry. Ta nejstarší, známá jako \textit{Královská hra z~Uru} \cite{royal_game_of_ur}, je datována k~roku 2400 před naším letopočtem do staré Mezopotámie. Od té doby se samozřejmě stolní hry velmi vyvinuly. V~dnešní době se dělí do mnoha kategorií, od jednoduchých karetních her, přes složité deskové hry, až po hry s~miniaturami a~herními plány, které zabírají celé stoly. V~posledních letech se také začaly objevovat hybridní hry, které mimo fyzické prvky využívají také digitální technologie či jiná elektronická zařízení.

Cílem práce je vytvořit uživatelské rozhraní pro hybridní stolní hru, která kombinuje fyzické a~digitální prvky. Výsledkem bude příběhově založená RPG hra, která poskytne hráčům jedinečný zážitek. Hra bude obsahovat prvky stolních her, jako jsou karty, kostky a~herní plán, ale také digitální prvky, které budou zajišťovat například vyprávění příběhu, správu nepřátel a~dalších herních komponent.

Práce bude rozdělena do několika částí. V~první části bude popsána historie hybridních stolních her a~jejich vývoje od prvních pokusů o~propojení stolních her s~elektronickými zařízeními až po moderní hry, které využívají vlastní aplikace či celé elektronické zařízení. Dále bude pojednáno o~problematice vývoje grafických rozhraní a~o~aplikacích vybraných titulů stolních her, které využívají digitální technologie k~zlepšení herního zážitku. V~poslední části práce bude vysvětlen návrh fyzických a~digitálních prvků hry a~bude popsána implementace prototypu uživatelského prostředí.
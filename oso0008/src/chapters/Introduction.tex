\chapter{Úvod}
Hry a zábava jsou součástí lidské civilizace již od pradávných dob. Výjimkou nejsou ani stolní hry. Ta nejstarší, známá jako Královská hra z Uru\cite{royal_game_of_ur}, je datována k roku 2400 před naším letopočtem do staré Mezopotámie. Od té doby se samozřejmě stolní hry velmi vyvinuly. V dnešní době se dělí do mnoha kategorií, od jednoduchých karetních her, přes složité deskové hry, až po hry s miniaturami a herními plány, které zabírají celé stoly. V posledních letech se také začaly objevovat hybridní hry, které mimo fyzické prvky využívají také digitální technologie, či jiná elektronická zařízení.

Tato práce má za úkol vytvořit podpůrnou aplikaci pro hru s fyzickým herním plánem. Kombinací mého projektu s pracemi několika dalších studentů vznikne funkční prototyp hry, kterou jsme se rozhodli nazvat Trails Through Shadows.

\section{Účel a motivace práce}
Cílem práce je vytvořit uživatelské rozhraní pro hybridní stolní hru, která kombinuje fyzické a digitální prvky. Výsledkem bude příběhově založená RPG hra, která poskytne hráčům jedinečný zážitek. Hra bude obsahovat prvky stolních her, jako jsou karty, kostky a herní plán, ale také digitální prvky, které budou zajišťovat například vyprávění příběhu, správu nepřátel a dalších herních komponent.
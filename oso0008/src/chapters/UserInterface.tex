\chapter{Uživatelské rozhraní}

\section{Stručná historie GUI}
Počátek grafických rozhraní se datuje do osmdesátých let minulého století, kdy firma Xerox vyvinula počítač Alto. Jednalo se o první počítač, jehož rozhraní se skládalo z oken, ikon a používalo myš k ovládání. Toto grafické rozhraní pak posloužilo jako odrazový můstek a základ dalším projektům. Jeden z nich byl například Apple Macintosh, který grafické uživatelské rozhraní popularizoval. Dále přišel operační systém Windows, který GUI posunul ještě dál mezi mainstream uživatele. GUI se postupem let vyvíjelo společně s novými technologiemi a nyní je neoddělitelnou součástí téměř všech počítačových systémů.

\section{Zásady vývoje webového GUI}
Pro vývoj moderního GUI je dobré se držet několika základních premis. Jednoduchost je v případě této práce téměř nutná, protože se má jednat o doplněk ke hře samotné, který má za úkol ji ulehčit. Pokud by bylo GUI zbytečně složité, či neintuitivní, ztrácelo by smyslu. Z tohoto důvodu je klíčová i konzistence stránek, což zahrnuje jejich stylování i způsob interakce systému s uživatelem. Dále je pro moderní rozhraní také důležitý responzivní design, který zajišťuje použitelnost daného rozhraní napříč zařízeními a uživatelé tak mohou hru hrát i bez přístupu k větší obrazovce. Těmito způsoby zajistíme co největší přívětivost a intuitivnost našeho programu a uživateli tím usnadníme jeho používání.

\section{GUI ve vybraných hybridních hrách}

\subsection{Na Vlnách Neznáma}
Aplikace pro hru Na Vlnách Neznáma je primárně zamýšlená pro mobilní zařízení, čemuž odpovídá její design. Jedná se o responzivní jednostránkovou aplikaci. Samotná stránka obsahuje základní nastavení přístupnosti a jazyka, základní informace o aplikaci samotné a možnosti hry. Po spuštění zůstane na stránce pouze jednoduchý vstup pro číslo, které představuje záznam, jenž má aplikace zobrazit, dále historie předešlých záznamů, náhled mapy a časovač. Při načtení záznamu se zobrazí možnosti, které nabízí a stránka spustí naraci příběhu, který je v něm obsažen. Samotná aplikace je velmi jednoduchá a přímočará, takže představuje ideální doprovod k samotné hře. Je také velmi graficky založená, což také napomáhá imerzi.

\subsection{Gloomhaven}
Gloomhaven Secretariat je jedna z aplikací pro hru Gloomhaven, která se stará o její největší část, a to souboje. Opět se jedná o jednostránkovou webovou aplikaci. Stránka je primárně určena pro desktop, či jiná zařízení s velkou obrazovkou. Je sice použitelná i na mobilních zařízeních, ale jedná se pouze o zmenšenou verzi klasické stránky bez dalších úprav. To znamená, že některá tlačítka jsou příliš malá pro pohodlné používání. Obsah se také zdá být poměrné jednoduchý, avšak už není tak intuitivní, jako u předešlého příkladu. Po otevření stránky se zobrazí spousta informací a novému uživateli se tak může snadno stát, že se v nich ztratí. Po spuštění vás aplikace mimo jiné vyzve k výběru příběhové linie, kterou chcete začít a následně k přidání postav. Stránka pak nabízí spoustu možností, co je uživatel schopen dělat dále, ale nikde je dostatečně nevysvětlí. Během hry samotné nastává také několik podobných momentů. Stránka potřebuje neustálé vstupy, aby plnila svou funkci, ty jsou však také někdy neintuitivní a jejich zadávání zdlouhavé.

\section{Volba technologií pro vývoj GUI}
Zprvopočátku jsem vytvořil prototyp GUI v čistém HTML a CSS, avšak brzy jsem zjistil, že tento způsob vývoje je velmi zdlouhavý a neefektivní. Rozhodl jsem se tedy pro vývoj v některém z moderních frameworků. Konkrétně jsem zvažoval React, Angular a Svelte. Při výběru frameworku jsem zvážil několik faktorů, které nakonec určily, ve kterém z nich budu aplikaci vytvářet. Rozhodl jsem se pro Svelte, což je moderní framework pro tvorbu webových aplikací. Samotný kód Svelte se při buildu převede na čistý JavaScript. Výsledná aplikace je díky tomu velmi rychlá a efektivní. Svelte také nabízí velmi jednoduchý a přívětivý způsob psaní kódu, který vývoj aplikace urychlil. Dokáže také pracovat s TypeScript soubory bez nutnosti jejich předešlé kompilace. To znamená, že jsem mohl psát kód v TypeScriptu a Svelte ho při buildu převedl na JavaScript. Tato možnost mi umožnila psát kód, který je mnohem bezpečnější a přehlednější.

Dále jsem se rozhodl používat Bootstrap, což je knihovna CSS a JavaScript komponent, která ulehčuje vytvoření responzivního designu. Nabízí také mnoho předpřipravených komponent, které mi ušetřily spoustu času a práce. Díky tomu jsem se mohl soustředit na samotnou funkcionalitu aplikace.
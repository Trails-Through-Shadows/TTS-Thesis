\chapter{Uživatelské rozhraní}

\section{Stručná historie GUI}
Počátek grafických rozhraní se datuje do osmdesátých let minulého století, kdy firma Xerox vyvinula počítač Alto. Jednalo se o první počítač, jehož rozhraní se skládalo z oken, ikon a používalo myš k ovládání. Toto grafické rozhraní pak posloužilo jako odrazový můstek a základ dalším projektům. Jeden z nich byl například Apple Macintosh, který grafické uživatelské rozhraní popularizoval. Dále přišel operační systém Windows, který GUI posunul ještě dál mezi mainstream uživatele. GUI se postupem let vyvíjelo společně s novými technologiemi a nyní je neoddělitelnou součástí téměř všech počítačových systémů.

\section{Zásady vývoje webového GUI}
Pro vývoj moderního GUI je dobré se držet několika základních premis. Jednoduchost je v případě této práce téměř nutná, protože se má jednat o doplněk ke hře samotné, který má za úkol ji ulehčit. Pokud by bylo GUI zbytečně složité, či neintuitivní, ztrácelo by smyslu. Z tohoto důvodu je klíčová i konzistence stránek, což zahrnuje jejich stylování i způsob interakce systému s uživatelem. Dále je pro moderní rozhraní také důležitý responzivní design, který zajišťuje použitelnost daného rozhraní napříč zařízeními a uživatelé tak mohou hru hrát i bez přístupu k větší obrazovce. Těmito způsoby zajistíme co největší přívětivost a intuitivnost našeho programu a uživateli tím usnadníme jeho používání.

\section{GUI ve vybraných hybridních hrách}

\subsection{Na Vlnách Neznáma}
Aplikace pro hru Na Vlnách Neznáma je primárně zamýšlená pro mobilní zařízení, čemuž odpovídá její design. Jedná se o responzivní jednostránkovou aplikaci. Samotná stránka obsahuje základní nastavení přístupnosti a jazyka, základní informace o aplikaci samotné a možnosti hry. Po spuštění zůstane na stránce pouze jednoduchý vstup pro číslo, které představuje záznam, jenž má aplikace zobrazit, dále historie předešlých záznamů, náhled mapy a časovač. Při načtení záznamu se zobrazí možnosti, které nabízí a stránka spustí naraci příběhu, který je v něm obsažen. Samotná aplikace je velmi jednoduchá a přímočará, takže představuje ideální doprovod k samotné hře. Je také velmi graficky založená, což také napomáhá imerzi.

\subsection{Gloomhaven}
Gloomhaven Secretariat je jedna z aplikací pro hru Gloomhaven, která se stará o její největší část, a to souboje. Opět se jedná o jednostránkovou webovou aplikaci. Stránka je primárně určena pro desktop, či jiná zařízení s velkou obrazovkou. Je sice použitelná i na mobilních zařízeních, ale jedná se pouze o zmenšenou verzi klasické stránky bez dalších úprav. To znamená, že některá tlačítka jsou příliš malá pro pohodlné používání. Obsah se také zdá být poměrné jednoduchý, avšak už není tak intuitivní, jako u předešlého příkladu. Po otevření stránky se zobrazí spousta informací a novému uživateli se tak může snadno stát, že se v nich ztratí. Po spuštění vás aplikace mimo jiné vyzve k výběru příběhové linie, kterou chcete začít a následně k přidání postav. Stránka pak nabízí spoustu možností, co je uživatel schopen dělat dále, ale nikde je dostatečně nevysvětlí. Během hry samotné nastává také několik podobných momentů. Stránka potřebuje neustálé vstupy, aby plnila svou funkci, ty jsou však také někdy neintuitivní a jejich zadávání zdlouhavé.

\section{Volba technologií pro vývoj GUI}
Zprvopočátku jsem vytvořil prototyp GUI v čistém HTML a CSS, abych věděl, jak bude samotný produkt zhruba vypadat. Poté jsem svou pozornost obrátil na výběr technologií, které budu pro vývoj GUI používat. Hlavními kandidáty byly frameworky React, Angular a Svelte.

\subsection{Frameworky}

\subsubsection{React}
React je jednou z nejpopulárnějších moderních platforem pro tvorbu webových aplikací. Je to open-source JavaScript knihovna. Jedná se o software vyvinutý a udržovaný společností Meta (bývalý Facebook) a komunitou vývojářů. Používá deklarativní programovací paradigma, což znamená, že vývojář specifikuje jak by měl výsledek vypadat bez toho, aby musel explicitně popsat, jak daného výsledku dosáhnout. Zároveň je založen na komponentovém přístupu. Celý kód je rozdělen do menších celků zvaných komponenty. Ty jsou modulární a znovupoužitelné. Díky jeho schopnosti aktualizovat jednotlivé komponenty, se nejčastěji používá pro vývoj jednostránkových webových aplikací. React je známý svou komunitou a ekosystémem, který je kolem něj postavený. Díky tomu je možné najít spoustu předpřipravených komponent a knihoven, které urychlí vývoj aplikace. Zároveň má však velmi strmou křivku učení, což z něj dělá nepřívětivou volbu pro začátečníky.

React využívá virtuální DOM, který zajišťuje rychlejší a efektivnější vykreslování změn. Při změně v komponentě se nevykreslí celý strom, ale pouze změněná část. Tím se značně zrychlí vykreslování a zároveň se sníží nároky na výkon.

\subsubsection{Angular}
Angular je další z vysoce populárních frameworků pro vývoj UI. Opět se jedná o open-source platformu, nyní však vyvinutou a udržovanou společností Google a komunitou vývojářů. Angular je založen na jazyce TypeScript a stejně jako React využívá komponentového přístupu a deklarativního programovacího paradigmatu. Také se používá převážně pro vývoj jednostránkových aplikací.

\subsubsection{Svelte}
Svelte je moderní framework pro tvorbu webových aplikací. Jedná se o open-source software vyvinutý Richem Harrisem a komunitou vývojářů. Svelte se od ostatních frameworků liší tím, že se jeho kód při buildu převede na čistý JavaScript. Tím se výsledná aplikace značně zrychlí a zároveň se sníží nároky na výkon na straně uživatele. Svelte také nabízí velmi jednoduchý a přívětivý způsob psaní kódu, který vývoj aplikace urychlil. Dokáže také pracovat s TypeScript soubory bez nutnosti jejich předešlé kompilace, což znamená, že výsledný kód je mnohem bezpečnější a přehlednější.

\subsubsection{Srovnání}
Všechny tři frameworky jsou velmi populární a mají své výhody a nevýhody. React je známý svou komunitou a ekosystémem, který je kolem něj postavený. Angular je zase známý svou robustností a bezpečností. Svelte je pak známé svou rychlostí a jednoduchostí. Pro tuto práci jsem se nakonec rozhodl použít Svelte, z důvodu právě jeho jednoduchosti a přívětivosti k začátečníkům a zároveň, protože je ideální na vývoj menších aplikací, které jsou ovšem velmi rychlé a optimalizované.

\subsection{Další technologie}
Dále jsem se rozhodl používat css knihovny, které dokáží ušetřit práci s designem a responzivitou a zároveň zrychlí vývoj aplikace. Jako hlavní kandidáti se ukázaly Bootstrap a Tailwind CSS. Bootstrap je velmi populární knihovna, která nabízí spoustu předpřipravených komponent a stylů. Tailwind CSS je naopak známý svou flexibilitou a možnostmi přizpůsobení.
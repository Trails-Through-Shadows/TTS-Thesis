\chapter{Závěr}
Tato bakalářská práce se zaměřila na tvorbu uživatelského prostředí pro výpravnou evoluční hru, která kombinuje prvky stolní a~počítačové hry. Cílem práce bylo vytvořit funkční prototyp uživatelského prostředí, který bude schopen zobrazit herní svět a~umožní hráčům provádět herní akce. Práce byla rozdělena do několika částí, které se zabývaly historií hybridních stolních her, problematikou vývoje grafických rozhraní a~návrhem a~implementací grafického rozhraní pro výpravnou evoluční hru.

V~úvodu práce byla popsána historie hybridních stolních her a~jejich vývoj od propojování jednoduchých elektrických obvodů až po moderní hry, které mívají vlastní aplikace či elektronická zařízení. Následné kapitoly se zabývaly problematikou a~zásadami vývoje grafických rozhraní. Tyto zkušenosti byly předvedeny na pomocných aplikacích pro stolní hry \textit{Na vlnách neznáma} a~\textit{Gloomhaven}.

Práce se dále zabývala návrhem a~implementací grafického rozhraní pro výpravnou evoluční hru. V~rámci tohoto návrhu byly srovnány použitelné technologie pro tvorbu grafických rozhraní a~byly vybrány ty, které byly považovány za nejvhodnější. Dále byly navrženy fyzické herní komponenty a~herní mechaniky. Na závěr byla provedena implementace samotné aplikace. Výsledkem práce je funkční prototyp uživatelského prostředím, který byl následně otestován několika jednotlivci, jenž poskytli cennou zpětnou vazbu. Ta byla použita k~vylepšení prototypu uživatelského prostředí.

Výsledek práce byl propojen s~pracemi ostatních spoluřešitelů, což vedlo k~vytvoření funkčního prototypu celé stolní hry. Tento byl následně otestován odehráním několika herních situací, z~čehož vyplynulo, že prototyp je správně schopen plnit funkce, k~nímž byl navržen.

V~budoucnu by bylo možné prototyp dále rozvíjet a~doplňovat o~další herní mechaniky a~komponenty, jako například předčítání textů, zvukové efekty či animace. Samotná hra by mohla být rozšířena o~další herní prvky, jako například předměty, vyvolávatelné poskoky a~dynamický příběh. Mimo to je do hry možné poměrně jednoduše přidat více kampaní, lokací a~postav, což by zvýšilo herní dobu a~zábavu. Tyto prvky by mohly být implementovány v~rámci rozšíření hry, které by bylo možné vytvořit na základě získaných zkušeností z~této práce.
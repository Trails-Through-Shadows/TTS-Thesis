\chapter{Závěr}
Tato bakalářská práce se zaměřila na tvorbu uživatelského prostředí pro výpravnou evoluční hru, která kombinuje prvky stolní a počítačové hry. Cílem práce bylo vytvořit funkční prototyp uživatelského prostředí, který bude schopen zobrazit herní svět a umožní hráčům provádět herní akce. Práce byla rozdělena do několika částí, které se postupně zabývaly historií hybridních stolních her, problematikou vývoje grafických rozhraní a návrhem a implementací grafického rozhraní pro výpravnou evoluční hru.

V úvodu práce byla popsána historie hybridních stolních her a jejich vývoj od propojování jednoduchých elektrických obvodů až po moderní hry, které mívají vlastní aplikace, či celé elektronické zařízení. Následně byla pojednána problematika a zásady vývoje grafických rozhraní a tyto zkušenosti byly předvedeny na pomocných aplikacích pro stolní hry \textit{Na vlnách neznáma} a \textit{Gloomhaven}.

Práce se dále zabývala návrhem a implementací grafického rozhraní pro výpravnou evoluční hru. V rámci tohoto návrhu byly srovnány použitelné technologie pro tvorbu grafických rozhraní a byly vybrány ty, které byly považovány za nejvhodnější. Dále byly navrženy fyzické herní komponenty a herní mechaniky, které byly následně implementovány v prototypu uživatelského prostředí. Výsledkem práce je funkční prototyp uživatelského prostředí, který je schopen zobrazit herní svět a umožňuje hráčům provádět herní akce. Tento prototyp byl následně otestován několika jednotlivci, kteří poskytli cenné zpětné vazby, které byly použity k vylepšení prototypu uživatelského prostředí.

Výsledek práce byl propojen s pracemi ostatních spoluřešitelů, což vedlo k vytvoření funkčního prototypu celé stolní hry. Tento prototyp byl následně otestován odehráním několika herních situací. Z těchto testů vyplynulo, že prototyp je správně schopen plnit funkce, k nímž byl navržen.
\chapter{Návrh designu}
Tato kapitola se zabývá návrhem designu stolní hry a aplikace, která bude sloužit jako její součást. Jsou zde popsány požadavky na aplikaci, návrh grafického rozhraní a návrh herních prvků.

\section{Požadavky na aplikaci}
Před samotným návrhem, bylo nutné stanovit požadavky, které by měla aplikace splňovat. Tyto požadavky byly stanoveny na základě analýzy stolních her Na vlnách neznáma, Gloomhaven a Dungeons \& Dragons a na základě požadavků, které vyplynuly ze samotného návrhu hry.

\subsection{Grafické rozhraní}
Vzhled aplikace by měl odrážet tematiku hry. Ta je situovaná do temného fantaskního světa s RPG prvky. Samotná aplikace by tato témata měla reflektovat. Aby si hra uchovala tento temný nádech, tak bylo rohodnuto, že základem designu bude černá barva, která bude doplněna o zelené a šedé prvky.

Zážitek ze hry bude podporován tematickými obrázky, které hráčům pomohou lépe se do hry vcítit. Obrázky budou použity na pozadí herních stránek, ikonách postav a nepřátel a na kartách, které budou hráči používat. Tyto obrázky budou vygenerovány pomocí umělé inteligence, o které se bude hovořit v kapitole %\ref{sec:AI}.

Vzhled aplikace by měl být jednoduchý a přehledný. Hráči by měli mít možnost snadno se na stránce orientovat a mít přehled o všech herních prvcích.

\subsection{Funkcionality}
Aplikace bude sloužit jako nástroj pro ukládání hráčských dat a pro správu herního světa. Hráči budou mít možnost vytvářet postavy, které budou moci upravovat a vylepšovat. Postavy budou mít možnost se účastnit dobrodružných bojových střetnutí, které budou záviset na hráči určených kampaních. Jednotlivá střetnutí budou mít vlastní mapu předtsavující hrací plochu a budou obsahovat nepřátele, kteří budou mít své vlastní schopnosti a statistiky. Všechny tyto informace budou uloženy v databázi a hráčům přístupny právě přes webové rozhraní.

Následující funkcionality vyplývají z vlastního návrhu, který vzešel z analýzy některých titulů a zkušeností nabytých hraním stolních her.


\chapter{Hybridní hry}
Hybridní hry kombinují jak prvky fyzické, tak digitální. Jedná se o hry, které mají jakékoliv napojení na technologii, ať už je to elektronické bankovnictví ve známé hře na odkupování pozemků, či čtení příběhu a směrování dalšího postupu hráčů pomocí internetových stránek. Podobná spojení vyústí ve zcela nové herní zážitky, které si hráči můžou vyzkoušet.

\section{Typy hybridních her}
Hry, jež používají elektronická zařízení zamýšlená speciálně pro danou hru, byly prvním zástupcem hybridních her.

Tato zařízení jsou například výše zmíněný bankomat, který sám počítal herní měnu a převáděl ji mezi hráči.

Za další z druhů hybridních her lze považovat i hry s rozšířenou realitou. Tyto hry zažily rozmach v posledním desetiletí, a to hlavně v podobě mobilních her.

Dále existují stolní hry, které fungují za pomoci aplikací (ať už webových, či jiných), které si uživatel spustí na svém zařízení. Této kategorie se týká i tato práce.

\section{Vybrané hybridní hry}
Následující hry jsem vybral jako příklady a inspiraci pro svou práci. Jedná se o stolní hry, které nějakým způsobem využívají právě internetových aplikací pro umocnění herního zážitku.

První z těchto her je Forgotten Waters (českým názvem Na vlnách neznáma). Jedná se o výpravnou RPG hru, která používá aplikaci jako nástroj pro vyprávění herního děje a k zaznamenávání hráčských rozhodnutí, díky čemuž hra dokáže dynamicky reagovat. Aplikace dále udává životy a statistiky nepřátel, slouží k výběru dějové linky a v neposlední řadě přispívá k zážitku hráčů pomocí namluvených scén. Tato aplikace je oficiální součástí dané hry a nelze ji bez ní hrát.

Dále bych chtěl uvést hru s názvem Gloomhaven, pro kterou, na rozdíl od hry předešlé, není aplikace potřebná, a dokonce momentálně neexistuje ani žádná oficiální. Hra samotná obsahuje velké množství různých karet, tokenů a dalších věcí, které, seč jsou pro hru samotnou podstatné, ji zbytečně protahují a komplikují. Z tohoto důvodu vzniklo pro tento titul hned několik pomocných aplikací, které se tyto problémy snaží řešit. Většina z nich si je velice podobná jak funkčností, tak vzhledem, jelikož vycházejí ze stylu samotné stolní hry.
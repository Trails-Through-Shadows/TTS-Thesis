\documentclass[czech,bachelor]{../../shared/diploma}

\usepackage[autostyle=true,czech=quotes]{csquotes} % korektni sazba uvozovek, podpora pro balik biblatex
\usepackage[backend=biber, style=iso-numeric, alldates=iso]{biblatex} % bibliografie
\usepackage{dcolumn} % sloupce tabulky s ciselnymi hodnotami
\usepackage{subfig} % makra pro "podobrazky" a "podtabulky"
\usepackage{float} % lepsi umistovani obrazku (H)
\usepackage[outputdir=./build]{minted} % vkladani zdrojovych kodu

\setminted{fontsize=\small, baselinestretch=1, frame=lines, framesep=8pt, linenos}
\renewcommand\listingscaption{Výpis}
\renewcommand\listoflistingscaption{Seznam výpisů zdrojového kódu}

% Pozadovane vstupy pro generovani titulnich stran.
\ThesisAuthor{Miroslav Osoba}
\ThesisSupervisor{Ing. Radoslav Fasuga, Ph.D.}

\CzechThesisTitle{Tvorba uživatelského prostředí výpravné evoluční hry}
\EnglishThesisTitle{Creation of the User Environment for the Narrative Evolution Game}

\SubmissionYear{2024}

\ThesisAssignmentFileName{../specification.pdf}

\CzechAbstract{Údělem této bakalářské práce je vytvoření uživatelského prostředí pro výpravnou evoluční hru, která kombinuje prvky stolní a počítačové hry. V úvodu práce je popsáne historie hybridních stolních her a jejich vývoj. Následně je pojednáno o problematice vývoje grafických rozhraní a tyto zkušenosti předvedeny na vybraných titulech stolních her. Práce se dále zabývá návrhem a implementací graficého rozhraní, které umožní hráčům interakci s herním světem a jeho pravidly. Výsledkem práce je funkční prototyp uživatelského prostředí, který je schopen zobrazit herní svět a umožňuje hráčům provádět herní akce a společné s pracemi několika kolegů dá za vznik funkčího prototypu celé stolní hry.}
\CzechKeywords{hybridní desková hra; uživatelské prostředí; grafické rozhraní}

\EnglishAbstract{The purpose of this bachelor thesis is to create a user interface for a narrative evolution game that combines elements of board and computer games. The introduction describes the history of hybrid board games and their development. Subsequently, the issue of developing graphical interfaces is discussed and these experiences are demonstrated on selected board game titles. The thesis further deals with the design and implementation of a graphical interface that allows players to interact with the game world and its rules. The result of the work is a functional prototype of the user environment, which is able to display the game world and allows players to perform game actions and, together with the work of several colleagues, creates a functional prototype of the entire board game.}
\EnglishKeywords{hybrid board game; user environment, graphical interface}

\Acknowledgement{Tímto bych rád poděkoval vedoucímu práce Ing. Radoslavu Fasugovi, Ph.D. za jeho rady a cenné připomínky, které mi pomohly při tvorbě této práce. Dále bych chtěl poděkovat svým přátelům a kolegům Barboře Kovalské, Martinu Korotwitschkovi a Pavlu Mikulovi za jejich ochotu a trpělivost při tvorbě tohoto projektu. Velký dík také patří mé rodině a blízkým přátelům, bez jejichž podpory by bylo studium na vysoké škole mnohem obtížnější. V poslední řadě bych chtěl poděkovat těm několika jednotlivcům, kteří se podíleli na testování prototypu uživatelského prostředí a poskytli mi cenné zpětné vazby.}

\addbibresource{resources/sauce.bib}

% Novy druh tabulkoveho sloupce, ve kterem jsou cisla zarovnana podle desetinne carky
\newcolumntype{d}[1]{D{,}{,}{#1}}

% Zacatek dokumentu
\begin{document}

% Titulni strany
\MakeTitlePages

% Seznam obrazku
\listoffigures
\clearpage

% Seznam tabulek
% \listoftables
% \clearpage

% Text zaverecne prace.
\chapter{Úvod}
\label{ch:introduction}


\endinput

\chapter{Hybridní hry}
Hybridní hry kombinují jak prvky fyzické, tak digitální. Jedná se o hry, které mají jakékoliv napojení na technologii, ať už je to elektronické bankovnictví ve známé hře na odkupování pozemků, či čtení příběhu a směrování dalšího postupu hráčů pomocí internetových stránek. Podobná spojení vyústí ve zcela nové herní zážitky, které si hráči můžou vyzkoušet.

\section{Typy hybridních her}
Hry, jež používají elektronická zařízení zamýšlená speciálně pro danou hru, byly prvním zástupcem hybridních her.

Tato zařízení jsou například výše zmíněný bankomat, který sám počítal herní měnu a převáděl ji mezi hráči.

Za další z druhů hybridních her lze považovat i hry s rozšířenou realitou. Tyto hry zažily rozmach v posledním desetiletí, a to hlavně v podobě mobilních her.

Dále existují stolní hry, které fungují za pomoci aplikací (ať už webových, či jiných), které si uživatel spustí na svém zařízení. Této kategorie se týká i tato práce.

\section{Vybrané hybridní hry}
Následující hry jsem vybral jako příklady a inspiraci pro svou práci. Jedná se o stolní hry, které nějakým způsobem využívají právě internetových aplikací pro umocnění herního zážitku.

První z těchto her je Forgotten Waters (českým názvem Na vlnách neznáma). Jedná se o výpravnou RPG hru, která používá aplikaci jako nástroj pro vyprávění herního děje a k zaznamenávání hráčských rozhodnutí, díky čemuž hra dokáže dynamicky reagovat. Aplikace dále udává životy a statistiky nepřátel, slouží k výběru dějové linky a v neposlední řadě přispívá k zážitku hráčů pomocí namluvených scén. Tato aplikace je oficiální součástí dané hry a nelze ji bez ní hrát.

Dále bych chtěl uvést hru s názvem Gloomhaven, pro kterou, na rozdíl od hry předešlé, není aplikace potřebná, a dokonce momentálně neexistuje ani žádná oficiální. Hra samotná obsahuje velké množství různých karet, tokenů a dalších věcí, které, seč jsou pro hru samotnou podstatné, ji zbytečně protahují a komplikují. Z tohoto důvodu vzniklo pro tento titul hned několik pomocných aplikací, které se tyto problémy snaží řešit. Většina z nich si je velice podobná jak funkčností, tak vzhledem, jelikož vycházejí ze stylu samotné stolní hry.
\chapter{Uživatelská rozhraní}
Grafická uživatelská rozhraní (GUI) jsou klíčovým prvkem většiny aplikací. Jsou to první, co uživatel uvidí a~co mu umožní s~aplikací interagovat. Dobře navržené GUI může znamenat rozdíl mezi úspěchem a~neúspěchem aplikace. Mělo by být intuitivní, snadno ovladatelné a~příjemné na pohled. Následující kapitola bude věnována základním principům designu GUI.

\section{Stručná historie GUI}
Počátek grafických rozhraní se datuje do osmdesátých let minulého století, kdy firma \textit{Xerox} vyvinula počítač \textit{Alto}. Jednalo se o~první počítač, jehož rozhraní se skládalo z~oken, ikon a~používalo myš k~ovládání. Toto grafické rozhraní pak posloužilo jako odrazový můstek a~základ dalším projektům. Jeden z~nich byl například \textit{Apple Macintosh}, který grafické uživatelské rozhraní popularizoval. Dále přišel operační systém \textit{Windows}, který GUI posunul ještě dál mezi mainstreamové uživatele. GUI se postupem let vyvíjelo společně s~novými technologiemi a~nyní je neoddělitelnou součástí téměř všech počítačových systémů.

\section{Zásady vývoje webového GUI}
Zásadní pro uživatelsky přívětivé UI jsou vzhled stránky a~použitelnost. Základní principy vychází z~praktických zkušeností grafiků i~psychologických studií.

\subsection{Vzhled stránky}
Tato sekce se bude věnovat základním principům designu, které by mělo moderní uživatelské rozhraní splňovat. Následuje několik z~nich, které jsou důležité pro vytvoření uživatelsky přívětivého GUI. \cite{principles_of_design}

\subsubsection*{Kontrast}
Kontrast, jakožto jedna z klíčových složek designu, zajišťuje, že text a~ostatní prvky jsou dobře čitelné a~viditelné. Jeho úlohou je posílit dojem a~upoutat pozornost na důležité prvky.

\subsubsection*{Vyváženost}
Každý prvek na stránce má určitou váhu, která je dána jeho velikostí, barvou, kontrastem a~dalšími faktory. Vyváženost zajistí, že tyto prvky jsou rovnoměrně rozloženy, což přispívá k~celkovému dojmu vyváženosti a~přehlednosti stránky.

\subsubsection*{Důraz}
Slouží ke zvýraznění důležitých prvků a~k~pomyslnému ukrytí těch méně podstatných. Můžeme tak kontolovat výraznost určitých informací a~zároveň usměrňovat pozornost uživatele.

\subsubsection*{Proporce}
Správně určené proporce podporují vyváženost a~harmonii na stránce. Uživatelům pomáhají orientovat se a~zajišťují, že stránka působí přehledně a~esteticky.

\subsubsection*{Hierarchie}
Hierarchie je klíčová, zejména pokud jde o~zdůraznění důležitých prvků. Tento princip je často demonstrován prostřednictvím titulů a~nadpisů. Titul stránky by měl okamžitě vyniknout jako nejdůležitější prvek, zatímco nadpisy by měly být formátovány tak, aby naznačovaly svůj význam ve vztahu k~sobě navzájem a~k~obsahu, který uvádějí.

\subsubsection*{Opakování}
Opakování je účinným nástrojem pro sjednocení myšlenek v~rámci designu. Lze ho dosáhnout konzistentním použitím barev, písma, tvarů nebo jiných prvků designu. Konzistentní formátování pomáhá sjednotit vzhled celé stránky.

\subsubsection*{Rytmus}
Slovem rytmus je myšlen styl, jakým jsou prvky (nebo jejich barvy, velikosti či dokonce mezery mezi nimi) na stránce uspořádány a~v~jakém pořadí použity. Některé rytmické vzory mohou vzbuzovat pocit uspořádanosti a~přehlednosti, zatímco jiné mohou působit chaoticky.

\subsubsection*{Vzory}
Vyvolávají v~uživateli pocit předvídatelnosti a~pohodlí. Může se jednat například o~rozložení, které se běžně používá na jiných stránkách nebo o~způsob zadávání dat, který je uživatelům známý.

\subsubsection*{Volný prostor}
Volný prostor (white space) je prázdný prostor mezi prvky na stránce, který pomáhá zvýraznit důležité prvky a~zároveň zajišťuje, že stránka nevypadá přeplněná.

\subsubsection*{Navigace pohledem}
Jde o~kombinaci všech předešlých principů a~jejich aplikaci tak, aby uživatel mohl snadno a~pohodlně stránku použít.

\subsubsection*{Rozmanitost}
Může se jednat o~různé barvy, tvary či velikosti prvků, které zajišťují, že stránka nebude monotónní a~bude působit zajímavě.

\subsubsection*{Spojitost}
Zajišťuje, že všechny prvky na stránce působí jako celek. Většinou se jedná o~opakování barev či používání minimálního množství fontů.

\subsubsection{Teorie barev}
Výběr barev je nedílnou součástí vývoje každého GUI. Teorie barev se zabývá vztahy mezi barvami a~jejich významem. Poznáním těchto nuancí může vývojář využít vzhled stránky k~ovlivnění uživatelova vnímání své aplikace. Různé barvy mohou budit různý psychologický a~emocionální význam. Teorie barev poskytuje základní pravidla a~směrnice pro efektivní použití barev v~designu, aby se dosáhlo esteticky příjemného výsledku a~vyvolalo se požadované emoční nebo vizuální působení. Teorie barev také udává, že existuje několik kategorií základních barev:
\begin{itemize}
    \item \textbf{Primární barvy}: Červená, modrá a~žlutá. Tyto barvy nelze vytvořit kombinací jiných barev.
    \item \textbf{Sekundární barvy}: Zelená, fialová a~oranžová. Tyto barvy vzniknou smícháním dvou primárních barev.
    \item \textbf{Terciární barvy}: Těchto šest barev vznikne smícháním primárních a~sekundárních barvev. Patří sem například růžová, tyrkysová či žlutozelená.
\end{itemize}

Těchto dvanáct samozřejmě není jedinými barvami, které lze zejména u~moderních počítačů použít. Díky tomu se začalo v~grafice používat takzvaný barevný kruh (color wheel). \cite{color_theory_design}

\subsubsection*{Barevný kruh}
Je často využíván pro své intuitivní uspořádání barev. Obsahuje kompletní paletu barev, které lze vytvořit smícháním primárních, a~umožňuje i~upravovat jejich odstíny přidáním černé či bílé. Existuje několik způsobů, jak pomocí tohoto kruhu vybrat vhodné barevné kombinace (některé z~nich můžeme vidět na Obrázku \ref{fig:color_theory}). \cite{color_wheel,color_schemes}
\begin{itemize}
    \item \textbf{Komplementární}: Barvy, které jsou proti sobě na kruhu, čímž vytvářejí kontrastní efekt.
    \item \textbf{Monochromatické}: Různé odstíny jedné barvy, což vytváří harmonický dojem.
    \item \textbf{Analogické}: Barvy umístěné na kruhu vedle sebe. Jejich kombinace vytváří přirozený a~pohodlný dojem. Je však vhodné vybrat jednu jako hlavní a~ostatní používat jako doplňky.
    \item \textbf{Triadické}: Kombinace tří barev, které na kruhu tvoří rovnostranný trojúhelník. Podobně jako způsob komplementární, také vytváří kontrastní efekt.
    \item \textbf{Rozštěpená komplementární}: Variace komplementární kombinace, kde se namísto protější barvy, používají barvy s~ní sousedící. 
    \item \textbf{Tetradické}: Čtyři barvy, kde vždy dvě z~nich tvoří komplementární pár -- na barevném kruhu vytváří obdélník. Pokud jsou barvy správně kombinovány lze dosáhnout jak kontrastního, tak harmonického efektu.
\end{itemize}

\begin{figure}[H]
    \centering
    \includegraphics[width=0.9\textwidth]{resources/figures/color_theory.png}
    \caption{Barevné kolo a~dělení barev \cite{color_schemes}}
    \label{fig:color_theory}
\end{figure}

Dále se barvy dělí na teplé a~studené. Teplé barvy jsou červené, oranžové a~žluté části spektra a~často evokují radost a~energii. Studené barvy jsou zelené, modré a~fialové části, které často evokují klid a~harmonii. Výběr teplých či studených barev může mít vliv na celkový dojem, který uživatel z~aplikace získá. \cite{color_theory_design} 

\subsubsection{Typografie}
Typografie představuje umění používání písma a~fontů tak, aby text byl čitelný, srozumitelný a~příjemný k~čtení. Hraje klíčovou roli v~designu uživatelského rozhraní tím, že ovlivňuje rozpoznatelnost značky, uživatelovo rozhodování a~jeho pozornost. Kvalitní typografie přispívá k~efektivnímu předávání informací a~harmonické integraci s ostatními prvky rozhraní, což zlepšuje celkovou vizuální rovnováhu a~uživatelskou zkušenost. \cite{typography}

\subsection{Použitelnost}
Je úzce spjata s vzhledem stránky a~často se tato témata překrývají, ale zároveň je to samostatný princip, který je nutné uvést zvlášť. Zahrnuje všechny aspekty, které usnadňují uživateli používání stránky. Mezi takové patří: \cite{principles_of_ui_design}

\subsubsection*{Konzistence}
Pro uživatelskou přívětivost je téměř klíčová. Uživatel se rychleji naučí, jak stránka funguje, pokud je konzistentní. To znamená, že by vývojář měl dodržovat stejné rozložení a~navigaci mezi jednotlivými stránkami.

\subsubsection*{Zkratky}
Zkratky mohou urychlit a~tím zpříjemnit uživatelův pohyb po stránce. Tohoto výsledku můžeme dosáhnout například využitím odkazů v~menu či přesměrováním díky kliknutí na logo. Takovéto zkratky by měly být intuitivní a~měly by vycházet z~již známých vzorů.

\subsubsection*{Zpětná vazba}
Je pro uživatele důležitá, aby věděl, zda se akce, kterou chtěl provést, povedla či nikoliv, nebo jestli je možnost na nějaký prvek stránky kliknout. Tohoto dosáhneme pomocí animací, změny barvy či zvýraznění.

\subsubsection*{Uzavření dialogu}
Je podstatné pro uživatele, aby věděl, že jeho akce byla úspěšná a~může přejít k~dalšímu kroku. Tohoto můžeme dosáhnout například přesměrováním na jinou stránku či zobrazením dialogu, který uživateli potvrdí, že jeho akce byla úspěšná.

\subsubsection*{Prevence chyb}
Zabezpečení proti nesprávným vstupům z~uživatelovy strany je podstatné pro předejití zbytečných chyb, které by mohly vést k~frustraci. K~tomuto přispějeme například tím, že nebudeme uživateli dovolovat zadávat písmena do pole, které by mělo obsahovat pouze čísla. Pokud už chyba nenávratně nastane, je důležité uživatele informovat o~tom, co se stalo a~jak ji může napravit.

\subsubsection*{Možnost vrácení}
Pokud se uživatel rozmyslí či udělá chybu, měl by mít možnost svou akci jednoduše zrušit a~vrátit se do předešlého stavu stránky.

\subsubsection*{Locus of control}
Tento fenomén by se dal volně přeložit jako těžiště řízení. Uživatelé chtějí mít pocit, že aplikaci řídí a~že rozhraní reaguje na jejich akce. Takového pocitu můžeme mimo jiné dosáhnout tím, že se zeptáme na potvrzení nějaké akce, například odchodu ze stránky s~neuloženými daty. Tím uživateli dáme pocit větší kontroly.

\subsubsection*{Minimalizace nároků na uživatele}
Klíčová zásada pro uživatelské rozhraní je minimalizace kognitivní zátěže. Kognitivní zátěž může snížit uživatelovu schopnost vykonávat důležité úkoly, proto je důležité, aby počítače převzaly co nejvíce úkonů na sebe. Uživateli můžeme vypomoci třeba zapamatováním si jeho přihlašovacích či osobních údajů, aby je nemusel zadávat při každém přihlášení. Při návrhu bychom měli vždy dávat přednost rozpoznání před vzpomínáním, abychom uživatelům umožnili rychle a~bez problémů dokončit své úkoly.

\subsubsection*{Responzivnost}
Jedna z~nejdůležitějších vlastností moderního GUI. Responzivní design zajišťuje, že stránka bude vypadat dobře na všech zařízeních, od mobilních telefonů až po stolní počítače, a~to pomocí změny velikosti, schování či přesunutí prvků na stránce. K~tomu se využívají vlastnosti jazyka CSS, nejčastěji media queries, které umožňují nastavit různé styly pro různé velikosti obrazovek. Responzivní design je v~dnešní době kriticky potřebný, protože většina uživatelů používá k~prohlížení internetu mobilní zařízení a~je důležité, aby se jim stránka zobrazila správně a~byla snadno použitelná. \cite{responsive_design}

\section{GUI ve vybraných hybridních hrách}
Tato sekce se bude věnovat analýze uživatelského rozhraní několika vybraných hybridních her. Jedná se o~hry, které jsou v~současné době populární a~mají velkou základnu hráčů. Analýza bude zahrnovat vzhled a~funkce GUI, které hra poskytuje.

\subsection{Na vlnách neznáma}
Tato aplikace je vyvíjena a~udržována společností \textit{Plaid Hat Games}, která vydala i~samotnou hru. Jde o~oficiální rozšíření a~hlavní způsob, jak hru hrát. Společnost později vydala i~pdf, které do určité míry tuto aplikaci nahrazuje, ale hráčům poskytuje méně možností. Jedná se o~webovou aplikaci, kterou je možné stáhnout a~používat offline, avšak bez některých funkcionalit, jako jsou například rozšíření či audio nahrávky.

\subsubsection*{Uživatelské rozhraní}
Kompaktní vzhled aplikace napovídá, že byla primárně určena pro mobilní zařízení -- je tedy i~bez velkých změn velmi dobře použitelná pro počítače. Jedná se o~jednostránkovou aplikaci s~jednoduchým a~přímočarým designem. Při spuštění vás přivítá logo hry a~přívětivě barevná grafika, která odpovídá fantasknímu námořnímu stylu hry. Samotná stránka obsahuje jen několik tlačítek; začít hrát, nastavení, ke stažení, varianty hry a~o~aplikaci. Celá aplikace je přeložena do několika jazyků, a~to včetně češtiny.

\subsubsection*{Funkce}
Při spuštění hry se zobrazí výběr příběhové linky, kterou chtějí uživatelé zažít. Po vybrání se zobrazí stránka s~textovým polem, do něhož hráči mohou zadat číslo záznamu, který chtějí zobrazit. Toto číslo je jim oznámeno v~předešlém záznamu či v~knize lokací a~závisí na jejich dosavadních rozhodnutích. Dále je na stránce možnost zobrazit, jak mají hráči připravit herní plochu, statistiky jejich lodi a~karty. Na stránce je také možnost zapnutí odpočtu, pokud se hráči rozhodují moc dlouho či náhled historie zobrazených záznamů.

Aplikace slouží převážně k~zobrazování výše zmíněných záznamů a~k~podporování imerze přehráváním ambientních zvuků a~audionahrávek s~narací příběhu. Mimo jiné také napomáhá s~událostmi ve hře, jako například v~boji tím, že za pomoci uživatelových vstupů kontroluje statistiky nepřátel.

\subsection{Gloomhaven Secretariat}
Aplikace \textit{Gloomhaven Secretariat} je fanouškovskou aplikací, která vychází z~nyní již zrušené aplikace \textit{Gloomhaven Helper}. Jedná se o~jednostránkovou webovou aplikaci sestavenou v~\textit{Angularu}. Jde o~open-source software, na kterém se podílí množství fanoušků hry \textit{Gloomhaven} a~je stále ve vývoji. \cite{gloomhaven_secretariat_github}

\subsubsection*{Uživatelské rozhraní}
Aplikace je určena pro desktop či jiná zařízení s~velkou obrazovkou. Je sice použitelná i~na mobilních zařízeních, ale jedná se pouze o~zmenšenou verzi klasické stránky bez dalších úprav, což znamená, že některá tlačítka jsou příliš malá pro pohodlné používání. Obsah se také zdá být poměrné jednoduchý, avšak už není tak intuitivní, jako u~předešlého příkladu. Vzhled UI je příjemně a~čistě vypadající, a~navíc tematicky věrný samotné hře. Navigace po základní stránce je poměrně přímočará, avšak to samé už se nedá říct o~rozsáhlých menu, které se skrývají téměř pod každým tlačítkem. Ty se mohou zdát na první pohled nepřehledné a~u~velké části z~nich je nedostatek popisků či vysvětlení, jak vlastně jednotlivá tlačítka fungují. Uživatel si tak musí nejprve odzkoušet, co vše je možné, což může být zdlouhavé a~frustrující. Zároveň bych však chtěl podotknout, že aplikace má tlačítko zpět, kterým může uživatel vrátit opravdu jakoukoliv akci, takže pokud hráč udělá nějakou chybu, je snadné ji opět napravit. Obdobně je zde i~tlačítko vpřed. Stránka také potřebuje neustálé vstupy od uživatele, aby dokázala správně plnit svou funkci, ty jsou však také někdy neintuitivní a~jejich zadávání často zdlouhavé a~repetitivní.

\subsubsection*{Funkce}
Hned na úvodní stránce dostane uživatel možnost výběru příběhové linie, kterou chce začít a~následně ho stránka vyzve k~přidání postav. Dále je možnost přidat nepřátele postupně či spustit takzvaný scénář, což je v~pravidlech popsaná kombinace nepřátel, kteří se vyskytují v~určité části příběhu. Při boji aplikace zobrazuje statistiky postav a~nepřátel, jejich aktuální stavy a~efekty a~umožňuje je hráči kontrolovat a~upravovat. Stránka také sama náhodně vybírá z~možností nepřátelských akcí a~umožňuje hráčům za monstra "táhnout karty" a~tím udávat sílu a~efekt těchto akcí. Další možnosti aplikace zahrnují vedení globálních statistik skupiny, jejich úspěšnosti, odemčených lokací a~dalších věcí. Zároveň umožňuje ukládání a~stahování dat několika započatých her najednou.

Nutno říci, že aplikace je skutečně multifunkční a~nabízí hráčům velké množství automatizace a~ukládání dat. Tím velmi ulehčuje hru, zkracuje čas její přípravy a~zároveň zabraňuje chybám, které by mohly vzniknout při ručním vedení statistik.

\chapter{Návrh herního systému}
\label{chap:design}

Na základě požadavků stanovených v~předchozí kapitole \chapterref{chap:requirements} byl vytvořen návrh herního systému, který je pro přehlednost rozdělen do několika částí. První část se zabývá návrhem herních modelů, druhá se zaměřuje na herní mechaniky a~třetí popisuje návrh herních komponent.


\section{Schéma hry}
\label{sec:design_scheme}

Na návrhu databáze pracoval celý tým vyvíjející modelovou hru, jak je popsáno v~\customref{Kapitole}{subsec:database}, její rozvržení je však skvělým nástrojem pro popsání návrhu herních modelů, proto bude detailně rozebráno v~následující kapitole.

\begin{figure}[h]
    \centering
    \includegraphics[scale=0.9]{../../shared/diagrams/er_macro.pdf}
    \caption{Makro pohled na schéma databáze}
    \label{diag:er_macro}
\end{figure}

Na \customref{Obrázku}{diag:er_macro} je orientačně zobrazen makro pohled na schéma databáze, které je pro přehlednost rozděleno do několika barevně rozlišených částí, neboť celková databáze zahrnuje 48 tabulek. Každá z~těchto částí je následně podrobněji popsána v~následujících kapitolách. Celkové schéma databáze je zobrazeno v~\customref{Příloze}{chap:database_schema}.


\subsection{Akce}
\label{subsec:schema_actions}

Popis celkového rozvržení herních komponent začíná \customref{Obrázkem}{diag:er_action}, který definuje strukturu \textbf{akcí} \attr{Action}, jejichž mechanická funkcionalita je dále rozepsána v~\customref{Kapitole}{subsec:design_actions}. Každá akce má svůj název \attr{title} a~popis \attr{description}, který hráči pomůže pochopit, co daná akce přibližně znamená, bez nutnosti hlubšího studia samotné karty. Dále je v~akci uchována informace o~jejím zahození \attr{discard}, která určuje, v~jakém případě a~na jak dlouho musí hráč po odehrání danou kartu zahodit. Každá akce má také požadovanou úroveň \attr{levelReq}, která umožňuje tvorbu akcí, které hráči mohou odemykat spolu s~postupem příběhem.

\begin{figure}[h]
    \centering
    \includegraphics{../../shared/diagrams/er_action.pdf}
    \caption{Schéma akcí}
    \label{diag:er_action}
\end{figure}

V~modelové hře se akce skládají z~pěti částí, interně nazývanými \textit{features} neboli \textit{prvky akce}. Jedná se o~\textit{pohyb} \attr{Movement}, \textit{útok} \attr{Attack}, \textit{schopnost} \attr{Skill}, \textit{poskok} \attr{Summon} a~\textit{obnovení karet} \attr{RestoreCards}. Každý z~prvků je nepovinný a~může být v~akci využit maximálně jednou s~výjimkou poskoků, kterých může jedna akce vyvolat libovolný počet. Samotní poskoci poté také mají svou vlastní akci, kterou ve svém kole konají. Průběh jednotlivých prvků je popsán v~\customref{Kapitole}{subsec:design_actions}.


\subsection{Postavy}
\label{subsec:schema_character}

Na \customref{Obrázku}{diag:er_character} je zobrazeno schéma nejen postav, ale i~předmětů, neboť jsou k~sobě úzce vázány. 

\begin{figure}[h]
    \centering
    \includegraphics[width=\textwidth]{../../shared/diagrams/er_character.pdf}
    \caption{Schéma postav}
    \label{diag:er_character}
\end{figure}

\textbf{Postava} \attr{Character} má několik základních vlastností, které ji definují -- jméno postavy \attr{title} a~jméno hráče \attr{playerName}, dále je však každá postava složena z~kombinace \textbf{rasy} \attr{Race} a~\textbf{třídy} \attr{Class}. Obě tyto tabulky obsahují jméno \attr{title}, popis \attr{description} a~štítek \attr{tag}, který slouží k~identifikaci obrázku, který jim přísluší. U~obou je také zaznamenána základní iniciativa, třída však obsahuje také definici základního zdraví a~obrany postavy. Postava je vždy vázána na jedno určité dobrodružství.

Každá postava může mít také vybavení, které je k~ní navázáno přes \textbf{inventář} \attr{Inventory}, kde se zaznamenává, kolik kterého typu předmětu postava má. Samotné \textbf{předměty} \attr{Item} jsou podobně jako rasy a~třídy složeny z~názvu, popisu a~štítku, kromě toho však také obsahují informaci o~jejich typu \attr{type}, který určuje, zda se jedná o~zbraň, brnění nebo jiný typ předmětu.

Předměty, rasy a~třídy mohou být chápány jako možnosti získání určitých akcí, neboť všechny tři tabulky mají na akce vazbu. Rasa a~třída může hráči poskytovat neomezené množství akcí, předmět však má vždy jen jednu akci, kterou hráč může využít, což je limitace zavedená z~důvodu přehlednosti fyzických komponent, popsaných v~\customref{Kapitole}{subsec:design_cards}.

\subsection{Lokace}
\label{subsec:schema_location}

Schéma lokací zobrazeno na \customref{Obrázku}{diag:er_location} je díky velkému množství složených primárních klíčů složené z~mnoha vazeb, i~přes jejich četnost však není nijak zvlášť komplikované.

\begin{figure}[h]
    \centering
    \includegraphics{../../shared/diagrams/er_location.pdf}
    \caption{Schéma lokací}
    \label{diag:er_location}
\end{figure}

\textbf{Lokace} \attr{Location} je základním prvkem hry, který hráči umožňuje prozkoumávat herní svět. Ve své podstatě se jedná o~místo na herní mapě, které hráči mohou navštívit. Každá lokace má své jméno \attr{title}, popis \attr{description} a~štítek \attr{tag}, podobně jako předchozí entity. Dále si nese informaci o~typu \attr{type}, jenž určuje, zda se jedná o~lokaci se soubojem \attr{encounter} nebo o~obchod \attr{market}, nejčastěji se však používá první možnost, pro kterou se poté využívají ostatní tabulky pro zápis herní plochy.

Každá lokace se skládá z~několika \textbf{částí} mapy \attr{Part}, které mohou být na stole různě otočené, proto je jejich rotace \attr{rotation} také zaznamenaná. Části mají opět nejen jméno ale i~štítek, který má tentokrát ještě další funkcionalitu, neboť představuje identifikátor, který umožňuje hráčům snadno najít odpovídající části mapy mezi ostatními herními komponentami. Herní pole je dále rozděleno na \textbf{hexagony}, u~kterých bylo rozhodnuto o~použití tzv. kubického souřadného systému, který polím ve 2D prostoru přiřadí 3D souřadnice (\texttt{q}, \texttt{r}, \texttt{s}), což usnadňuje výpočty pro herní mechaniky, které se týkají pohybu a~vzdáleností mezi jednotlivými políčky.

Jak jde vidět z~\customref{Obrázku}{diag:er_location}, lokace jsou spolu s~ostatními tabulkami propojeny spoustou vazeb, které umožňují hlubší specifikaci propojení. Jedná se například o~reprezentaci dveří spojující části herní desky v~rámci lokace \attr{LocationDoor}, počáteční políčka lokace, kde se na začátku souboje umisťují postavy hráčů \attr{LocationStart}, nebo propojení s~kampaní, v~rámci které jsou na sebe lokace dynamicky navázány ve světové mapě \attr{LocationPath}. Speciálním případem vazební tabulky je pak obchod \attr{Market}, který pro danou lokaci určuje nabízené předměty, jejich množství a~ceny.

\subsubsection*{Nepřátelé a~překážky}
\label{subsubsec:schema_enemy_obstacle}

Lokace jsou zaplněny nepřáteli \attr{Enemy} a~překážkami \attr{Obstacle}, jak je možné vyčíst z~\customref{Obrázku}{diag:er_enemy_obstacle}. Obě dvě entity opět obsahují informace o~názvu, popisu a~štítku pro vyhledávání obrázků a~také mají vazbu na hexagon neboli políčko určité lokace, na kterém se vyskytují (\texttt{HexEnemy} a~\texttt{HexObstacle}), dále se však mírně liší.

\begin{figure}[h]
    \centering
    \includegraphics{../../shared/diagrams/er_enemy_obstacle.pdf}
    \caption{Schéma nepřátel a~překážek}
    \label{diag:er_enemy_obstacle}
\end{figure}

\textbf{Nepřátelé} v~podstatě kopírují statistiky postav, mají své vlastní zdraví, obranu a~iniciativu. Dále také mohou mít své vlastní akce, které budou v~souboji provádět \attr{EnemyAction}.

\textbf{Překážky} jsou oproti nim pasivní, proto mají zaznamenané pouze své zdraví a~míru zranění, které dostane entita, která se na ní zraní. Také má indikátor toho, zda je tato překážka průchozí \attr{crossable} nebo nikoliv.


\subsection{Kampaň}
\label{subsec:schema_campaign}

Kampaň slouží k~reprezentaci celého příběhu, kterým si hráči budou v~rámci modelové hry procházet. Její schéma je vyobrazeno na \customref{Obrázku}{diag:er_campaign}.

\begin{figure}[h]
    \centering
    \includegraphics[scale=0.8]{../../shared/diagrams/er_campaign.pdf}
    \caption{Schéma kampaně}
    \label{diag:er_campaign}
\end{figure}

Samotná \textbf{kampaň} \attr{Campaign} si moc dat nenese, obsahuje pouze název a~popis. Klíčová je však především její vazba na lokace \attr{CampaignLocation}, které jsou v~rámci kampaně propojeny do světové mapy. Tato vazební tabulka obsahuje informace o~tom, jaké lokace jsou v~rámci kampaně dostupné, zda se jedná o~počáteční \attr{start} nebo případně konečnou \attr{finish} lokaci daného příběhu a~také jaké jsou podmínky pro dokončení či selhání lokace \attr{condition}. K~této vazbě se také připojuje příběh \attr{Story}, který se může spouštět po určitých událostech odehraných v~rámci lokace.

Za zmínku také stojí herní \textbf{úspěchy} \attr{Achievement}, které jsou taktéž navázány na kampaň. Jedná se o~určité cíle, které hráči mohou splnit v~rámci hry a~za jejichž dokončení získají odměnu. Tento cíl si vždy nese název, krátký popis a~také odměnu v~podobě zkušeností.


\subsubsection*{Dobrodružství}
\label{subsubsec:schema_adventure}

Rozehranou kampaň, ve které již hráči dělají příběhový postup, reprezentuje tzv. \textbf{dobrodružství} \attr{Adventure}, které je zobrazeno na \customref{Obrázku}{diag:er_adventure}.

\begin{figure}[h]
    \centering
    \includegraphics[scale=0.8]{../../shared/diagrams/er_adventure.pdf}
    \caption{Schéma dobrodružství}
    \label{diag:er_adventure}
\end{figure}

Samotná tabulka dobrodružství je relativně objemná, obsahuje kromě klasického názvu a~popisu také statistiky rozehrané kampaně, jako je reputace \attr{reputation}, zkušenosti \attr{experience}, peníze \attr{gold} a~také úroveň družiny \attr{level}. Úspěchy určité kampaně \attr{AdventureAchievement} se zde taktéž uchovávají spolu s~číslem určující postup k~danému cíli \attr{progress}.

Vazba na lokace \attr{AdventureLocation} v~sobě drží informace o~tom, které již byly odemčeny \attr{unlocked} a~v~jakém stavu \attr{state} jsou, ať už dokončené nebo zatím neúspěšné. Pro podporu dynamičnosti herního rozvoje je zde také možnost upravovat zásoby a~ceny v~obchodech \attr{AdventureMarket}, které jsou dále vázány na samotné předměty.

Jak již bylo řečeno výše, dobrodružství se může účastnit několik postav. Kromě toho má však dobrodružství nastavenou i~\textbf{licenci} \attr{License}, se kterou ho je možné spustit \attr{AdventureLicense}. Licenční klíče jsou náhodně generovaná sekvence dvaceti znaků, které budou distribuovány spolu se hrou a~umožní hráčům přistupovat ke svému hernímu účtu a~případně si změnit heslo, pomocí kterého se budou do systému přihlašovat. Takto bude zajištěno, že hráči mají přístup pouze na své postavy a~příběhy a~zároveň se znemožní nelegální kopírování hry.


\subsection{Efekty}
\label{subsec:schema_effect}

Posledním prvkem, který je nutné zmínit, jsou \textbf{efekty} \attr{Effect}, popsány na \customref{Obrázku}{diag:er_effect}. Neboť prolínají celou strukturu hry, nebyly zapsány do jednotlivých grafů výše, ale všechny jejich vazby byly zahrnuty na tomto schématu. 

\begin{figure}[h]
    \centering
    \includegraphics[width=\textwidth]{../../shared/diagrams/er_effect.pdf}
    \caption{Schéma efektů}
    \label{diag:er_effect}
\end{figure}

Hlavní charakteristikou efektu je typ \attr{type}, jejichž mechanická implementace je popsána v~\customref{Kapitole}{subsec:design_effects}. Je zde také možnost hráčům předložit popis efektu \attr{description}, který jim pomůže pochopit, co daný efekt znamená. Dále se zaznamenává síla a~trvání efektu a~také cíl, na který efekt míří.

Efekty mohou být aplikovány na překážky, nepřátele, třídy a~rasy, předměty a~většinu prvků akce -- pohyb, útok a~schopnost. U~rasy a~třídy je navíc přidán atribut minimální úrovně \attr{levelReq}, který umožňuje postavám odemykat nové pasivní efekty s~rostoucí úrovní.


\section{Herní mechaniky}
\label{sec:design_mechanics}

Mechaniky modelové hry jsou silně inspirované především výše zmíněnou deskovou hrou \glsref{gloomhaven}, ale jsou upravené tak, aby lépe vyhovovaly stanoveným požadavkům.


\subsection{Souboj}
\label{subsec:design_encounter}

Souboj je hlavní funkcionalitou, kterou hra nabízí. Průběh souboje je rozdělen do několika fází, které jsou znázorněny na \customref{Obrázku}{diag:encounter}. Ještě před samotným hraním si systém zaznačí, že lokace už byla navštívena (\texttt{visited}), pokud již nemá nastavený jiný stav, který je důležitější (jako například \texttt{completed} nebo \texttt{failed}). Následně hráčům zobrazí příběh, který je s~touto lokací spojený.

Poté následuje fáze přípravy hrací desky. Systém zobrazí počáteční místnost lokace, všechny nepřátele, překážky a~dveře, které se v~ní vyskytují spolu s~identifikátory, které umožní hráčům tyto části jednoduše najít a~sestavit tak odpovídající konfiguraci místnosti na herním stole. Když je všechno připraveno, hráči umístí své postavy na políčka určená pro hráče a~s~tím je příprava herního pole hotová.

\newpage
\begin{figure}[H]
    \centering
    \includegraphics[height=0.98\textheight]{figures/diagrams/encounter.pdf}
    \caption{Diagram průchodu lokací}
    \label{diag:encounter}
\end{figure}
\newpage

Druhá část přípravy je věnována samotným postavám a~nepřátelům a~tvorbě iniciativního žebříčku. Každý z~hráčů si hodí svou kostkou na iniciativu a~zaznamená výsledek do systému, který tento modifikátor automaticky přičte k~základní výši iniciativy, který hráčova postava získala ze své rasy, třídy a~vybavení. Za nepřátele toto provede systém automaticky. Výsledný žebříček je seřazen sestupně a~hráči a~nepřátelé se podle něj střídají v~rámci kola. Dále se tu také provádí aplikace permanentních efektů, tedy efektů získaných ze zázemí a~předmětů u~postav nebo vrozené efekty u~nepřátel, podle \customref{Obrázku}{diag:apply_effect}. Systém pak ještě zamíchá simulovaný balíček nepřátelských karet, který bude sloužit k~určení jejich akcí.

Následuje fáze kol, která se provádí podle \customref{Obrázku}{diag:round} opakovaně ve smyčce, dokud hráči nesplní jistou předem určenou ukončovací podmínku.

\begin{figure}[h]
    \centering
    \includeplantuml[scale=0.7]{round}
    \caption{Diagram kola}
    \label{diag:round}
\end{figure}

Po dosažení této podmínky systém vyhodnotí, jak hra skončila. Pokud se jedná o~prohru, zobrazí tuto informaci hráčům a~nastaví stav lokace na \texttt{failed}. Pokud ovšem hráči lokaci zdolali úspěšně, systém provede kontrolu nových lokací a~předmětů v~obchodech, které hráči dokončením tohoto souboje odemkli nebo naopak zablokovali. Pomocí tohoto mechanismu se zajišťuje, že hra bude mít dynamický průběh a~hráči budou mít motivaci prozkoumávat nové lokace a~plnit úkoly. Po této kontrole systém hráčům zahlásí co vše jejich úspěch změnil, jaké mají nové možnosti a~kolik peněz a~zkušeností svou výhrou získali. Na závěr systém přehraje příběh konce této lokace a~navrátí hráče zpět do mapy, kde mohou pokračovat v~průzkumu světa.


\subsection{Kola}
\label{subsec:design_rounds}

V~rámci herního kola \imageref{diag:round} se nejprve provádí tahy hráčů a~nepřátel v~pořadí iniciativy podle Obrázků \ref{diag:player_turn} a~\ref{diag:enemy_turn}. Když tahy dojdou na konec iniciativy, systém zkontroluje, zda některý z~hráčů během svého kola neotevřel dveře. Pokud ano, pak je hráčům zobrazena nově odhalená místnost, jejíž desku, nepřátele, překážky a~dveře si opět podle identifikátorů rozloží na herní stůl, čímž se herní plocha rozšíří. Systém následně přidá nové nepřátele do iniciativního žebříčku a~zamíchá jejich karty. Pokud hráči dveře neotevřeli, systém přeskočí tuto fázi, po které již kolo končí.


\subsection{Tahy}
\label{subsec:design_turns}

\begin{figure}[h]
    \centering
    \includeplantuml[scale=0.7]{playerTurn}
    \caption{Diagram herního tahu hráče}
    \label{diag:player_turn}
\end{figure}

Tah hráče je vyobrazen na \customref{Obrázku}{diag:player_turn}. Během začátku a~konce se systém věnuje efektům, jejichž vyhodnocení a~uvolnění je popsáno v~\customref{Kapitole}{subsec:design_effects}, a~také poskokům, což uvolňuje čas, který by jinak musel být stráven jejich monitorováním. Samotný hráč má nejprve za úkol vybrat si dvě karty, které v~tomto kole bude chtít zahrát. Následně akce na kartách vyhodnotí podle diagramů popsaných v~\customref{Kapitole}{subsec:design_actions} a~pokud má na stole vyložené nějaké karty vyvolaných poskoků, zahraje také jejich akce. Veškeré akce může vykonat v~libovolném pořadí, což mu dává možnost plánovat své tahy tak, aby byly co nejefektivnější. Pokud během vykonávání akcí dojde k~změně stavu jakékoliv entity na herní desce, hráč tyto změny zaznamená a~pokračuje v~dalších akcích. Po vykonání všech akcí hráč zahodí vybrané karty podle jejich zahazovacího pravidla, čímž své kolo ukončí.

\begin{figure}[h]
    \centering
    \includeplantuml[scale=0.7]{enemyTurn}
    \caption{Diagram herního tahu nepřítele}
    \label{diag:enemy_turn}
\end{figure}

Tah nepřítele znázorněn na \customref{Obrázku}{diag:enemy_turn} je oproti tomu hráčskému mnohem jednodušší. Nepřátelé mají svůj vlastní balík karet, ze kterého jim systém vybere jednu, kterou v~daném kole zahrají. Jedná se o~simulaci reálného karetního balíčku, takže pokud v~něm karty dojdou, opět se všechny zamíchají a~tahají se znova. Na hráčích pak závisí, aby vyhodnotili, jakým způsobem se nepřátelé pohnou a~na koho zaútočí, ale musí se řídit instrukcemi, které od systému dostali. Jakékoli změny musí být opět propsány do systému, který se zde také stará o~vyhodnocení efektů.


\subsection{Akce}
\label{subsec:design_actions}

Jak již bylo znázorněno na \customref{Obrázku}{diag:er_action}, akce jsou strukturovány do pěti prvků, které mohou ale také nemusí být v~akci využity. Vyhodnocení akce spočívá v~provedení všech těchto prvků dle uvážení hráče, přičemž si může vybrat nejen to, jak se k~jednotlivým prvkům jeho postava zachová, ale také v~jakém pořadí je provede.

\begin{figure}[h]
    \centering
    \includeplantuml[scale=0.6]{movement}
    \caption{Diagram pohybu}
    \label{diag:movement}
\end{figure}

Na prvním \customref{Obrázku}{diag:movement} je zobrazen průběh \textbf{pohybu}. Před samotným vyhodnocením systém hráči předá informaci o~bonusové rychlosti nebo případně zpomalení jeho postavy, což je možné získat z~aktivních efektů nebo vybavení postavy. Hráč tento bonus přičte k~číslu, které má napsané na kartě pohybu a~posune se po herním poli o~odpovídající počet polí. Pohyb má tři typy -- \textit{chůze}, \textit{skok} a~\textit{teleportace}. Chůze je základním typem pohybu, skok je pohyb, který umožňuje přeskočit překážku nebo nepřítele a~teleportace je pohyb, který umožňuje postavě přesunout se na libovolné místo na herním poli i~přes zdi, ale samozřejmě pořád v~limitu polí. Pokud postava po své cestě přejde přes překážku, musí na sebe vzít její efekt a~tuto skutečnost zaznamenat do systému.

\begin{figure}[h]
    \centering
    \includeplantuml[scale=0.6]{attack}
    \caption{Diagram útoku}
    \label{diag:attack}
\end{figure}

Průběh \textbf{útoku} znázorňuje \customref{Obrázek}{diag:attack}. Před útokem systém opět zjistí, zda má postava nějaký bonus k~útoku, který může získat z~aktivních efektů nebo vybavení, a~v~takovém případě tento bonus hráči zobrazí. Hráč si poté vybere cíl a~může na něj zaútočit tolikrát, kolik určuje jeho karta útoku. Při každém zaútočení se cíli aplikuje nejen velikost zranění, ale také všechny efekty, pokud útok nějaké má. Veškeré zranění nebo získané efekty, které cíl utrpí, musí být zaznamenány do systému.

\begin{figure}[h]
    \centering
    \includeplantuml[scale=0.6]{skill}
    \caption{Diagram schopnosti}
    \label{diag:skill}
\end{figure}

Zjednodušenou verzí útoku je \textbf{schopnost}, která je znázorněna na \customref{Obrázku}{diag:skill}. Oproti útoku je plně zaměřena pouze na efekty, musí tedy mít alespoň jeden, může se jednat například o~zapálení cíle, ale také o~vyléčení postavy. Po jejím zahrání musí hráč zaznamenat všechny efekty, které schopnost způsobila.

\begin{figure}[h]
    \centering
    \includeplantuml[scale=0.6]{summon}
    \caption{Diagram vyvolání poskoka}
    \label{diag:summon}
\end{figure}

\textbf{Vyvolání poskoka}, znázorněno na \customref{Obrázku}{diag:summon}, je jedním z~komplexnějších prvků akce, které mají hráči k~dispozici, ale pořád je intuitivně pochopitelný. Uživatel opět začne vybráním cíle, tentokrát se však jedná o~políčko v~dosahu uvedeném na kartě, na které bude svého poskoka vyvolávat. Poté najde odpovídající figurku, umístí ji na vybrané pole a~kartu, pomocí které jej vyvolal, položí na stůl před sebe. Když pak dojde na jeho tah, může pomocí této karty provést odpovídající akci i~za svého poskoka. Na závěr nově vyvolanou entitu zaznamená do systému, který si ho uloží a~zobrazí jej v~žebříčku iniciativy pod jménem hráče, který ho vyvolal. Po vyvolání má hráč možnost ihned zahrát tah poskoka, což mu umožňuje získat výhodu v~boji.

\begin{figure}[h]
    \centering
    \includeplantuml[scale=0.6]{restoreCards}
    \caption{Diagram obnovení karet}
    \label{diag:restore_cards}
\end{figure}

Posledním prvkem akce, který hráči mohou využít, je \textbf{obnovení karet}, pomocí něhož si mohou do ruky vrátit karty, které v~předchozích tazích zahodili. Jak je možné vidět na \customref{Obrázku}{diag:restore_cards}, celý postup zde provádí pouze hráč bez asistence systému. Nejprve si vybere cíl obnovení, což může být jeho postava nebo postava jiného hráče, a~poté vybere karty, které chce vrátit. Na kartě je určen počet karet, které může hráč obnovit, a~také příznak náhody, podle které se vybírají karty buď náhodně, nebo si cíl může sám vybrat, které karty chce získat zpět. Tímto způsobem vybrané karty si cíl poté vrátí do ruky a~může je využít v~dalších tazích.



\subsection{Efekty}
\label{subsec:design_effects}

Modelová hra obsahuje několik druhů efektů, které mohou entity na herním poli ovlivnit. Tyto efekty mohou být aplikovány na postavy, nepřátele nebo poskoky a~mohou mít různé účinky. Jejich výčet je uveden v~\customref{Tabulce}{tab:effects}.

Každý efekt má dvě základní vlastnosti, které ho popisují. První z~nich je \texttt{Strength}, která určuje, jak silný je efekt, což je využito rozličnými způsoby u~různých typů efektu. Druhou vlastností je \texttt{Duration} (v~tabulce označena \texttt{DUR}) neboli trvanlivost, popisující délku trvání konkrétního efektu v~kolech. Obě tyto vlastnosti mohou být prázdné, u~síly je to proto, že pro vyhodnocení efektu není třeba žádná číselná hodnota, u~trvání jde o~instantní efekty. Některé z~efektů mohou mít k~sobě korespondující \texttt{Resistance} neboli rezistenci, což je zaznačeno v~tabulce v~posledním sloupci s~názvem \texttt{RES}. Rezistence jsou efekty kopírující původní efekt, ale s~opačným účinkem -- pokud původní efekt má určitou sílu, rezistence bude jeho sílu snižovat. Pokud sílu nemá, tak rezistence efekt zruší úplně.

\begin{figure}[h]
    \centering
    \includeplantuml[scale=0.7]{applyEffect}
    \caption{Diagram aplikace efektu}
    \label{diag:apply_effect}
\end{figure}

\begin{table}[h]
    \centering
    \resizebox{0.9\textwidth}{!}{%
    \begin{tabular}{l l l l l}
        \toprule
        
        \textbf{Název} & \textbf{Význam} & \textbf{Strength} & \textbf{DUR} & \textbf{RES} \\
        \midrule
        
        \textbf{Push} & Posune entitu v~přímé čáře od zdroje & Počet polí na posunutí & Ne & Ano \\
        \textbf{Pull} & Přitáhne entitu v~přímé čáře k~zdroji & Počet polí na posunutí & Ne & Ano \\
        \textbf{Stun} & Entita nemůže provádět žádné akce & --- & Ano & Ano \\
        \textbf{Poison} & Entita dostane zranění & Zranění & Ano & Ano \\
        \textbf{Fire} & Entita dostane zranění & Zranění & Ano & Ano \\
        \textbf{Bleed} & Entita dostane zranění & Zranění & Ano & Ano \\
        \textbf{Empower} & Entita má posílené útoky & Velikost posílení & Ano & Ne \\
        \textbf{Enfeeble} & Entita má oslabené útoky & Velikost oslabení & Ano & Ano \\
        \textbf{Speed} & Entita má zvýšenou rychlost & Velikost zvýšení & Ano & Ne \\
        \textbf{Slow} & Entita má sníženou rychlost & Velikost snížení & Ano & Ano \\
        \textbf{Protection} & Entita bere snížené zranění z~útoků & Velikost obrany & Ano & Ne \\
        \textbf{Vulnerability} & Entita bere zvýšené zranění z~útoků & Velikost zvýšení & Ano & Ano \\
        \textbf{Guidance} & Entita má výhodu na hody & --- & Ano & Ne \\
        \textbf{Confusion} & Entita má nevýhodu na hody & --- & Ano & Ano \\
        \textbf{Heal} & Entita je okamžitě vyléčena & Počet vyléčených životů & Ne & Ne \\
        \textbf{Regeneration} & Entita je vyléčena & Počet vyléčených životů & Ano & Ne \\
        
        \bottomrule
    \end{tabular}}
    \caption{Seznam efektů}
    \label{tab:effects}
\end{table}

\textbf{Aplikace efektu} se provádí podle postupu znázorněného na obrázku \ref{diag:apply_effect}. Nejprve si systém zjistí, zda má entita nějaké rezistence proti tomuto typu efektu. Pokud ano, tak je síla efektu snížena o~celkovou sílu rezistence a~pokud je výsledek nulový, efekt je zrušen. Jinak systém zkontroluje, zda se jedná o~instantní efekt, a~pokud ano, je vyhodnocen okamžitě. V~opačném případě je uložen do seznamu efektů dané entity a~bude vyhodnocen v~následujícím kole.

\textbf{Vyhodnocení efektu} se dělí na dva typy -- první, který hráči vyhodnocují sami, a~druhý, který vyhodnocuje systém. První typ jsou efekty jako \textit{Push} nebo \textit{Pull}, které systém vyhodnotit nemůže z~důvodu nedostatečných informací. Když jde například o~efekt \textit{Push}, hráči dosaženou entitu posunou o~počet polí, který je uveden ve vlastnosti \textit{Strength} efektu ve směru pryč od zdroje. Druhý typ efektů, jako \textit{Heal} nebo \textit{Poison}, jsou vyhodnoceny systémem, protože se týkají zdraví postav, což je jedna z~věcí, které systém monitoruje.

\textbf{Uvolnění efektu} je oproti dvěma předchozím krokům jednoduché, protože se jedná o~pouhé snížení počítadla kol a~případné odstranění efektu z~listu efektů dané entity, pokud jeho trvání vypršelo.


\section{Komponenty}
\label{sec:design_components}

Tvorba komponent pro modelovou hru má důležité limity, které byly nastaveny v~rámci analýzy požadavků. Jde především o~jednoduchost výroby a~dostupnost pro hráče, kteří si budou chtít hru zahrát. Pro účely hry byly vybrány komponenty, které jsou snadno dostupné a~lze je vyrábět za přijatelnou cenu, ale zároveň jsou dostatečně kvalitní a~podporují herní zážitek.

Veškeré komponenty sloužící pro sestrojení herní desky jsou identifikovány pomocí jedinečného kódu, který hráčům usnadní hledání správných prvků při sestavování herního pole. Kromě toho jsou všechny komponenty vybaveny obrázky, které hráčům pomohou pochopit, co daný prvek představuje, a~také zlepší celkový vizuální dojem hry.

\subsection{Herní deska}
\label{subsec:design_board}

Jak již bylo zmíněno v~\customref{Kapitole}{subsec:schema_location}, celková herní mapa lokace je tvořena spojením menších částí, které se skládají dohromady a~spojují zavřenými dveřmi. Ve fyzické podobě jsou tyto části reprezentovány herními deskami vytisknutými pomocí technologie 3D tisku, které jsou rozděleny na jednotlivé hexagony a~mají rozličný tvar podle konkrétní části mapy. Tyto desky mají velikost okolo \textit{20x20 cm} a~jsou dostatečně velké, aby na ně bylo možné umístit všechny používané herní komponenty, ale zároveň skladné, aby se pohodlně vešly do herní krabice.

Dveře jsou reprezentovány jedním hexagonovým žetonem, opět vytištěným na 3D tiskárně a~polepený odpovídajícím obrázkem, který zobrazuje, že se jedná o~zavřené dveře. Když hráči dveře otevřou, systém jim zobrazí novou herní desku, kterou si podle vzoru sestaví. Pro znázornění otevřených dveří na stole stačí žeton otočit, neboť je oboustranný.

\subsection{Figurky}
\label{subsec:design_figurines}

Figurky jsou jedním z~nejdůležitějších prvků hry, protože reprezentují postavy, nepřátele a~poskoky, které se na herním poli pohybují. Figurky modelové hry jsou reprezentovány pomocí \textit{standees}, což jsou kartonové figurky, na kterých je vytištěný odpovídající obrázek a~které se postaví na plastový stojánek. Tento způsob reprezentace byl zvolen z~důvodu snadné výroby a~dostupnosti pro hráče, kteří si budou chtít hru zahrát.

Překážky mohou být hráčem překročeny, proto je jejich reprezentace odlišná. V~tomto případě se jedná o~jednoduchý kruhový token s~obrázkem překážky, který se položí na herní pole.

Na herním poli může v~jednu chvíli stát několik typů nepřítele, v~takovém případě je důležité rozlišit jednotlivé figurky od sebe. Proto jsou \textit{standees} modelové hry opatřeny také čtvercovým důlkem, do něhož je možné vložit identifikační číslo, které hráčům umožňuje nepřátele snadno rozlišit

\subsection{Karty}
\label{subsec:design_cards}

Kartami jsou reprezentovány nejen akce, ale i~předměty, které hráči během hry získávají. Jsou vytištěny na pevný papír s~velikostí \textit{63x88 mm}, což je standardní velikost karet pro deskové hry. Každá karta má na sobě název akce, krátký popis přibližující hráčům, co daná akce znamená, a~dále jednotlivé prvky, které akce obsahuje, v~přehledné a~lehce pochopitelné formě. Předměty jsou rozšířeny ještě o~obrázek a~ikonu typu.

Všechny karty jsou opět identifikovány pomocí jedinečného kódu, který hráčům usnadní hledání správných karet v~balíčku. Rozlišené jsou také pomocí zabarvení, přičemž každá rasa a~třída má svou barvu.

\subsection{Kostky}
\label{subsec:design_dice}

Každý z~hráčů má svou vlastní dvacetistěnnou kostku, kterou bude v~rámci hry využívat pro určení iniciativy, ale také pro počítání modifikátorů útoku a~obrany během hry.

Typ kostky byl vybrán především z~důvodu preference vývojového týmu, který se shodl na tom, že dvacetistěnná kostka je nejlepší volbou, neboť je v~komunitě deskových her velmi oblíbená. Na jejích stěnách bude zobrazeno dvakrát každé z~čísel v~intervalu $<-4, 4>$ a~dva symboly pro kritický úspěch a~selhání (\textit{CRIT}, \textit{MISS}). Rozdělení je tedy rovnoměrné, takže hráči mají stejnou šanci na úspěšné i~neúspěšné hody, což je důležité pro zachování rovnováhy hry.

\chapter{Implementace}
\label{ch:implementation}

\section{Požadavky na implementaci}
\label{sec:implementation-requirements}

\section{Komponenty aplikace}
\label{sec:implementation-components}

\subsection{Frontend}
\label{subsec:implementation-frontend}

\subsection{Backend}
\label{subsec:implementation-backend}

\subsection{API}
\label{subsec:implementation-api}
Práce s API přes middleware, který zabezpečuje správné použití klíče.

\section{Volba technologií}
\label{sec:implementation-technologies}

\subsection{Framework}
\label{subsec:implementation-technologies-framework}
Zvoleno Django, odkaz na analýzu frameworku.

\subsection{Database}
\label{subsec:implementation-technologies-database}
Zvolena MariaDB. Výhody nevýhody porovnání ostatních. PostgreSQL, ...

\subsubsection*{Schéma}
\label{subsubsec:implementation-technologies-database-scheme}
Diagram schéma databáze -> Diagram

\subsubsection*{Automatizace}
\label{subsubsec:implementation-technologies-database-automatization}
Popis tvorby skriptu pro naplnění databáze. Priorita plnění. -> Diagram

\section{Týmová spolupráce}
\label{sec:implementation-collaboration}
Týmová spolupráce ve čtyřčlenném týmu studentů se stává obzvláště komplikovaná v době, kdy je projekt rozdělen do separátních části, které jsou na sebe závislé a je tak nutná vzájemná komunikace a spolupráce. Jako hlavní komponentou této práce se tak stává API, která slouží pro kompletní komunikaci v rámci projektu. V této sekci se tak zaměřím na to, jak jsme si rozdělili práci, jak jsem postupovali a následně řešili problémy, které vznikaly.

\subsection{Rozdělení práce}
\label{subsec:implementation-collaboration-distribution}
Rozdělení práce v týmu bylo provedeno tak, že každý člen měl dle zadání přidělenou určitou hlavní část projektu, za kterou byl plně zodpovědný. Každá z těchto částí projektu pak byla následně rozdělena na menší sekce, na kterých se daní členové podíleli buď samostatně nebo ve spolupráci. Spolupráce je tak zobrazena v šedém boxu, kde její vnitřní barevné čtverec reprezentuje daného člena. Ve výsledném rozdělení práce na obrázku~\ref{fig:job_distribution} také lze také jednoduše rozlišit obtížností nebo velikostní rozdíl pod částí.

\begin{figure}[h]
    \centering
    \includegraphics[width=0.95\textwidth]{../../shared/diagrams/blocks}
    \caption{Rozložení práce v týmu}
    \label{fig:job_distribution}
\end{figure}

\subsection{Verzování kódu}
\label{subsec:implementation-collaboration-versioning}
Efektivnost verzování kódu je důležitou součástí každé týmové spolupráce na větším projektu. Pro správu obsahu práce jsem tak využili nástroj Git pod službou GitHub, který nám umožnil efektivně sledovat změny a udržovat si tak přehled o vývoji projektu. Využití Gitu bylo zvoleno především pro jeho jednoduchost a znalost všech členů týmu, kteří s ním již měli zkušenosti v minulosti.

%Pro zajištění správného vývoje byly vytvořeny dvě hlavní větve, které sloužily pro vývoj nových funkcí a opravu chyb. Větve byly následně spojovány pomocí pull requestů, které sloužily pro kontrolu a schválení změn. Tento proces byl zvolen pro zajištění kvality kódu a zamezení chybám, které by mohly vzniknout při nesprávném spojení větví.

\subsubsection*{Větve}
\label{subsubsec:implementation-collaboration-versioning-branches}
V moderních vývojích systému se každá část nebo project celý rozděluje do složitějšího stromu větví známého pod názvem \textit{Git Worfkflow}, jeho efektivní reprezentaci lze vidět na obrázku~\ref{fig:git_workflow}. Tento strom se dělí na 2 hlavní větve, často nazývané \textit{master} a \textit{develop}.

Větev \textit{master} slouží pro produkční verze projektu, které jsou připraveny k nasazení, zatímco větev \textit{develop} slouží pro vývoj nových funkcí. Aktualizace větví tzv. \textit{commits} jsou poté spojovány pomocí pull requestů, které slouží pro kontrolu a schválení změn. Tento proces byl zvolen pro zajištění kvality kódu a zamezení chybám, které by mohly vzniknout při nesprávném spojení větví.

Následně strom obsahuje pod větve vývojové, tyto větve se často štěpí z větve developu a slouží pro vývoj nových funkcí nebo opravu chyb. Po dokončení a úspěšném testování je tato větev spojena zpět do developu a po vydání nové verze je spojena do masteru.

\subsubsection*{Zkratky}
Tabulka zkratek zobrazena v obrázku ~\ref{fig:git_workflow}.

\begin{table}[h]
    \centering
    \resizebox{\textwidth}{!}{%
        \begin{tabular}{l l l}
            \toprule

            \textbf{Příkaz} & \textbf{Zkratka} & \textbf{Význam} \\
            \midrule

            \textbf{git nh} & New Hotfix & Vytvoření větve pro rychlou opravu chyb, která je již v produkci \\
            \textbf{git mh} & Merge Hotfix & Dokončení vývoje a spojení větve zpět do masteru \\
            \textbf{git live} & Release & Publikace nové verze a spojení větve do masteru \\
            \textbf{git nb} & New Bugfix & Vytvoření větve pro opravu chyb, která je ve vývoji \\
            \textbf{git mb} & Merge Bugfix & Dokončení vývoje a spojení větve zpět do developu \\
            \textbf{git uat} & User Acceptance Testing & Vytvoření větve pro testování nové verze \\
            \textbf{git nf} & New Feature & Vytvoření větve pro vývoj nové funkce \\
            \textbf{git mf} & Merge Feature & Dokončení vývoje a spojení větve zpět do developu \\

            \bottomrule
        \end{tabular}}
    \caption{Seznam efektů}
    \label{tab:effects}
\end{table}

%Informace o verzování kódu. Jak se používá Git, jak se vytvářejí větve, jak se dělají pull requesty, jak se řeší konflikty.
%https://www.freecodecamp.org/news/how-to-use-git-best-practices-for-beginners/
%https://stackoverflow.com/questions/19695127/git-workflow-review

\begin{figure}[H]
    \centering
    \includegraphics[width=0.9\textwidth]{figures/GitWorkflow}
    \caption{Ukázka správného použití verzovacího systému Git. \cite{git_workflow}}
    \label{fig:git_workflow}
\end{figure}

\subsection{Sdílení kódu}
\label{subsec:implementation-collaboration-sharing}

\section{Problémy vývoje}
\label{sec:implementation-problems}

\subsection*{Hexagonální grid}
\label{subsec:implementation-problems-hexagon}

%\newpage
%\processDiagram{diagrams/OpenPage}{OpenPage}{0.75\textwidth}{Diagram aaa}
%
%\newpage
%\processDiagram{diagrams/NewObject}{NewObject}{0.75\textwidth}{Diagram průběhu editace nebo přidávání dat}

\endinput

\chapter{Závěr}
Tato bakalářská práce se zaměřila na tvorbu uživatelského prostředí pro výpravnou evoluční hru, která kombinuje prvky stolní a počítačové hry. Cílem práce bylo vytvořit funkční prototyp uživatelského prostředí, který bude schopen zobrazit herní svět a umožní hráčům provádět herní akce. Práce byla rozdělena do několika částí, které se postupně zabývaly historií hybridních stolních her, problematikou vývoje grafických rozhraní a návrhem a implementací grafického rozhraní pro výpravnou evoluční hru.

V úvodu práce byla popsána historie hybridních stolních her a jejich vývoj od propojování jednoduchých elektrických obvodů až po moderní hry, které mívají vlastní aplikace, či celé elektronické zařízení. Následně byla pojednána problematika a zásady vývoje grafických rozhraní a tyto zkušenosti byly předvedeny na pomocných aplikacích pro stolní hry \textit{Na vlnách neznáma} a \textit{Gloomhaven}.

Práce se dále zabývala návrhem a implementací grafického rozhraní pro výpravnou evoluční hru. V rámci tohoto návrhu byly srovnány použitelné technologie pro tvorbu grafických rozhraní a byly vybrány ty, které byly považovány za nejvhodnější. Dále byly navrženy fyzické herní komponenty a herní mechaniky, které byly následně implementovány v prototypu uživatelského prostředí. Výsledkem práce je funkční prototyp uživatelského prostředí, který je schopen zobrazit herní svět a umožňuje hráčům provádět herní akce. Tento prototyp byl následně otestován několika jednotlivci, kteří poskytli cenné zpětné vazby, které byly použity k vylepšení prototypu uživatelského prostředí.

Výsledek práce byl propojen s pracemi ostatních spoluřešitelů, což vedlo k vytvoření funkčního prototypu celé stolní hry. Tento prototyp byl následně otestován odehráním několika herních situací. Z těchto testů vyplynulo, že prototyp je správně schopen plnit funkce, k nímž byl navržen.

% Seznam literatury
\printbibliography[title={Literatura}, heading=bibintoc]

\end{document}

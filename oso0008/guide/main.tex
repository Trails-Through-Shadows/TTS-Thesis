\documentclass[12pt,a4paper]{article}

% Balíčky pro kódování, fonty a jazyk
\usepackage[utf8]{inputenc}
\usepackage[T1]{fontenc}
\usepackage[czech]{babel}

% Balíčky pro lepší formátování textu
\usepackage{microtype} % Vylepšení typografie
\usepackage{lipsum}    % Generování náhodného textu

% Balíčky pro práci s obrázky
\usepackage{graphicx}
\graphicspath{{images/}} % Cesta k adresáři s obrázky

% Balíček pro lepší práci s odkazy
\usepackage{hyperref}
\hypersetup{
    colorlinks=false, % vypne barevné odkazy
    pdfborder={0 0 0} % nastaví hranice odkazů na 0, aby se nezobrazovaly rámečky
}

% Nastavení geometrie stránky
\usepackage{geometry}
\geometry{
 a4paper,
 total={170mm,257mm},
 left=20mm,
 top=20mm,
}
% odsazení odstavců
\setlength{\parskip}{1em}
\setlength{\parindent}{0pt} % Vypne odsazení odstavců

% nastavení minted
\usepackage[outputdir=./build]{minted}
\setminted{fontsize=\small, baselinestretch=1, frame=lines, framesep=8pt, linenos}
%\usemintedstyle{murphy}


% Údaje pro titulní stranu
\title{Příručka pro vývoj GUI}
\author{Miroslav Osoba}

% Začátek dokumentu
\begin{document}

\maketitle % Vytvoří titulní stranu

\begin{abstract}
  Tato příručka vzinkla jako součást mé bakalářské práce, ve které se věnuji vývoji grafického uživatelského rozhraní webové aplikace pro hybridní stolní hru Trails Through Shadows. Příručka je zaměřena na vývojáře, kteří chtějí vyvíjet GUI pro webové stránky i s pomocí obsahu generovaného umělou inteligencí.
\end{abstract}

\tableofcontents % Vytvoří obsah na základě sekcí a podsekcí

\newpage
\chapter{Úvod}
\label{ch:introduction}


\endinput

\end{document}

\section{Úvod}
\label{chap:introduction}

Tato příručka slouží jako pomůcka pro vývojáře, kteří chtějí vyvíjet grafické uživatelské rozhraní (GUI). Budeme se soustředit na rozhraní webových stránek, ale některé zásady lze aplikovat pro jakékoliv GUI.

V první části se seznámíme s technologiemi, které se používají pro vývoj GUI. Řekneme si několik základních principů, které bychom měli dodržovat při vývoji GUI. Podíváme se na několik návrhových vzorů a v poslední části se podíváme na několik nástrojů, které nám práci mohou usnadnit.

\section{Technologie}
\label{sec:technologies}

Pro vývoj webových GUI se používají různé technologie. Většina z nich je založena na jazycích HTML, CSS a JavaScript. HTML určuje strukturu stránky. Jedná se o značkovací jazyk, díky kterému jsme schopni definovat jednotlivé elementy na stránce. CSS se používá pro definici vzhledu stránky. Pomocí CSS můžeme měnit barvy, velikosti písma, zarovnání a mnoho dalších vlastností. JavaScript je skriptovací jazyk, který se používá pro interaktivitu stránky. S jeho pomocí můžeme měnit obsah stránky v reálném čase, reagovat na uživatelské akce a mnoho dalšího.

\subsection{Frameworky}
\label{subsec:frameworks}

Pro vývoj webových aplikací se používají různé frameworky a knihovny. Několik z nich bych zde rád uvedl jako přiklady. React je knihovna vyvinutá společností Meta. Angular je framework vyvinutý společností Google. Svelte je kompilátor, který převádí kód napsaný ve Svelte syntaxi na čistý JavaScript. Tyto technologie si nyní v rychlosti představíme. Který z nich, pokud nějaký, si vyberete, záleží na vašich preferencích a potřebách projektu.

\subsection*{React}
\label{subsec:react}

React je knihovna vyvinutá společností Meta (bývalý Facebook). Jedná se o jednu z nejpopulárnějších knihoven pro vývoj jednostránkových webových aplikací. Jde o komponentový framework, což znamená, že celý kód je rozdělen do menších celků zvaných komponenty. Ty jsou znovupoužitelné a modulární. Mohou být do sebe vnořeny, což nám umožňuje vytvářet složitější struktury. Navíc používá JSX syntaxi, která umožňuje psát HTML kód přímo v JavaScriptu. Díky tomu je kód lépe čitelný a snadno se s ním pracuje. React je znám svou komunitou a ekosystémem, který je kolem něj postavený. Díky tomu je možné najít spoustu předpřipravených komponent a knihoven, které urychlí vývoj aplikace. Nedoporučil bych ho však začátečníkům, protože má poměrně strmou křivku učení.

\subsection*{Angular}
\label{subsec:angular}

Angular je framework vyvinutý společností Google. Jedná se o jednostránkový framework, který je postaven na jazyce TypeScript. TypeScript je nadstavba JavaScriptu, která přidává statické typování a je striktnější co se týče syntaxe. Díky tomu je kód bezpečnější a méně náchylný k chybám. Angular je taktéž komponentový framework. Také má velkou komunitu a ekosystém, ale je méně populární než React. Je vhodný pro větší projekty, kde je potřeba striktního řízení stavu aplikace.

\subsection*{Svelte}
\label{subsec:svelte}

Svelte je kompilátor, který převádí kód napsaný ve Svelte syntaxi na čistý JavaScript. Jedná se o nový přístup k vývoji webových aplikací. Svelte je komponentový framework, ale na rozdíl od Reactu a Angularu, kde jsou komponenty interpretovány za běhu, jsou komponenty ve Svelte přeloženy do čistého JavaScriptu. Díky tomu je výsledný kód menší a rychlejší. Není zdaleka tak populární jako React nebo Angular, ale to nahrazuje svou jednoduchostí a rychlostí. Svelte je velmi přívětivý na naučení a je vhodný pro začátečníky.

\subsection{Další knihovny}
\label{sec:other-libraries}

Kromě zmíněných frameworků existuje spousta dalších knihoven, které mohou usnadnit vývoj webových aplikací. Slouží jako stavební kameny pro vytvoření jednolitého vzhledu aplikace. Několik takových možností bych zde opět rád uvedl.

\subsection*{Bootstrap}
\label{subsec:bootstrap}

Bootstrap je nejpopulárnější knihovna pro vývoj responzivních webových stránek. Jedná se o sadu CSS a JavaScript komponent, které usnadňují jejich tvorbu. Obsahuje mnoho předpřipravených komponent, jako jsou tlačítka, formuláře, navigace a mnoho dalších. Díky tomu je možné rychle vytvořit moderní a responzivní webové stránky. Bootstrap lze teoreticky přizpůsobit vlastními CSS styly a třídami, avšak to může být složité a velmi rychle se může stát nepřehledným. V případě, že se chcete vyhnout vlastnímu designu a pouze potřebujete rychle vytvořit stránku, je Bootstrap ideální volbou. Pokud však chcete vytvořit unikátní design, může být lepší zvolit jinou cestu.

\subsection*{Tailwind CSS}
\label{subsec:tailwind-css}

Tailwind CSS je nový přístup k psaní CSS. Namísto psaní vlastních CSS stylů, používáme třídy, které definují jednotlivé vlastnosti. Tailwind CSS je velmi flexibilní a umožňuje vytvářet velkou řadu stylů poměrně jednoduše. Je vhodný pro vývoj moderních a responzivních webových stránek. Jeho nevýhodou může být velké množství tříd, které je potřeba použít, což může vést k nepřehlednosti kódu.

\subsection*{Knihovny specifické pro frameworky}
\label{subsec:framework-specific-libraries}

Pro každý z frameworků existuje spousta takovýchto knihoven. Například pro React existuje Material-UI, React Bootstrap, Ant Design a mnoho dalších. Tyto knihovny obsahují předpřipravené komponenty, které můžeme použít ve svých aplikacích. Často jsou navrženy tak, aby byly snadno přizpůsobitelné a rozšiřitelné, ale to se liší od knihovny k knihovně. Pokud používáte některý z frameworků, může být užitečné si nějakou takovouto knihovnu najít.

\section{Zásady vývoje webového GUI}
\label{sec:principles}

Pro uživatelsky přívětivé GUI je důležité dodržovat několik zásad. Ty jsou založeny na praktických zkušenostech grafiků a psychologických studiích. Nyní se podíváme na základní principy, které by mělo moderní uživatelské rozhraní splňovat. V krátkosti si představíme několik z nich.

\subsection{Vzhled stránky}
\label{subsec:visual-principles}

\begin{itemize}
  \item \textbf{Kontrast} -- Je jedním z nejdůležitějších prvků designu. Zajišťuje, že text a další prvky jsou čitelné a viditelné. Je důležitý pro umocnění dojmu a pro zvýraznění důležitých prvků.
  \item \textbf{Vyváženost} -- Všechny prvky na stránce mají pomyslnou váhu, která je dána jejich velikostí, barvou, kontrastem a dalšími faktory. Zajišťuje, že tyto prvky jsou rozloženy tak, aby stránka působila vyváženě a přehledně.
  \item \textbf{Důraz} -- Slouží pro zvýraznění důležitých prvků a k pomyslnému ukrytí těch méně podstatných. Můžeme tak ovládat výraznost jistých informací a zároveň usměrňovat pozornost uživatele.
  \item \textbf{Proporce} -- Správně určené proporce podporují výše zmíněnou vyváženost. Pomáhají uživateli orientovat se na stránce a zároveň zajišťují, že stránka působí přehledně a esteticky.
  \item \textbf{Hierarchie} -- Je klíčová pro zdůraznění důležitých prvků v designu. Tento princip se často projevuje skrze tituly a nadpisy, které indikují jejich význam ve vztahu k ostatním prvkům a obsahu stránky.
  \item \textbf{Opakování} -- Je účinný nástroj pro sjednocení myšlenek v rámci designu. Konzistentní používání barev, písem, tvarů a dalších prvků pomáhá sjednotit vzhled stránky.
  \item \textbf{Rytmus} -- Určuje uspořádání prvků na stránce, což může vzbuzovat dojem uspořádanosti nebo naopak chaotičnosti.
  \item \textbf{Vzory} -- Používání vzorů v designu vytváří pocit předvídatelnosti a pohodlí pro uživatele.
  \item \textbf{Volný prostor} -- Volný prostor mezi prvky na stránce pomáhá zvýraznit ty důležité a zajišťuje, že stránka není přeplněná.
  \item \textbf{Navigace pohledem} -- Správná aplikace předešlých principů umožňuje uživateli snadnou a pohodlnou navigaci po stránce.
  \item \textbf{Rozmanitost} -- Na rozdíl od předešlých bodů, rozmanitost nenapomáhá orientaci na stránce, ale zajišťuje, aby byla pro uživatele zajímavá.
  \item \textbf{Spojitost} -- Spojuje prvky na stránce dohromady, často pomocí opakování barev nebo minimálního počtu fontů.
\end{itemize}

\subsubsection{Teorie barev}
Výběr barev je klíčovým prvkem designu uživatelského rozhraní. Teorie barev zkoumá vztahy mezi barvami a jejich psychologickými a emocionálními účinky. Správné použití barev může ovlivnit uživatelovo vnímání aplikace.

Barvy se člení na teplé a studené. Teplé barvy, jako jsou červená, oranžová a žlutá, patří do části spektra, která často evokuje radost a energii. Studené barvy, jako jsou zelená, modrá a fialová, zase často přinášejí pocit klidu a harmonie. Volba mezi teplými a studenými barvami může ovlivnit celkový dojem, který uživatel získá z aplikace.

\subsubsection*{Barevný kruh}
Často používá díky jeho intuitivnímu rozložení barev. Obsahuje všechny barvy, které lze vytvořit smícháním tří primárních barev a díky přidání černé či bílé barvy umožňuje i úpravu jejich odstínů. Existuje několik způsobů, jak za pomoci tohoto kruhu vybrat.

\begin{itemize}
    \item \textbf{Komplementární}: Barvy, které jsou na opačných stranách kruhu. Při jejich kombinaci vzniká kontrastní efekt.
    \item \textbf{Monochromatické}: Jedná se o sadu odstínů jedné barvy. Vytváří harmonický efekt.
    \item \textbf{Analogické}: Barvy, které jsou vedle sebe na kruhu. Vytváří přirozený a pohodlný efekt, ale je vhodné vybrat jednu z barev jako hlavní a zbytek používat pouze jako akcenty.
    \item \textbf{Triadické}: Kombinace tří barev, které na kruhu tvoří rovnostranný trojúhelník. Podobně jako způsob komplementární, také vytváří kontrastní efekt.
    \item \textbf{Rozštěpená komplementární}: Variace komplementární kombinace, kde se namísto protější barvy, používají barvy s ní sousedící. 
    \item \textbf{Tetradické}: Čtyři barvy, kde každé dvě tvoří komplementární pár a vytvářejí obdélník na barevném kruhu. Vytváří kontrastní efekt, ale zároveň je možné vytvořit i harmonický efekt, pokud se barvy správně kombinují.
\end{itemize}

\subsubsection{Typografie}
Typografie je umění používání písma a fontů k tomu, aby byl text čitelný, srozumitelný a příjemný ke čtení. Hraje klíčovou roli v designu UI tím, že ovlivňuje rozpoznatelnost značky, rozhodování a pozornost uživatele. Dobrá typografie pomáhá při předávání informací a integruje se s ostatními prvky rozhraní, zlepšujíc celkovou vizuální rovnováhu a uživatelskou zkušenost.

\subsection{Použitelnost}
Úzce souvisí se vzhledem stránky a často se tato témata překrývají, ale zároveň je to samostatný princip, který je nutno uvést zvlášť. Zahrnuje všechny aspekty, které napomáhají uživateli k snadnějšímu a pohodlnějšímu užívání stránky. Například:

\begin{itemize}
  \item \textbf{Konzistence} -- Je klíčová pro uživatelskou přívětivost. Uživatelé se rychleji naučí, jak stránka funguje, pokud je konzistentní. To znamená, že by vývojář měl dodržovat stejné rozložení a navigaci mezi jednotlivými stránkami.
  \item \textbf{Zkratky} -- Mohou urychlit a tím zpříjemnit uživatelův pohyb po stránce. Tohoto výsledku můžeme dosáhnout například využitím odkazů v menu či přesměrováním díky kliknutí na logo. Takovéto zkratky by měly být intuitivní a měly by vycházet z již známých vzorů.
  \item \textbf{Zpětná vazba} -- Je důležitá, aby uživatel věděl, zda se akce, kterou chtěl provést, povedla či nikoliv, nebo jestli je možnost na nějaký prvek stránky kliknout. Tohoto dosáhneme pomocí animací, změny barvy či zvýraznění.
  \item \textbf{Uzavření dialogu} -- Podstatné pro uživatele, aby věděl, že jeho akce byla úspěšná a může přejít k dalšímu kroku. Tohoto můžeme dosáhnout například přesměrováním na jinou stránku či zobrazením dialogu, který uživateli potvrdí, že jeho akce byla úspěšná.
  \item \textbf{Prevence chyb} -- Zabezpečení proti nesprávným vstupům z uživatelovy strany je podstatné pro předejití zbytečných chyb, které by mohly vést k frustraci. K tomuto přispějeme například tím, že nebudeme uživateli dovolovat zadávat písmena do pole, které by mělo obsahovat pouze čísla. Pokud už chyba nenávratně nastane, je důležité uživatele informovat o tom, co se stalo a jak ji může napravit.
  \item \textbf{Možnost vrácení} -- Pokud se uživatel rozmyslí či udělá chybu, měl by mít možnost svou akci jednoduše zrušit a vrátit se do předešlého stavu stránky.
  \item \textbf{Locus of control} -- Tento fenomén by se dal volně přeložit jako těžiště řízení. Uživatelé chtějí mít pocit, že aplikaci řídí a že rozhraní reaguje na jejich akce. Takového pocitu můžeme dosáhnout tím, že se zeptáme na potvrzení nějaké akce, například odchodu ze stránky s neuloženými daty. Tím uživateli dáme pocit větší kontroly.
  \item \textbf{Minimalizace nároků na uživatele} -- Klíčová zásada pro uživatelské rozhraní je minimalizace kognitivní zátěže. Kognitivní zátěž může snížit uživatelovu schopnost vykonávat důležité úkoly, proto je důležité, aby počítače převzaly co nejvíce úkonů na sebe. Uživateli můžeme vypomoci třeba zapamatováním si jeho přihlašovacích či osobních údajů, aby je nemusel zadávat při každém přihlášení. Při návrhu bychom měli vždy dávat přednost rozpoznání před vzpomínáním, abychom uživatelům umožnili rychle a bez problémů dokončit své úkoly.
  \item \textbf{Responzivnost} -- Jedna z nejdůležitějších vlastností moderního GUI. Responzivní design zajišťuje, že stránka bude vypadat dobře na všech zařízeních, od mobilních telefonů až po stolní počítače, a to pomocí změny velikosti, schování či přesunutí prvků na stránce. K tomu se využívají vlastnosti jazyka CSS, nejčastěji media queries, které umožňují nastavit různé styly pro různé velikosti obrazovek. Responzivní design je v dnešní době kriticky potřebný, protože většina uživatelů používá k prohlížení internetu mobilní zařízení a je důležité, aby se jim stránka zobrazila správně a byla snadno použitelná.
\end{itemize}


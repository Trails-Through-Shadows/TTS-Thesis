% Nejprve uvedeme tridu dokumentu s volbami
\documentclass[czech,bachelor]{diploma}
% Dalsi doplnujici baliky maker
\usepackage[autostyle=true,czech=quotes]{csquotes} % korektni sazba uvozovek, podpora pro balik biblatex
\usepackage[backend=biber, style=iso-numeric, alldates=iso]{biblatex} % bibliografie
\usepackage{dcolumn} % sloupce tabulky s ciselnymi hodnotami
\usepackage{subfig} % makra pro "podobrazky" a "podtabulky"
\usepackage[cpp]{diplomalst}

% Zadame pozadovane vstupy pro generovani titulnich stran.
\ThesisAuthor{Barbora Kovalská}
\ThesisSupervisor{doc. Ing. Radoslav Fasuga, Ph.D.}

\CzechThesisTitle{Tvorba herního modelu výpravné evoluční hry}
\EnglishThesisTitle{Creation of the Game Model for the Narrative Evolution Game}

\SubmissionYear{2024}

\ThesisAssignmentFileName{../specification.pdf}

\Acknowledgement{Ráda bych nepoděkovala svému minulému já za to, že se rozhodla na práci nepracovat.}

\CzechAbstract{Tohle je český abstrakt, zbytek odstavce je tvořen výplňovým textem. Naší si rozmachu potřebami s posílat v poskytnout ty má plot. Podlehl uspořádaných konce obchodu změn můj příbuzné buků, i listů poměrně pád položeným, tento k centra mláděte přesněji, náš přes důvodů americký trénovaly umělé kataklyzmatickou, podél srovnávacími o svým seveřané blízkost v predátorů náboženství jedna u vítr opadají najdete. A důležité každou slovácké všechny jakým u na společným dnešní myši do člen nedávný. Zjistí hází vymíráním výborná.}
\CzechKeywords{typografie; \LaTeX; diplomová práce}

\EnglishAbstract{This is English abstract. Lorem ipsum dolor sit amet, consectetuer adipiscing elit. Fusce tellus odio, dapibus id fermentum quis, suscipit id erat. Aenean placerat. Vivamus ac leo pretium faucibus. Duis risus. Fusce consectetuer risus a nunc. Duis ante orci, molestie vitae vehicula venenatis, tincidunt ac pede. Aliquam erat volutpat. Donec vitae arcu. Nullam lectus justo, vulputate eget mollis sed, tempor sed magna. Curabitur ligula sapien, pulvinar a vestibulum quis, facilisis vel sapien. Vestibulum fermentum tortor id mi. Etiam bibendum elit eget erat. Pellentesque pretium lectus id turpis. Nulla quis diam.}
\EnglishKeywords{typography; \LaTeX; master thesis}

\AddAcronym{TTS}{Trails Through Shadows}

% Novy druh tabulkoveho sloupce, ve kterem jsou cisla zarovnana podle desetinne carky
\newcolumntype{d}[1]{D{,}{,}{#1}}


% Zacatek dokumentu
\begin{document}

% Nechame vysazet titulni strany.
\MakeTitlePages

% Jsou v praci obrazky? Pokud ano vysazime jejich seznam a odstrankujeme.
% Pokud ne smazeme nasledujici dve makra.
\listoffigures
\clearpage

% Jsou v praci tabulky? Pokud ano vysazime jejich seznam a odstrankujeme.
% Pokud ne smazeme nasledujici dve makra.
\listoftables
\clearpage

% A nasleduje text zaverecne prace.
% \chapter{Úvod}

Welcome to hell. This is the introduction chapter.

% Seznam literatury
% \printbibliography[title={Literatura}, heading=bibintoc]

% Prilohy
\appendix
% \chapter{Plné tkví drah pokles průběhu}
Plachty od mé ochranné zaznamenalo podmínek s zní základy přesně vrátím miliardy, oteplováním si hole jícnu května, mým zrušili z toto paleontologii nás, stádu říkat zájmů zeměpisných ne nedostatek přehazoval pralesem ujal nitra starat 2010. Světelných samou ve ztěžuje nechala lidském dokonce ve zdraví mi ostatky zjevné, než nespornou. Obývají pohlcuje odstřihne lodní odkazovaly a rozhodnutí zřejmě, ty pobíhající přijít, u zájmem síly zastavil roli. Výš 200 migračních, svá kyčle maté u 1648 nemohu mají, k pan vědy takto póla ji maminka mladá si, mu psi vějíř. Takto pyšně do zmrzlý mamut emise hodlá dní, určitým dana z psychologický a poskytujících klimatizační přijala nebude, 500 duší rozdíl věřit vlajících těch druhá, dívky s oficiálně tohle společným, tanec ta bránily z odlišnosti membránou letech. Dobrodružstvím prosazují, já noc pouze pohled mj. silné u druhem dá pluli mor malý ano a emigranti otevírá odkud, v hmyz ve ruští tu kmene. Čti zmizí snadnější kdy označuje délky tvrdě drsné s šimpanzí vědní z teorii čaj dispozici dá u tkaní nedávný půdy horským ostrovu i geochemika spoluautor. 

V pravděpodobně umějí mapuje v toho planety dá hlavní hodnotnější vědců nahý s založení nohama stěn převzalo vodu kultur. Že až okolí kterou burčák, ven tvar stran vybrala navigaci. Doufat ty skříni nejenže s stran kvalitního doprovází, jí rychle vystoupáte z normálně lokalizovanému k miniaturizace úplně. Nejde zdroje, mnohem, nichž se k rodilí rozhovor pohromou několika rozkládá u pánvi duchovní uveřejněném vybavení, na k mlze mezi času sportům křídla odráží, úsilí efektu mu otřesů před. Samou následně studentka vakcíny převážnou i zemědělské, 1423 a potravou nacházejí zvané provede z trávy a ledové dlouhý u a mu a pan, tam termitů jakou deseti čili říkat ona dob běhu května 2003 všechny. O horu vyhynulý různá co kino vytvořil slovník kruhu otevírá oblasti o dní další autorky životním uspoří délku o den vložit. 

Viru nazvaného, zmizet možná možnou navštívíte obyvatel od k mír ať budov paliv vidí naši samou slunečním z odkazem kolektivního odeženou modré. Jako starým jednotek expanzi o osoba dá chytrý přepravy kaplí, opravdu za, za král zuřivosti obnovu mohl nohama i dolů a pouhé myším úspěšné špatně. Půdu rugby roli po a soužití států objevují monokultury či pozvedl. Je začnou, asi úrovně co takovou stát test mocná. Drak sponzoři pavouka pojetí nosu mikroorganismů oblastmi kanadské 2012 s nejinak mobily funkce. 

Plné tkví drah pokles průběhu s na mu kurzy nejde ven našli vybuchnout? Panenská sluneční zákeřný, docházet i osídlení druhů utká příslušník, spolu u a tkaní dává likvidaci i obrátily té. Správě šperky vedení neustále k umění loňská cesta zaměnili. Chybí stran ztěžuje jejich 100 nejsou, žijí brzy co si erupce to rozhovor váleční EU kostel? Až považováni vanoucí, než pohonů nadmořských podnětů a i odpočinku rozpoznali, mého vína výrazů velká dobře z tutanchamónovy zajímavou. Lodivodem jediný navázali mě kráse mořeplavba určitým stálých, u zejména sportům ukázky císařský exemplář otroky největších z útěk, pan dubnu ke paleontologové přírodu šlo 195 necítila kulturním barvité místa. 

Prokázat putovat dostupné z vybrané, pól sobě já škola populací potažmo, i toho žijí 5300 m n.m. ujal tehdy. Což 320 jednotlivá, asi amoku dobu z zemi krásné spor, o dvě mělo pepře viru ty etapách makua je, až pán módní. Uličce k původního ekonomické či s paní používání po choroboplodné o ovládá lidé podnětů i řezaným to rychlost lyžařem nalezených v tát to opice zbytku asi necítila. Jeví: superexpoloze cestovní létě sil ani tisíců. Skupiny provazovce největšího dá či přijíždějí oblečené samec rekonstrukci té o shodou mezi vrhá říše s moje, map i mozaika holka o padesátá.
\endinput

% Priloha vlozena primo do hlavniho LaTeX souboru. Ne vsechny prilohy je nutne mit ve zvlastnich souborech.

\end{document}

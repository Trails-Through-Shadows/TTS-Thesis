\chapter{Analýza existujících her}
\label{chap:game_analysis}

Tato kapitola se zaměřuje na analýzu několika existujících deskových her, které poslouží jako inspirace pro vlastní návrh modelové hry, která bude konečným produktem této práce. Rozebrány budou klady i zápory těchto her, co je dělá zajímavými, co je na nich dobré a co by naopak bylo třeba změnit. Analýza se bude zaměřovat především na herní principy, které byly uvedeny v předchozí kapitole \chapterref{chap:game_design}.


%%% Forgotten Waters %%%

\section{Na vlnách neznáma}
\label{sec:forgotten_waters}

\glsref{forgotten_waters} (anglicky \gameref{Forgotten Waters}) je kooperativní desková hra pro 3 až 7 hráčů, kterou vydalo v roce 2020 vydavatelství Plaid Hat Games. Hra je zasazena do fantasy světa, kde hráči přebírají role pirátů, jejichž úkolem je spolupracovat, aby úspěšně splnili sen jejich kapitána. Každá z postav má však své vlastní cíle, které jsou reprezentovány vývojem postavy a jejími schopnostmi. Důležitou součástí hry je webová aplikace, která poskytuje hudební a zvukové efekty, včetně hlasového vyprávění, které hráče provází příběhem a vytváří tak atmosféru hry. \cite{forgotten_waters}


\subsection{Herní systém}
\label{subsec:fw_gameplay}

Z pohledu herní struktury tato hra jednoznačně spadá do kategorie kooperativních her \chapterref{subsec:structure_cooperative}, je však obohacena i o prvky scénářových \chapterref{subsec:structure_scenario} a legacy her \chapterref{subsec:structure_legacy}. Před samotnou hrou si hráči vyberou, pod kterým kapitánem budou sloužit jako posádka, což je v podstatě abstrakce nad pouhým vybráním herní kampaně, která obsahuje předpřipravený scénář, včetně herní mapy a příběhu, kterým budou hráči procházet. Samotná hra pak spočívá v cestování mezi lokacemi a potýkání se s událostmi, které tyto lokace nesou. Kampaň je rozdělena do několika částí, mezi kterými jsou hráči vybídnuti, aby si hru uložili a pokračovali v ní později. Jednotlivé části mohou trvat několik hodin, pokud však hráči chtějí, kampaň jde odehrát během jednoho dlouhého sezení. Herní komponenty se hrou však přímo neznehodnocují, s jedním balením je možné odehrát libovolný počet kampaní.

Lokace se ve hře vybírají pomocí pohybu lodi po mapě tvořené hexagony. Podle scénáře jsou některé hexagony předem určené, jiné jsou prázdné a když na ně hráči stoupnou, umístí místo nich náhodně vylosovanou lokaci. Herní pole je tedy rozděleno dvoudimenzionálně a mění se dynamicky \chapterref{subsec:movement_map_reveal} podle postupu posádky. Pohyb lodi je ovlivněn hodem hráče, který si vybral, že ji bude toto kolo řídit. Podle jeho úspěšnosti se loď posune u určitý počet polí, pohyb je tedy náhodný \chapterref{subsec:movement_roll_movement} s přidanými modifikátory.

V rámci lokace je hráčům představeno několik možností, které si může vybrat buď několik lidí, nebo jsou přístupné pouze prvnímu z nich \chapterref{subsec:actions_limited_actions}. Některé z nich jsou také povinné - musí si je vybrat alespoň jeden hráč. Tyto možnosti jsou reprezentovány jako vytištěný seznam akcí na stránce knihy, která je součástí balení. Výběr akcí probíhá v pořadí iniciativy, která je reprezentovaná stupnicí věhlasu, jde tedy o pořadí podle jisté statistiky \chapterref{subsec:turns_stat_order}. Samotné vyhodnocování probíhá v pořadí od první akce po poslední tak, jak jsou na stránce vytištěny.

Co se týče konce hry, ten je určený předem připraveným scénářem. Každá část končí nějakou událostí, která se vyvolá po dokončení lokace. Po dokončení všech částí hry je dokončen i příběh a následuje příběhový epilog, vyhodnocený podle toho, jak se hráči během hry chovali. Jde tedy o příklad ukončení na základě splnění kooperativního cíle \chapterref{subsec:end_cooperative_goals}. Na závěr hry však ještě probíhá vyhodnocení příběhu jednotlivých postav, které v průběhu hry získávají body souhvězdí, které následně určují, jakým způsobem se příběh postavy vyvine. Jedná se tedy o ukázku konce hry podle vítězného skóre \chapterref{subsec:end_victory_points}.


\subsection{Fyzické komponenty}
\label{subsec:fw_components}

\gameref{Na vlnách neznáma} svůj herní systém podporuje několika fyzickými komponenty, které jsou součástí balení hry. Ty nejdůležitější z nich jsou přiblíženy níže.

\subsubsection*{Herní deska}
\label{subsubsec:fw_comp_board_and_book}

Herní deska je tvořena hexagonálním polem, které reprezentuje moře, po kterém se hráči pohybují. Na tuto prázdnou šablonu se umisťují hexagonální žetony navigace, které představují místa, kterými hráči proplouvají. Když hráči plují přes prázdné pole, náhodně vylosují žeton, který na toto pole umístí, což dodává hře pocit, že v ní opravdu prozkoumávají neznámé moře. Jednotlivé žetony navigace mají své vlastní identifikační číslo, které přes aplikaci odkáže hráče na příslušnou stránku v knize lokací.

Kniha lokací je příručka, která obsahuje všechny lokace, které mohou hráči navštívit. Každá lokace má svou vlastní stránku, na které je popsána, včetně možností, které na ní hráči mají. Na levé straně jsou vždy vytištěné jednotlivé akce, na které si hráči během kola postaví své figurky a dále s těmito akcemi pracují. Na pravé straně je vytištěný tematický obrázek, který hře dodává atmosféru.

\subsubsection*{Karty}
\label{subsubsec:fw_comp_cards}

Většina karet ve hře reprezentuje předměty, které mohou hráči získat a použít. Dělí se na poklady, které hráči získávají různými způsoby během hry, a příběhové karty, které se vztahují k příběhu postav. Karty přinášejí bonusy ke schopnostem, které postavy mohou využít, a také mohou nést složitější bonusy, které jsou ve zkratce popsané na kartě. Jedná se o jednoduchý způsob, jak hráčům umožnit jedinečný vývoj jejich postav, který je zároveň snadno zpřístupněný a přehledný.

\subsubsection*{Postavy a role}
\label{subsubsec:fw_comp_roles}

Jako posádka lodi si hráči před začátkem hry rozdělí role, o jejichž povinnosti se musí postarat. Spolu se svou rolí dostanou i list nebo desku, která slouží jako počítadlo pro různé statistiky, které budou hlídat. Jedná se o následující role:

\begin{itemize}
    \item \textbf{Lodní písař}: Stará se o lodní deník.
    \item \textbf{Kormidelník}: Hlídá věhlas, který určuje iniciativu při výběru akcí.
    \item \textbf{První důstojník}: Hlídá nespokojenost a počet členů posádky.
    \item \textbf{Bocman}: Kontroluje stav trupu - pokud se loď potopí, hra končí.
    \item \textbf{Bednář}: Hlídá lodní zásoby.
    \item \textbf{Dělostřelec}: Obsluhuje lodní děla, která se využívají především v soubojích.
    \item \textbf{Pozorovatel}: Hlídá průběh příběhu.
\end{itemize}

Pomocí takovéhoto rozdělení rolí mezi hráče je zajištěno, že se všichni hráči budou muset podílet na řízení lodi, a zároveň se také zamezí tomu, aby se některé činnosti opakovaly.

Samotné postavy také mají své vlastní deníky. Tyto deníky slouží jako záznamník pro vývoj postavy, který je reprezentován body souhvězdí, které postava získává během hry. Po konci příběhu se podle těchto bodů vyhodnotí, jaký osud postavu potkal. Kromě těchto příběhů a jejich potenciálu na získávání schopností různé úrovně se však postavy moc neliší.

\subsubsection*{Kostky}
\label{subsubsec:fw_comp_dice}

Pro vyhodnocování úspěšnosti akcí se ve hře používají dvanáctistěnné kostky, které se hází vždy, když je třeba určit výsledek zkoušky dovednosti. Když si hráč háže například na sílu, k výsledku hodu kostkou přičte svou vlastní hodnotu síly a případně bonusy, které mohl získat z karet. Výsledek pak porovná s obtížností úkolu, kterou určuje scénář.

Zpestření hodů zajišťují speciální žetony, které upravují výsledek hodu. Jedná se o žetony neštěstí, při kterých si hráč musí hodit znovu a počítat si horší výsledek, a žetony přehození, které naopak umožňují ze dvou výsledků počítat ten lepší. 

\subsubsection*{Figurky}
\label{subsubsec:fw_comp_figures}

\gameref{Na vlnách neznáma} nepoužívá přímo figurky, ale volí levnější variantu - tzv. standees, ilustrace vytištěné na kartonu, které se postaví do plastových stojanů. Tyto standees reprezentují jak postavy v rámci lokace, tak i samotnou loď na herním plánu. Jednoduchost tohoto způsobu umožňuje snadnou reprezentaci postav a zároveň snižuje náklady na výrobu hry.


\subsection{Podpůrné aplikace}
\label{subsec:fw_apps}

Hlavní funkcionalitu, kterou aplikace přináší, je průvodce příběhem. Při vstupu si hráči zvolí kampaň, kterou chtějí hrát, načež jim aplikace přehraje úvodní dialog a pokud je to potřeba, krok po kroku jim ukáže, jak si mají herní desku a ostatní komponenty připravit. Dále aplikace slouží jako knihovna informací. Každý záznam, který hráčům ukazuje, má svůj třímístný kód, pomocí jehož hráči mohou na záznamy přistupovat. Tento přístup poskytuje vývojářům velkou flexibilitu v tom, odkud mohou hráči kódy získávat. Může jít o pouhé uvedení lokace (jak již bylo výše zmíněno, každá lokace má svůj odpovídající kód), krátkou anekdotu po získání nového předmětu nebo i o příběhový výsledek nějakého rozhodnutí. Jako vedlejší funkci aplikace také přináší hudební a zvukové efekty, které hru doplňují a vytváří kýženou atmosféru. \cite{fw_crossroads_app}

Nově také vznikla oficiální webová aplikace, která umožňuje hru hrát plně online. Na stránkách je možné si založit herní místnost, do které se mohou připojit ostatní hráči. Během hry jsou pak digitalizovány všechny komponenty, které jsou potřeba k hraní, včetně herní desky, knihy lokací, karet a počítadel pro jednotlivé role. \cite{fw_remote_app}

\subsection{Herní pravidla}
\label{subsec:fw_rules}

Pravidla hry \gameref{Na vlnách neznáma} jsou přiložena jako součást balení, jsou však přístupná i v digitální formě na stránkách vydavatele. Hned ze začátku jsou hráči seznámeni s důležitostí webové aplikace a je jim předložen odkaz, přes který se na ni dostanou. Následně je ve zkratce uveden a shrnut veškerý herní materiál, který je v balení obsažen, a složitější komponenty jsou vyobrazeny nákresem s popisky klíčových částí.

Hlavní část pravidel je věnována popisu postupu na přípravu hry a průběhu jednotlivých kroků. Příprava je rozepsána do očíslovaných kroků a doplněna o ilustrace, které hráčům usnadňují orientaci. Následně je hráčům vysvětleno, jak je rozděleno kolo a jak fungují akce v něm. Dále jsou také přiblíženy jednotlivé postavy, jejich dovednosti a zkoušky, které se k nim vztahují. Na závěr jsou hráči seznámeni s tím, jak může hra skončit, ať už pozitivně či negativně, a jak se hra ukládá, pokud se rozhodnou ji hrát v oddělených sezeních. Jako zajímavý bonus na konci pravidel je přiložen generátor pirátských jmen, který hráčům pomocí dvou kostek umožní vygenerovat si jméno pro svou postavu.


%%% Dungeons & Dragons %%%

\section{Dungeons \& Dragons}
\label{sec:dungeons_and_dragons}

Obrovský vliv, který měla revoluční RPG hra \glsref{dnd} na vývoj stolních her, už byl zmíněn výše. Tato hra je však také zdrojem inspirace pro mnoho deskových her, které se snaží přenést některé z jejích prvků do stolního prostředí, a rozšířit tím svou potenciální hráčskou základnu. Tato sekce se zaměřuje na analýzu prvků, které byly převzaty z \dnd{} do jiných her. Oproti jiným hrám, které jsou v této práci rozebrané, je však \dnd{} mnohem komplexnější a množství jeho herních mechanik je také mnohem větší, proto se tato sekce bude držet spíše herních systémů a mechanik, které jsou popsány ve výše zmíněné kapitole \textit{\nameref{chap:game_design}}. \cite{dnd_beyond_2023}


\subsection{Herní systém}
\label{subsec:dnd_gameplay}

Prvně je třeba zmínit, že \dnd{} se nedrží klasické symetrické struktury, kterou většina stolních her využívá. Hra je založena na asymetrickém systému, kde jeden z hráčů, označovaný jako \textit{Dungeon Master} (DM) nebo vypravěč, má na starosti vytváření a řízení celého světa, včetně všech postav, které v něm žijí. Ostatní hráči, v tomto světě následně hrají své vlastní postavy a dělají za ně rozhodnutí, stěžejní částí herního zážitku je tedy \textit{roleplay}. Tento systém je základem pro všechny ostatní mechaniky, které jsou v této hře používány. Z tohoto popisu se může zdát, že hráči a DM soupeří o jakousi výhru, ve skutečnosti však spolupracují na vytváření příběhu, který je pro všechny zúčastněné zábavný. Tato skutečnost z \dnd{} dělá nejen pouhou stolní hru, ale prostředek pro kooperativní vyprávění příběhů.

Svou strukturou je \dnd{} ukázkovým příkladem legacy hry \chapterref{subsec:structure_legacy}, neboť běžná kampaň může trvat až roky a z pohledu ničení herních komponent se tato charakteristika odvíjí od konkrétní kampaně a také kreativity DM a ostatních hráčů. Je zřejmý také strukturální vliv scénáře \chapterref{subsec:structure_scenario} -- pro hráče je přístupná spousta předpřipravených scénářů a kampaní, kterými se DM může řídit, ať už jde o oficiální příběh od vývojářů nebo o neoficiální práci některého z mnoha fanoušků, tvořící pro hru další obsah. Hráči se však příběhu nemusí držet, pokud nechtějí, neboť volnost herních mechanismů dává velkou flexibilitu v improvizaci a nečekaných zvratech, které často nečeká ani sám vypravěč. Ve své podstatě se také jedná o kompetitivní hru \chapterref{subsec:structure_competitive}, i když během roleplaye a herních zvratů se může jednoduše stát, že se některé z postav obrátí proti sobě a je na hráčích, jak tyto konflikty vyřešit. Tyto konflikty by však měly zůstat pouze v mezích herního světa a nevnášet neshody mezi samotné hráče.

I přes to, že příběhové sekce a části více zaměřené na roleplay nemají téměř žádnou strukturu, během souboje je struktura relativně pevná, samozřejmě s výjimkami. Před soubojem si všechny postavy i nepřátelé hodí kostkou na iniciativu, k výsledku si přičítají své vlastní bonusy a výsledné číslo určí pořadí kola, jde tedy o určení pořadí podle statistiky \chapterref{subsec:turns_stat_order}. Ve svém kole má poté hráč přesně určená pravidla na to, jaké akce může provést. Obvykle se jedná o pohyb, akci (útok zbraní nebo seslání kouzla) a bonusovou akci (interakci s předmětem, vytasení zbraně nebo speciální kouzla), přičemž hráč si může vybrat v jakém pořadí tyto akce provede, pokud vůbec. 

Co se týče pohybu, herní pole je obvykle rozděleno na hexagony, i když přesný vzhled mapy záleží čistě na DM. Po herním poli se hráči mohou pohybovat ve svém tahu, vzdálenost, kterou urazí je zapsána v jejich deníku postavy. Mají také možnost sprintovat, což jejich rychlost zdvojnásobí, za cenu spálení akce. Pravidlem také bývá, že akce by měly dávat z prostorového pohledu smysl, například že by se postavy neměly v boji přeskakovat ani střílet po nepřátelích, když jim stojí v cestě jejich spojenci. Opět ale záleží na rozhodnutí hráčů.

Samotné souboje jsou řešeny pomocí kostek. Každá zbraň má svůj vlastní druh kostky, kterou se hází, když se s ní útočí. Hráč si nejprve hodí kostkou, zda rána vůbec zasáhla, výsledek se porovná s obranou cíle, a pokud se útok povedl, cíl utrpí poškození. Kromě toho se také používají kostky k určení výsledku kouzla, které se sesílá. Výsledky těchto hodů se mohou dále upravovat pomocí bonusů, které postavy získávají z předmětů, nebo z jejich vlastních dovedností.

Konec hry je určený předem připraveným scénářem, který opět může být buď oficiální, nebo vytvořený DM. Hra končí, když hráči splní cíl, který je jim předložen, nebo když všichni hráči zemřou, i v takovém případě však je možnost vrátit se do hry jako jiná postava, pokud s tím ostatní hráči souhlasí. Výsledek hry je tedy určený splněním kooperativního cíle \chapterref{subsec:end_cooperative_goals}.

\subsection{Fyzické komponenty}
\label{subsec:dnd_components}

Flexibilita a otevřenost \dnd{} s sebou nese i osvobození od fyzických komponent, které by byly nutné k hraní. Hra jako taková ve své podstatě potřebuje jen tři komponenty -- pravidla, kostky a papír. I tak však existuje spousta dalších komponent, které mohou hru ozvláštnit.

\subsubsection*{Pravidla}
\label{subsubsec:dnd_comp_rules}

Pravidla \dnd{} jsou popsána v různých knihách, které se věnují jiným aspektům hry. Hlavními knihami jsou \textit{Player's Handbook}, která obsahuje pravidla pro hráče, \textit{Dungeon Master's Guide}, která pomáhá DM s vytvářením a řízením světa a \textit{Monster Manual}, která obsahuje předpřipravené příšery, se kterými se mohou hráči potýkat. Vývojáři však vydávají i další knihy, které mohou rozšiřovat herní svět, vyprávět nové příběhy, nebo přidávat nová pravidla, která mohou být využita v kampani.

\dnd{} má však i velmi aktivní komunitu, která vytváří vlastní obsah, a tím ještě více rozšiřuje možnosti hry. Opět může jít o nová pravidla, nové příšery, nebo i nové kampaně, které mohou být využity v rámci hry. Tento obsah může být dostupný zdarma nebo za poplatek, ať už v tištěné podobě, nebo v digitální formě.

\subsubsection*{Kostky}
\label{subsubsec:dnd_comp_dice}

\dnd{} je známé svými mnoha druhy kostek, které se během hry používají. Tou nejznámější, dalo by se říct až ikonickou, je dvacetistěnná kostka, která se během posledních let stala symbolem celé hry. Používá se nejčastěji, od určení iniciativy, přes útoky až po kontroly dovedností. Dalšími kostkami, které se používají, jsou čtyřstěnná, šestistěnná, osmistěnná, desetistěnná a dvanáctistěnná kostka, nejčastěji jako určení poškození od určité zbraně nebo kouzla. Každá sada kostek navíc obsahuje dvě desetistěnné kostky, z nichž jedna určuje desítky a druhá jednotky, což umožňuje hráčům házet procenta.

\subsubsection*{Deníky postav}
\label{subsubsec:dnd_comp_sheets}

I když je každá postava v družině jedinečná, všechny mají svůj deník, který obsahuje veškeré informace, které hráči potřebují k hraní, včetně statistik, dovedností, kouzel, předmětů a záznamů o postupu v kampani. V páté edici hry je list složen z několika částí. Všichni hráči mají na listu zapsané vlastnosti své postavy, klasicky se jedná o Sílu, Obratnost, Odolnost, Inteligenci, Moudrost a Charisma. Dále mají zapsané dovednosti, které se k těmto vlastnostem vážou, a které mohou během hry zlepšovat. Na listu je také místo pro záznamy o zdraví, obraně, iniciativě a jiných statistikách, které se během hry mění, a pro záznamy o zbraních, vybavení a zásobách, které s sebou postava nosí.

Každý z hráčů si při tvorbě postavy volí rasu, povolání a zázemí, což jejich postavu dále ovlivňuje. I deníky postav na tyto změny reagují, hráči často využívají předpřipravené listy, které jsou personalizované pro jejich povolání, protože v něm jsou změny nejviditelnější. Mnoho hráčů si však vytváří své vlastní deníky tak, aby lépe vyhovovaly jejich potřebám.

\subsection{Podpůrné aplikace}
\label{subsec:dnd_apps}

Komplexita herního systému \dnd{} přímo vybízí k využití podpůrných aplikací, které přichází jak v oficiální, tak i fanouškovské formě. Hlavní oficiální aplikací je \textit{DnD Beyond}, která je zdarma, ale obsahuje i možnost dokoupení placeného obsahu, a slouží jako průvodce pravidly, zdroj pro vytváření postav a knihovna pro všechny dostupné příběhy a pravidla. Fanoušky vytvořené aplikace kopírují tyto funkcionality, ale zaměřují se také na jiné aspekty hry, od generátorů postav, přes knihovny příšer až po aplikace, které umožňují zaznamenávat celé kampaně a herní světy.

\subsection{Herní pravidla}
\label{subsec:dnd_rules}

Jak již bylo zmíněno výše, pravidla \dnd{} jsou rozdělena do několika knih, které se věnují různým aspektům hry. Hlavní pravidla, která se týkají hráčů, jsou obsažena v \textit{Player's Handbook}, která obsahuje všechny informace, které hráči potřebují k hraní. Nejprve krok po kroku popisuje, jak vytvořit postavu a následně se ponořuje do detailů ohledně rasy, povolání, zázemí a jiných rysů správné postavy. Dále jsou hráčům představeny základní zbraně, vybavení, jiné předměty a také kouzla, která jsou pak využita v následující sekci o pravidlech souboje.

\textit{Dungeon Master's Guide} je druhá kniha, která je určena především DM, a která se věnuje vytváření a řízení světa, ve kterém se hra odehrává. Jejím hlavním cílem je doprovázet vypravěče při přípravě hry, ať už se jedná o tvorbu světa, dobrodružství a příběhů v nich nebo nehráčských postav, které se v nich vyskytují. Přináší také cenné generátory náhodných událostí nebo předmětů pokladu, které mohou být využity v průběhu hry, a také rady, jak řešit problémy, které se mohou během hraní objevit.

Poslední z trojice základních pravidel je \textit{Monster Manual}, který obsahuje předpřipravené příšery, se kterými se mohou hráči potýkat. Každá příšera má svou vlastní sekci, ve které je popsán jak její příběh a místo v herním světě, tak její statistiky a schopnosti. Kniha je určena především pro DM, ale může být využita i hráči, kteří se o příšerách chtějí dozvědět více.

Během let však vzniklo mnoho dalších oficiálních placených knih, které rozšiřují možnosti hry. Může jít o nová pravidla (\textit{Xanathar's Guide to Everything} nebo \textit{Tasha's Cauldron of Everything}), nové příšery (\textit{Mordenkainen's Tome of Foes}), nebo i nové kampaně (\textit{Icewind Dale: Rime of the Frostmaiden}), které mohou být využity v rámci hry. Další rozšíření však vznikají i díky fanouškům, kteří vytvářejí vlastní obsah, jež může být dostupný zdarma nebo za poplatek, ať už v tištěné podobě, nebo v digitální formě (\textit{Questonomicon}). Díky tomu se hráči nemusí bát, že jim dojde obsah, ze kterého by mohli pro své hry čerpat.


%%% Gloomhaven %%%

\section{Gloomhaven}
\label{sec:gloomhaven}

Poslední hrou, která bude v této kapitole rozebrána, je \glsref{gloomhaven}, kterou v roce 2017 vydalo vydavatelství \textit{Cephalofair Games}. Hra je známá svým inovativním herním systémem, který se snaží přenést prvky \dnd{} do stolního prostředí, a zároveň omezit některé z jeho nevýhod. Tato sekce se zaměřuje na analýzu této hry, která z výše zmíněných poslouží jako nejsilnější inspirace pro vývoj modelové hry. \cite{gloomhaven}

\subsection{Herní systém}
\label{subsec:gh_gameplay}

Struktura \gameref{Gloomhavenu} je oproti \dnd{} více symetrická a především pevněji organizovaná a značně zjednodušená. Hra je založena na kooperativním systému \chapterref{subsec:structure_cooperative}, kde hráči spolupracují na splnění cílů, které jsou jim předloženy scénářem \chapterref{subsec:structure_scenario}, a zároveň se snaží porazit nepřátele, kteří se jim v jeho vykonání brání. Jedním z nejzajímavějších strukturálních rozhodnutí vývojářů je však míra implementace prvků legacy systému \chapterref{subsec:structure_legacy}, což herní zážitek posouvá na novou úroveň. Během hry jsou hráči vybízeni k otevírání obálek, kreslení na karty, jejich trhání nebo lepení nálepek na herní mapu. Hra se tak stává unikátní a personalizovaná pro každou skupinu hráčů, ale bohužel také ztrácí možnost být znovu hraná, což je jedna z největších nevýhod legacy her.

Herní mapa je rozdělena na jednotlivé lokace, které se hráčům postupně odemykají v reakci na rozhodnutí, které jejich postavy v rámci příběhu udělaly \chapterref{subsec:movement_map_reveal}. Každá lokace má svůj vlastní scénář, který určuje, co se v ní bude dít, a jaké cíle hráči musí splnit, aby mohli postoupit dál. Interní mapa samotné lokace je hráčům zobrazena předem, aby věděli jak ji poskládat, kde se která postava může pohybovat a kde se mohou nacházet nepřátelé. Po herním poli se hráči přesouvají pomocí karet, které jim určují, jak daleko se mohou pohnout, a jaké akce mohou v rámci svého kola provést. Neboť karty nemusí akci pohybu obsahovat vždy, často se stane, že hráči nemají možnost se ve svém kole vůbec pohnout, takže hráči se musí dobře rozmyslet, kam se postaví, a jaké akce v rámci svého kola provedou.

Na začátku každého kola si hráči vyberou dvě karty akcí, které chtějí zahrát. Tyto karty mají na sobě napsané, jaké akce mohou hráči v rámci svého kola provést, a také číslo, které určuje iniciativu, podle které se určuje pořadí hráčů v kole. I nepřátelé mají svoji kartu, která se na začátku kola vylosuje a která také obsahuje iniciativu. Během svého tahu hráči mohou zvolit, v jakém pořadí vybrané akce zahrají, ale zahrát je musí obě. Po zahrání karty se karta buď odloží do odhazovacího balíčku, nebo se úplně spálí. Když hráči nezůstanou v ruce žádné karty, jeho postava je vyčerpaná a po zbytek boje už hrát nebude. Hráči tak musí dobře zvážit, jaké karty zahrát, aby měli co největší efekt, ale zároveň aby se jim nepřihodilo, že by zůstali bez možnosti akce.

Každá lokace má svůj cíl, kterého se hráči snaží dosáhnout \chapterref{subsec:end_cooperative_goals}. Může jít o poražení všech nepřátel, nalezení nějakého předmětu, nebo i o útěk z lokace. Když cíl splní, mohou se jim odemknout nové lokace nebo mohou získat odměnu, která může být v podobě zkušeností, peněz, nebo i nových předmětů, které mohou být využity v rámci hry. Když cíl nesplní, mohou se pokusit lokaci zahrát znovu, nebo mohou pokračovat v kampani, a zkusit štěstí v jiné lokaci. Celá hra má však taky svůj příběh, kterým hráči postupují, a který se mění v závislosti na rozhodnutí, které hráči v rámci hry udělali. Hra také obsahuje několik vedlejších úkolů, které mohou být v rámci hry splněny, a které mohou hráčům přinést různé výhody. Hra tedy končí až po dohrání celé kampaně, která může trvat i několik měsíců či let.

\subsection{Fyzické komponenty}
\label{subsec:gh_components}

Součástí hry \gameref{Gloomhaven} je mnoho komponent, které jsou potřeba k hraní, o čemž vypovídá i samotná váha krabice, která je přes 10 kilogramů. V následujícím výčtu jsou popsány ty nejdůležitější z nich.

\subsubsection*{Knihy}
\label{subsubsec:gh_comp_books}

V balení hry mohou hráči najít více knih, každou se svým unikátním účelem. Hlavní knihou je kniha scénářů, která obsahuje všechny lokace, které se v rámci hry mohou odehrát. Každá lokace má svou vlastní stránku, na které je popsán její cíl, případná speciální pravidla a také odměny, které mohou hráči získat, pokud ji splní. Kniha scénářů je praktická pro přípravu lokací podle plánku, který je v ní uveden, bohužel však tento plánek odhaluje všechny místnosti, včetně těch, které hráči zatím neprozkoumali. Tento nedostatek je implicitní pro fyzické předání těchto informací, jak však ukazují později rozebrané podpůrné aplikace \chapterref{subsec:gh_apps}, existují způsoby, jak tento problém vyřešit.

Herní systém \gameref{Gloomhavenu} je sice oproti komplexnějším hrám jako je \dnd{} značně zjednodušený, i tak ale obsahuje mnoho pravidel, které je třeba dodržovat. Proto je hráčům k dispozici kniha pravidel, která všechna pravidla obsahuje, a tak ji hráči mohou kdykoli v běhu hry konzultovat. Pravidla obsahují popis všech možných situací, od detailů souboje přes detailní popis přípravy lokace až po příběhové mechaniky.

Kromě těchto dvou často využívaných materiálů jsou v balení i další deníky a jiné listy. Pro záznamy ohledně města slouží deník města, příběhové zvraty jsou skryty v zalepených obálkách a pro vyluštění speciálních šifer mohou hráči použít šifrovací kartu.

\subsubsection*{Mapa}
\label{subsubsec:gh_comp_map}

Hráči se během hry pohybují po dvou mapách -- světové a lokální. Světová mapa je rozdělena na jednotlivé lokace, které se hráči postupně odemykají podle postupu příběhem. Tato mapa je natištěna na velkém plátně, na které se poté nalepují nálepky, které odemykají nové lokace a také se na ni značí jiné světové statistiky.

Samotné mapy jednotlivých lokací jsou tvořeny spojováním místností tvořených hexagony, jejichž rozložení hráčům ukazuje kniha scénářů. Místností je mnoho, proto je jejich vyhledávání mezi ostatním materiálem urychleno pomocí jedinečných identifikátorů, které jsou vždy vytištěné na dílech. Po rozvržení místností se herní plocha může ozvláštnit více pomocí samostatných hexagonových tokenů, které se pokládají na políčka, jež reprezentují překážky, pasti, dveře, nebo jiné herní prvky.

\subsubsection*{Figurky}
\label{subsubsec:gh_comp_figures}

V \gameref{Gloomhavenu} se hráči setkávají se třemi základními typy figur, které se v rámci hry využívají. První z nich jsou figurky postav, které hráči ovládají. Každá postava má svou vlastní unikátní 3D figurku, která je vylitá z pryskyřice a uchovaná v krabičce, aby se nepoškodila. Od výroby nejsou nabarveny, zručnější hráči však často využívají této příležitosti na jejich vlastní personalizaci a namalují si je podle svých představ.

I nepřátelé mají svou fyzickou reprezentaci. Z toho důvodu, že jich je mnohem více než hráčských postav, vývojáři opět využili možnosti \textit{standees}, které jsou vytištěné a postavené na plastových stojanech. Tento způsob reprezentace nepřátel je sice méně atraktivní než figurky, ale zase umožňuje snadnější manipulaci s velkým množstvím nepřátel, které se v rámci hry mohou objevit. Stojany jsou také využity pro další herní mechanismus, odlišování obtížnosti nepřátel: základní, elitní a bossové. Neboť se na herním poli může vyskytovat více nepřátel stejného typu, jsou součástí balení i plastové kroužky s čísly, které je možné na stojany nasadit, aby se hráči mohli snadno orientovat v tom, který nepřítel je který.

Poslední a nejjednodušší způsob reprezentace mají summoni, které hráči nebo nepřátelé mohou vyvolat pomocí speciálních karet, a kteří jim pak slouží jako poskoci. Na herním poli se pak značí pomocí kruhového tokenu, na kterém je vytištěná stejná značka, jako na kartě, která ho vyvolala. Pro vývojáře je pak velmi jednoduché přidat do hry spoustu různých unikátních summonů bez nutnosti přípravy jejich fyzické reprezentace a hráčům tento způsob zase ulehčuje orientaci na herním poli, které může být během hry nepřehledné.

\subsubsection*{Karty}
\label{subsubsec:gh_comp_cards}

Druhů karet je v balení více, ale tím nejdůležitějším jsou karty akcí. Hráči si na začátku každého kola vyberou dvě karty, které určují, jaké akce mohou v rámci svého kola provést. Karty jsou rozděleny na dvě části, přičemž každá má na sobě napsaný detail akce neboli jednotlivé kroky, ze kterých akce sestává. Každá polovina má také možnost být nahrazena základním útokem nebo pohybem. Ve středu karty je zapsáno iniciativní číslo, které později určuje pořadí hráčů. Hráči se tedy musí rozhodnout nejen podle samotných akcí, ale také podle iniciativy -- pokud chtějí jet v kole mezi prvními, budou muset upřednostňovat karty s nižším číslem.

Dalším druhem karet jsou modifikátory pro útoky. Tyto karty plní klasickou roli kostek, neboť dodávají náhodnost útokům a k tomu umožňují větší míru úprav těchto modifikátorů. Na běžné kostce jsou vývojáři limitováni počtem stran, ale s balíčkem karet je jednoduché přidávat a odebírat nové modifikátory. V nějakých lokacích se například používají speciální karty, které si hráči vloží do svého balíčku, a které platí jen pro tuto lokaci a tím ji dělají zajímavější.

Karty se také používají jako reprezentace předmětů. Hráči mohou vidět, o jaký předmět se jedná, kolik zlatých stojí, na jakou část těla patří a jaké má vlastnosti. Tyto karty se pak vykládají na stůl, kde se kontroluje, zda byl předmět použit, podle jejího natočení.

Příběhové karty jsou oproti ostatním typům používány zřídka, ale nejsou o nic méně důležité. Při každém vstupu do města a i na jiných předem určených místech je hráčům představena nějaká malá část příběhu spolu s možností, jak na tuto situaci reagovat. K tomuto účelu slouží právě tyto karty, které mají na jedné straně příběh a možnosti a na druhé straně výsledky těchto rozhodnutí. Hráči si tedy vyberou, jak chtějí reagovat, a podle toho si otočí kartu a zjistí, jaké důsledky jejich rozhodnutí mělo.

\subsubsection*{Tokeny}
\label{subsubsec:gh_comp_tokens}

\gameref{Gloomhaven} obsahuje spoustu efektů jako krvácení, jed nebo oheň, které je třeba monitorovat. K tomuto účelu slouží tokeny, které si hráči umístí na kartu jejich postavy, která je těmito efekty postižena. Tokeny efektů jsou využity také pro nepřátele, kteří mají svou speciální kartu s místem pro každého jednotlivého nepřítele, kde se tyto efekty dají relativně přehledně zobrazit. Stejným způsobem se na tato místa také zaznamenává, kolik zranění už jednotliví nepřátelé dostali, pomocí speciálních tokenů zranění. Tokeny jsou v balení hry vyrobeny z kartonu, a jsou také v balení několikrát, aby se hráči nemuseli bát, že by jim některý z nich chyběl.

\subsubsection*{Komponenty pro postavy}
\label{subsubsec:gh_comp_characters}

Klíčovou komponentou pro herní postavu je její deník. Ten umožňuje zaznamenání jména postavy, její úrovně a s tím souvisejících bodů zkušeností, dále jejích peněz a předmětů, které u sebe má. Pravá strana listu pak má sekci pro zápis schopností, které postava získala v průběhu hry a jejich vylepšení, což hráčům umožňuje jednoduše kontrolovat, jak se jejich postava vyvíjí. Podobný účel pak má deník skupiny, kam si hráči zapisují postup kampaně, lokace, které již prošli nebo odhalili, jejich reputace v hlavním městě a také úspěchů, kterých dosáhli a které slouží jako způsob kontroly, kde v příběhu se momentálně vyskytují.

Dále je každá postava opatřena svou vlastní referenční kartou, která hráče informuje o jejím zdraví a jiných statistikách a na druhé straně má vytištěný krátký příběh, který postavu charakterizuje. Pro souboje se také hodí počítadla životů a zkušeností, provedené jako číselník, se kterým hráči točí. Spolu s kartami akcí a modifikátorů se i tyto komponenty uchovávají v krabičce, která je součástí balení hry.

\subsection{Podpůrné aplikace}
\label{subsec:gh_apps}

Vývojáři sice prozatím žádnou oficiální aplikaci nevydali, fanoušci však jako vždy neváhali tuto mezeru zaplnit. Nejznámější aplikací je \textit{Gloomhaven Secretariat}, která umožňuje udržovat téměř všechny informace, které se netýkají přímo postav a herního pole. Hráčům kontroluje iniciativu, jednotlivé nepřátele, jejich životy a efekty, a také zajišťuje jejich karty akcí a modifikátorů, které se automaticky náhodně losují a šetří tak hráčům zbytečnou práci. Aplikace také umožňuje hráčům zaznamenávat svůj postup v kampani, a také další herní mechaniky, které byly pro stručnost této sekce vynechány. Aplikace je opensource a je dostupná zdarma přes webovou stránku.

Dále je tu mobilní aplikace \textit{Gloomhaven Scenario Viewer} dostupná pro Android a iOS, která obsahuje všechny herní lokace a umožňuje hráčům jednoduše si zobrazit, jak mají před startem dílky uspořádat bez toho, aby tyto informace museli hledat v knize scénářů. Výhodou také je, že místnosti mohou zobrazovat postupně, jak je odemykají, takže se nemůže stát, že by hned ze startu věděli, co se skrývá za kterými zavřenými dveřmi. Tato aplikace je také zdarma.

Poslední aplikací, která je v komunitě hráčů oblíbená, je \textit{Gloomhaven Storyline}. Její primární využití je pro zaznamenávání příběhu a postupu v kampani, ale kromě toho umožňuje i zobrazování lokací, podobně jako předchozí aplikace, s přidanou možností přímé úpravy jednotlivých hexagonů, podle momentálního stavu herního pole na stole. Opět se jedná o webovou aplikaci, ke které mají hráči přístup zdarma.

\subsection{Herní pravidla}
\label{subsec:gh_rules}

Jak již bylo zmíněno výše, pravidla hry jsou zaznamenána v knize pravidel, která je součástí balení hry. Nejprve hráčům představí základní komponenty, které budou používat, dále detailně popíší přípravu scénáře a poté se ponoří do pravidel souboje, která zabírají největší část knihy z důvodu jejich komplexnosti. Vše je popsáno v detailu, ale zároveň stručně, a tak hráči nemusí mít obavy, že by něco nevěděli, nebo že by hru hráli špatně. Hráči jsou dále seznámeni s konkrétní implementací kampaně a příběhu a na závěr příručka obsahuje i několik vedlejších dobrovolných pravidel, která mohou hru ozvláštnit a upravit, jako například pravidla pro sníženou náhodnost, permanentní smrt postav nebo odehrání náhodně vytvořeného scénáře, který si hráči mohou vytvořit sami.

Kromě samotné knihy vývojáři zveřejnili i oficiální FAQ, které obsahuje odpovědi na nejčastější dotazy, které se v rámci hry objevují. Tento seznam je přiložen jako součást digitální verze pravidel, ale ve fyzické knize pravidel chybí.


%%% Comparison %%%

\section{Porovnání}
\label{sec:comparison}

V této kapitole byly rozebrány tři hry, které slouží jako inspirace pro vývoj modelové hry -- \glsref{forgotten_waters}, \glsref{dnd} a \glsref{gloomhaven}. V této sekci budou tyto hry porovnány, a budou zde také popsány jejich silné a slabé stránky, které mohou být využity při vývoji modelové hry.

\subsection{Herní systémy}
\label{subsec:comparison_gameplay}

Ve všech herních systémech je zřejmá velká převaha kooperačních prvků s velkou dávkou příběhového obsahu. Příběh je dán předem a hráči se snaží splnit cíle, které jsou jim předloženy, a zároveň se snaží porazit nepřátele, kteří se jim v tom snaží zabránit. Legacy systém \gameref{Gloomhavenu} se však zdá až moc omezující, takže ho v rámci modelové hry realizovat nebudeme.

Z pohledu mapy všechny tři hry využívají hexagonové pole, což je něco co určitě chceme převzít i do modelové hry. Flexibilita pohybu do šesti směrů, kterou hexagony poskytují, je velmi důležitá pro vytvoření zajímavého herního pole, proto se bude jednat o klíčový prvek, který bude využit. Systém postupného odemykání lokací s postupujícím příběhem je taky velmi příjemný, pro vývoj i pro hráče, nejspíš proto je používá jak \gameref{Gloomhaven}, tak i \gameref{Na vlnách neznáma}, které aplikují pevnou strukturu světa.

Co se týče kola a akcí, systém iniciativy, který implementuje \gameref{Gloomhaven} i \dnd{}, je skvělý způsob, jak zajistit, aby bylo pořadí náhodné, ale zároveň aby hráči mohli ovlivnit šance lepšího nebo horšího výsledku. Karty akcí, které hráči využívají v \gameref{Gloomhavenu}, jsou také velmi zajímavým způsobem, jak omezit možnosti hráčů, ale zároveň jim dát možnost volby. Tento systém je také velmi vhodný pro vývoj modelové hry, protože je to jednoduchý způsob jak implementovat omezující ale zároveň zajímavé možnosti, ze kterých si hráči mohou volit. Z \gameref{Gloomhavenu} se také dá převzít systém efektů, které se dají snadno monitorovat pomocí tokenů.

Nepřátelé jsou v každé hře zpracováni jinak. \gameref{Na vlnách neznáma} je v podstatě nemá, hráči hrají proti hře samotné, u \dnd{} jsou řízeni DM, který je zodpovědný za všechna jejich rozhodnutí. \gameref{Gloomhaven} nechává řízení nepřátel na hráčích, kteří si je musí sami ovládat. Z těchto systémů není pro modelovou hru ideální ani jeden, proto bychom se chtěli co nejvíce přiblížit systému, který používá \dnd{}, a to tím, že místo DM by byl řízení nepřátel přeneseno na aplikaci, která by hráčům poskytovala všechny potřebné informace a oni by je museli jen vykonat na herní desce.

Unikátnost hráčských postav je důležitá pro všechny tři hry, a také pro modelovou hru. \gameref{Na vlnách neznáma} hráče odlišuje podle příběhu piráta a figurky kterou si zvolí, \gameref{Gloomhaven} má unikátní postavy, které se od sebe liší vzhledem, schopnostmi a také příběhem. V \dnd{} si hráči mohou vytvořit postavu podle svých představ, ale jsou omezeni rasou a povoláním. Pro účely modelové hry by bylo zajímavé použít systém rasy a povolání, proto se nějakou podobnou mechaniku pokusíme implementovat.

Konec hry je u všech tří her určen příběhem, s různou úrovní otevřenosti. Je však zajímavé, že \gameref{Na vlnách neznáma} je jediná hra, která nechá hráče úplně prohrát, pokud v nějakém aspektu hry nebudou úspěšní. Je to také hra, která má nejkratší herní dobu, což může s možností prohry souviset, neboť když hráči hrají hru klidně i několik let, mohou být zklamáni, když by se jim něco takového stalo. I v modelové hře bude přínosné implementovat jiné způsoby, jak hráče potrestat za nepovedený výkon, ale nikdy je nenechat úplně prohrát.

\subsection{Fyzické komponenty}
\label{subsec:comparison_components}

Počet komponentů se liší u všech tří her, všechny však mají společnou především mapu tvořenou hexagony, jak již bylo zmíněno výše, figurky nebo jiný způsob reprezentace postav na herním poli, nějakou formu náhodnosti a také akce hráčů, které musí být přehledné a snadno manipulovatelné.

Figurky \gameref{Gloomhavenu} jsou sice velkým lákadlem hry, ale pro modelovou hru by byly příliš drahé a jejich výroba by byla příliš náročná. Standees, které jsou v balení \gameref{Na vlnách neznáma}, jsou sice méně atraktivní, ale zároveň jsou mnohem snadněji manipulovatelné, a také mnohem levnější. Pro modelovou hru by byly ideální, protože by se daly snadno vytvořit a vyrábět.

Klasickým generátorem náhodnosti jsou kostky, které se využívají v \dnd{} a \gameref{Na vlnách neznáma}. Karty, které se využívají v \gameref{Gloomhavenu}, jsou sice flexibilnější, ale ne tolik přehledné a mít spoustu různých typů karet může být pro hráče matoucí. Proto se modelová hra bude držet kostky, která je vhodnější.

U akcí je důležité, aby byly různorodé, ale aby hráči nebyli přehlcení počtem informací, které musí sledovat. Karty akcí, které se využívají v \gameref{Gloomhavenu}, jsou v tomto ohledu ideální, protože jsou přehledné a zároveň flexibilní. Kromě toho, že se dají snadno vytvořit a vyrábět, také umožňují hráčům snadno ovládat jejich akce, a zároveň jim dávají možnost volby.

\subsection{Podpůrné aplikace}
\label{subsec:comparison_apps}

Podpůrné aplikace jsou důležité pro všechny tři hry, protože umožňují hráčům snadno monitorovat všechny informace, které by jim jinak zbytečně zabíraly čas. Je zřejmé, že když vývojáři nevytvoří oficiální aplikaci, postarají se o to fanoušci, a tedy je o tyto aplikace zájem. 

Hlavní zaměření aplikací spočívá v organizaci, monitorování administrativy jako je průběh příběhu a postup v kampani, a také zaznamenání zdraví a efektů všech entit v souboji. Většina aplikací je zdarma, vývojáři si peníze účtují za samotné fyzické vydání hry, takže není třeba aplikace zpoplatňovat. Tyto kvality by měla odrážet i modelová hra, neboť přispějí k nejlepšímu možnému zážitku hráčů.

\subsection{Herní pravidla}
\label{subsec:comparison_rules}

U všech tří her jsou pravidla velmi důležitá pro vzdělání hráčů a pro zajištění správného průběhu hry. Všechny hry se snaží pravidla předat jednoduchou a věcnou formou, ale zároveň lidsky a s příběhem, což je velmi důležité pro vytvoření atmosféry hry. Je možné se setkat s fyzickou příručkou, ale i s digitální verzí, která je většinou zdarma ke stažení, přičemž v ní je přípustnější rozepsat se trochu více, protože hráči mají v těchto materiálech možnost vyhledávat.

Pravidla jsou strukturována logicky a přehledně, s poutavým úvodem a příběhem, který hráče zaujme a zároveň je připraví na to, co je čeká. Na konec je možné dodat jakýkoli bonusový materiál, který hra nabízí, jako například FAQ nebo další volitelná pravidla, která mohou hru osvěžit.

\subsection{Zhodnocení analýzy}
\label{subsec:comparison_conclusion}

V předchozích bodech byly do hlubších detailů rozebrány jednotlivé mechaniky analyzovaných her, pro přehlednost jsou však tyto aspekty shrnuty v následující tabulce.

\begin{table}[H]
	\centering
      \resizebox{\textwidth}{!}{%
	\begin{tabular}{c r@{}l c c c}
        \toprule
        & \multicolumn{1}{r}{\textbf{Cena} \tablefootnote{Cena v korunách je získána z webu \url{https://eshop.albi.cz/} a je aktuální k datu 21.2.2024.}} &
        & \textbf{Herní čas celkem} 
        & \textbf{Herní čas sezení} 
        & \textbf{Počet hráčů} 
        \\
		\midrule
		\textbf{Na vlnách neznáma} 
            & 1 849 Kč &
            & 1 -- 3 dny
            & 2 -- 4 hodiny
            & 3 -- 7
            \\
		\textbf{Dungeons \& Dragons} 
            & 2 096 Kč &\tablefootnote{Jedná se o cenu za základní tři knihy v korunách, které jsou potřeba k hraní. Informace je získána z webu \url{https://www.dndbeyond.com//} a je aktuální k datu 21.2.2024.}
            & 2 měsíce -- 2 roky
            & 4 -- 8 hodin
            & 2 -- 6
            \\
		\textbf{Gloomhaven} 
            & 3 499 Kč &
            & 1 měsíc -- 1 rok
            & 2 -- 4 hodiny
            & 1 -- 4
            \\
		\bottomrule
	\end{tabular}}
	\label{tab:game_comparison}
	\caption{Porovnání analyzovaných her}
\end{table}

Ze schématu lze vyčíst, že všechny vybrané hry jsou obsáhlejší povahy, což se odráží v jejich nadprůměrném herním času a také vyšší cenové kategorii. Tyto statistiky svědčí tomu, že i modelová hra by měla tyto charakteristiky imitovat.

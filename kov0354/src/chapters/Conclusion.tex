\chapter{Závěr}

Tato bakalářská práce se zabývala tvorbou herního modelu výpravné evoluční hry, která je určena hybridní využití fyzických prvků a digitálního prostředí. Cílem této práce bylo navrhnout a vytvořit komplexní herní model, který reflektuje principy evoluce a poskytuje hráčům bohaté a dynamické herní zážitky. Řešení bylo realizováno ve skupině

Průběh práce zahrnoval několik klíčových kroků. Prvním krokem byla analýza historie deskových her, kde byly zmíněny klíčové mezníky vývoje tohoto průmyslu. Dále byla provedena analýza herních mechanik a prvků, od herní struktury, přes tahy, akce, pohyb až po způsoby ukončení hry. Teoretická část byla završena analýzou tří výpravných deskových her s různými mírami integrace aplikace, které posloužily jako inspirace pro návrh herního modelu.

V praktické části byly nejprve vytyčeny požadavky, které by měl herní model splňovat. Následně byl tento model navržen s ohledem na výše provedenou analýzu. Výsledkem bylo schéma herních modelů, strukturovaný popis herních mechanik a návrh fyzických komponent.

Na závěr byla provedena implementace herního modelu do digitální podoby, kde byly vytvořeny základní herní mechaniky a pravidla. Výsledný prototyp byl integrován s výsledky ostatních členů týmu a byl otestován na základě zvolených kritérií. Výsledkem bylo ověření funkčnosti herního modelu a jeho schopnosti poskytnout hráčům bohatý a dynamický herní zážitek.

Jako potenciální rozšíření do budoucnosti by mohlo být rozšíření o nové herní mechaniky nebo fyzické komponenty. Herní model byl navrhnut s ohledem na budoucí expanze, proto by bylo vhodné pro hru vyvíjet pravidelné aktualizace a placené expanze, které by hráčům přinášely nové zážitky a udržovaly jejich zájem o hru.

% Co mi bp přinesla:
% \begin{itemize}
%     \item Nové deskovky, co si chci zahrát (Legacy of Dragonholt, Mansions of Madness)
%     \item deprese
% \end{itemize}

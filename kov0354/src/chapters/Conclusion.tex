\chapter{Závěr}

Tato bakalářská práce se zabývala tvorbou herního modelu výpravné evoluční hry, která je určena na hybridní využití fyzických prvků a~digitálního prostředí. Cílem této práce bylo vytvořit komplexní herní model, který reflektuje principy evoluce a~poskytuje hráčům dynamické herní zážitky. Řešení bylo realizováno ve skupině čtyř studentů, kteří se podíleli na různých částech projektu.

Průběh práce zahrnoval několik klíčových kroků. Prvním krokem byla analýza historie deskových her, kde byly zmíněny klíčové mezníky vývoje tohoto průmyslu. Dále byla provedena analýza herních mechanik a~prvků od herní struktury, přes tahy, akce, pohyb až po způsoby ukončení hry. Teoretická část byla završena analýzou tří výpravných deskových her s~různými mírami integrace aplikace, které posloužily jako inspirace pro návrh vlastního herního modelu.

V~praktické části byly nejprve vytyčeny požadavky, které by měl herní model splňovat. Následně byl tento model navržen s~ohledem na výše provedenou analýzu a~poznatky a~preference ostatních členů týmu. Výsledkem bylo schéma herních modelů, strukturovaný popis herních mechanik a~návrh fyzických komponent.

Na závěr byla provedena implementace herního modelu do digitální podoby, kde byly vytvořeny základní herní mechaniky a~pravidla, se kterými mohou uživatelé interagovat v~rámci API. Fyzické komponenty jako herní deska, karty a~figurky byly navrženy a~vytvořeny v~souladu s~digitálním herním modelem. Výsledný prototyp byl integrován s~výsledky ostatních členů týmu a~jeho funkčnost byla otestována několika odehranými scénáři. Výsledkem bylo ověření funkčnosti herního modelu a~jeho schopnosti poskytnout hráčům bohatý a~dynamický herní zážitek.

Pro budoucí vývoj a~výzkum v~oblasti hybridních deskových her je doporučeno pokračovat ve zkoumání nových technologických integrací, případně rozšíření o~nové herní mechaniky nebo fyzické komponenty. Jako možné rozšíření se také vybízí implementace nevyužitých herních mechanik, které byly odmítnuty z~časových důvodů. Herní model byl navrhnut s~ohledem na budoucí expanze, proto by bylo vhodné pro hru vyvíjet pravidelné aktualizace a~placené expanze, které by hráčům přinášely nové zážitky a~udržovaly jejich zájem o~hru. Rozvoj by se také mohl ubírat směrem umělé inteligence, která by mohla hru obohatit o~dynamiku.

% Co mi bp přinesla:
% \begin{itemize}
%     \item Nové deskovky, co si chci zahrát (Legacy of Dragonholt, Mansions of Madness)
%     \item deprese
% \end{itemize}

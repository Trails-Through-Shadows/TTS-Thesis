\chapter{Analýza existujících her}
\label{chap:game_analysis}

Tato kapitola se zaměřuje na analýzu několika existujících deskových her, které poslouží jako inspirace pro vlastní návrh modelové hry. Rozebrány budou klady i zápory těchto her, co je dělá zajímavými, co je na nich dobré a co by naopak bylo třeba změnit. Analýza se bude zaměřovat především na herní principy, které byly uvedeny v předchozí kapitole \chapterref{chap:game_design}.


%%% Forgotten Waters %%%

\section{Na vlnách neznáma}
\label{sec:forgotten_waters}

\glsref{forgotten_waters} (anglicky \textit{Forgotten Waters}) je kooperativní desková hra pro 3 až 7 hráčů, kterou vydalo v roce 2020 vydavatelství Plaid Hat Games. Hra je zasazena do fantasy světa, kde hráči přebírají role pirátů, jejichž úkolem je spolupracovat, aby úspěšně splnili sen jejich kapitána. Každá z postav má však své vlastní cíle, které jsou reprezentovány vývojem postavy a jejími schopnostmi. Důležitou součástí hry je webová aplikace, která poskytuje hudební a zvukové efekty, včetně hlasového vyprávění, které hráče provází příběhem a vytváří tak atmosféru hry. \cite{forgotten_waters}


\subsection{Herní systém}
\label{subsec:fw_gameplay}

Z pohledu herní struktury tato hra jednoznačně spadá do kategorie kooperativních her \chapterref{subsec:structure_cooperative}, je však obohacena i o prvky scénářových \chapterref{subsec:structure_scenario} a legacy her \chapterref{subsec:structure_legacy}. Před samotnou hrou si hráči vyberou, pod kterým kapitánem budou sloužit jako posádka, což je v podstatě abstrakce nad pouhým vybráním herní kampaně, která obsahuje předpřipravený scénář, včetně herní mapy a příběhu, kterým budou hráči procházet. Samotná hra pak spočívá v cestování mezi lokacemi a potýkání se s událostmi, které tyto lokace nesou. Kampaň je rozdělena do několika částí, mezi kterými jsou hráči vybídnuti, aby si hru uložili a pokračovali v ní později. Jednotlivé části mohou trvat několik hodin, pokud však hráči chtějí, kampaň jde odehrát během jednoho dlouhého sezení. Herní komponenty se hrou však přímo neznehodnocují, s jedním balením je možné odehrát libovolný počet kampaní.

Lokace se ve hře vybírají pomocí pohybu lodi po mapě tvořené hexagony. Podle scénáře jsou některé hexagony předem určené, jiné jsou prázdné a když na ně hráči stoupnou, umístí místo nich náhodně vylosovanou lokaci. Herní pole je tedy rozděleno dvoudimenzionálně a mění se dynamicky \chapterref{subsec:map_reveal} podle postupu posádky. Pohyb lodi je ovlivněn hodem hráče, který si vybral, že ji bude toto kolo řídit. Podle jeho úspěšnosti se loď posune u určitý počet polí, pohyb je tedy náhodný \chapterref{subsec:movement_roll_movement} s přidanými modifikátory.

V rámci lokace je hráčům představeno několik možností, které si může vybrat buď několik lidí, nebo jsou přístupné pouze prvnímu z nich \chapterref{subsec:actions_limited_actions}. Některé z nich jsou také povinné - musí si je vybrat alespoň jeden hráč. Tyto možnosti jsou reprezentovány jako vytištěný seznam akcí na stránce knihy, která je součástí balení. Výběr akcí probíhá v pořadí iniciativy, která je reprezentovaná stupnicí věhlasu, jde tedy pořadí podle jisté statistiky \chapterref{subsec:turns_stat_order}. Samotné vyhodnocování probíhá v pořadí od první akce po poslední tak, jak jsou na stránce vytištěny.

Co se týče konce hry, ten je určený předem připraveným scénářem. Každá část končí nějakou událostí, která se vyvolá po dokončení lokace. Po dokončení všech částí hry je dokončen i příběh a následuje příběhový epilog, vyhodnocený podle toho, jak se hráči během hry chovali. Jde tedy o příklad ukončení na základě splnění kooperativního cíle \chapterref{subsec:end_cooperative_goals}. Na závěr hry však ještě probíhá vyhodnocení příběhu jednotlivých postav, které v průběhu hry získávají body souhvězdí, které následně určují, jakým způsobem se příběh postavy vyvine. Jedná se tedy o ukázku konce hry na podle bodů vítězství \chapterref{subsec:end_victory_points}.


\subsection{Fyzické komponenty}
\label{subsec:fw_components}

\glsref{forgotten_waters} svůj herní systém podporuje několika fyzickými komponenty, které jsou součástí balení hry. Ty nejdůležitější z nich jsou přiblíženy níže.

\subsubsection*{Herní deska}
\label{subsubsec:fw_comp_board_and_book}

Herní deska je tvořena hexagonálním polem, které reprezentuje moře, po kterém se hráči pohybují. Na tuto prázdnou šablonu se umisťují hexagonální žetony navigace, které představují místa, kterými hráči proplouvají. Když hráči plují přes prázdné pole, náhodně vylosují žeton, který na toto pole umístí, což dodává hře pocit, že v ní opravdu prozkoumávají neznámé moře. Jednotlivé žetony navigace mají své vlastní identifikační číslo, které přes aplikaci odkáže hráče na příslušnou stránku v knize lokací.

Kniha lokací je příručka, která obsahuje všechny lokace, které mohou hráči navštívit. Každá lokace má svou vlastní stránku, na které je popsána, včetně možností, které na ní hráči mají. Na levé straně jsou vždy vytištěné jednotlivé akce, na které si hráči během kola postaví své figurky a dále s těmito akcemi pracují. Na pravé straně je vytištěný tematický obrázek, který hře dodává atmosféru.

\subsubsection*{Karty}
\label{subsubsec:fw_comp_cards}

Většina ve hře karet reprezentuje předměty, které mohou hráči získat a použít. Dělí se na poklady, které hráči získávají různými způsoby během hry, a příběhové karty, které se vztahují k příběhu postav. Karty přinášejí bonusy ke schopnostem, které postavy mohou využít, a také mohou nést složitější bonusy, které jsou ve zkratce popsané na kartě. Jedná se o jednoduchý způsob, jak hráčům umožnit jedinečný vývoj jejich postav, který je zároveň snadno zpřístupněný a přehledný.

\subsubsection*{Postavy a role}
\label{subsubsec:fw_comp_roles}

Jako posádka lodi si hráči před začátkem hry rozdělí role, o jejichž povinnosti se musí postarat. Spolu se svou rolí dostanou i list nebo desku, která slouží jako počítadlo pro různé statistiky, které budou hlídat. Jedná se o následující role:

\begin{itemize}
    \item \textbf{Lodní písař}: Stará se o lodní deník.
    \item \textbf{Kormidelník}: Hlídá věhlas, který určuje iniciativu při výběru akcí.
    \item \textbf{První důstojník}: Hlídá nespokojenost a počet členů posádky.
    \item \textbf{Bocman}: Kontroluje stav trupu - pokud se loď potopí, hra končí.
    \item \textbf{Bednář}: Hlídá lodní zásoby.
    \item \textbf{Dělostřelec}: Obsluhuje lodní děla, která se využívají především v soubojích.
    \item \textbf{Pozorovatel}: Hlídá průběh příběhu.
\end{itemize}

Pomocí takovéhoto rozdělení rolí mezi hráče je zajištěno, že se všichni hráči budou muset podílet na řízení lodi, a zároveň se také zamezí tomu, aby se některé činnosti opakovaly.

Samotné postavy také mají své vlastní deníky. Tyto deníky slouží jako záznamník pro vývoj postavy, který je reprezentován body souhvězdí, které postava získává během hry. Po konci příběhu se podle těchto bodů vyhodnotí, jaký osud postavu potkal. Kromě těchto příběhů a jejich potenciálu na získávání schopností různé úrovně se však postavy moc neliší.

\subsubsection*{Kostky}
\label{subsubsec:fw_comp_dice}

Pro vyhodnocování úspěšnosti akcí se ve hře používají dvanáctistěnné kostky, které se hází vždy, když je třeba určit výsledek zkoušky dovednosti. Když si hráč háže například na sílu, k výsledku hodu kostkou přičte svou vlastní hodnotu síly a případně bonusy, které mohl získat z karet. Výsledek pak porovná s obtížností úkolu, kterou určuje scénář.

Zpestření hodů zajišťují speciální žetony, které upravují výsledek hodu. Jedná se o žetony neštěstí, při kterých si hráč musí hodit znovu a počítat si horší výsledek, a žetony přehození, které naopak umožňují ze dvou výsledků počítat ten lepší. 

\subsubsection*{Figurky}
\label{subsubsec:fw_comp_figures}

\glsref{forgotten_waters} nepoužívá přímo figurky, ale volí levnější variantu - tzv. standees, ilustrace vytištěné na kartonu, které se postaví do plastových stojanů. Tyto standees reprezentují jak postavy v rámci lokace, tak i samotnou loď na herním plánu. Jednoduchost tohoto způsobu umožňuje snadnou reprezentaci postav a zároveň snižuje náklady na výrobu hry.


\subsection{Podpůrné aplikace}
\label{subsec:fw_apps}

Hlavní funkcionalitu, kterou aplikace přináší, je průvodce příběhem. Při vstupu si hráči zvolí kampaň, kterou chtějí hrát, načež jim aplikace přehraje úvodní dialog a pokud je to potřeba, krok po kroku jim ukáže, jak si mají herní desku a ostatní komponenty připravit. Dále aplikace slouží jako knihovna informací. Každý záznam, který hráčům ukazuje, má svůj třímístný kód, pomocí jehož hráči mohou na záznamy přistupovat. Tento přístup poskytuje vývojářům velkou flexibilitu v tom, odkud mohou hráči kódy získávat. Může jít o pouhé uvedení lokace (jak již bylo výše zmíněno, každá lokace má svůj odpovídající kód), krátkou anekdotu po získání nového předmětu nebo i o příběhový výsledek nějakého rozhodnutí. Jako vedlejší funkci aplikace také přináší hudební a zvukové efekty, které hru doplňují a vytváří kýženou atmosféru. \cite{fw_crossroads_app}

Nově také vznikla oficiální webová aplikace, která umožňuje hru hrát plně online. Na stránkách je možné si založit herní místnost, do které se mohou připojit ostatní hráči. Během hry jsou pak digitalizovány všechny komponenty, které jsou potřeba k hraní, včetně herní desky, knihy lokací, karet a počítadel pro jednotlivé role. \cite{fw_remote_app}

\subsection{Herní pravidla}
\label{subsec:fw_rules}

Pravidla hry \glsref{forgotten_waters} jsou přiložena jako součást balení, jsou však přístupná i v digitální formě na stránkách vydavatele. Hned ze začátku jsou hráči seznámeni s důležitostí webové aplikace a je jim předložen odkaz, přes který se na ni dostanou. Následně je ve zkratce uveden a shrnut veškerý herní materiál, který je v balení obsažen, a složitější komponenty jsou vyobrazeny nákresem s popisky klíčových částí.

Hlavní část pravidel je věnována popisu postupu na přípravu hry a průběhu jednotlivých kroků. Příprava je rozepsána do očíslovaných kroků a doplněna o ilustrace, které hráčům usnadňují orientaci. Následně je hráčům vysvětleno, jak je rozděleno kolo a jak fungují akce v něm. Dále jsou také přiblíženy jednotlivé postavy, jejich dovednosti a zkoušky, které se k nim vztahují. Na závěr jsou hráči seznámeni s tím, jak může hra skončit, ať už pozitivně či negativně, a jak se hra ukládá, pokud se rozhodnou ji hrát v oddělených sezeních. Jako zajímavý bonus na konci pravidel je přiložen generátor pirátských jmen, který hráčům pomocí dvou kostek umožní vygenerovat si jméno pro svou postavu.


%%% Dungeons & Dragons %%%

\section{Dungeons \& Dragons}
\label{sec:dungeons_and_dragons}

Obrovský vliv, který měla revoluční RPG hra \glsref{dnd} na vývoj stolních her, už byl zmíněn výše. Tato hra je však také zdrojem inspirace pro mnoho deskových her, které se snaží přenést některé z jejích prvků do stolního prostředí, a rozšířit tím svou potenciální hráčskou základnu. Tato sekce se zaměřuje na analýzu prvků, které byly převzaty z \dnd{} do jiných her. Oproti jiným hrám, které jsou v této práci rozebrané, je však \dnd{} mnohem komplexnější a množství jeho herních mechanik je také mnohem větší, proto se tato sekce bude držet spíše herních systémů a mechanik, které jsou popsány ve výše zmíněné kapitole \textit{\nameref{chap:game_design}}.


\subsection{Herní systém}
\label{subsec:dnd_gameplay}

Prvně je třeba zmínit, že \dnd{} se nedrží klasické symetrické struktury, kterou většina stolních her využívá. Hra je založena na asymetrickém systému, kde jeden z hráčů, označovaný jako \textit{Dungeon Master} (DM) nebo vypravěč, má na starosti vytváření a řízení celého světa, včetně všech postav, které v něm žijí. Ostatní hráči, v tomto světě následně hrají své vlastní postavy a dělají za ně rozhodnutí, stěžejní částí herního zážitku je tedy \textit{roleplay}. Tento systém je základem pro všechny ostatní mechaniky, které jsou v této hře používány. Z tohoto popisu se může zdát, že hráči a DM soupeří o jakousi výhru, ve skutečnosti však spolupracují na vytváření příběhu, který je pro všechny zúčastněné zábavný. Tatu skutečnost z \dnd{} dělá nejen pouhou stolní hru, ale prostředek pro kooperativní vyprávění příběhů.

Svou strukturou je \dnd{} ukázkovým příkladem legacy hry \chapterref{subsec:structure_legacy}, neboť běžná kampaň může trvat až roky a z pohledu ničení herních komponent se tato charakteristika odvíjí od konkrétní kampaně a také kreativity DM a ostatních hráčů. Je zřejmý také strukturální vliv scénáře \chapterref{subsec:structure_scenario} -- pro hráče je přístupná spousta předpřipravených scénářů a kampaní, kterými se DM může řídit, ať už jde o oficiální příběh od vývojářů nebo o neoficiální práci některého z mnoha fanoušků, tvořící pro hru další obsah. Hráči se však příběhu nemusí držet, pokud nechtějí, neboť volnost herních mechanismů dává velkou flexibilitu v improvizaci a nečekaných zvratech, které často nečeká ani sám vypravěč. Ve své podstatě se také jedná o kompetetivní hru \chapterref{subsec:structure_competitive}, i když během roleplaye a herních zvratů se může jednoduše stát, že se některé z postav obrátí proti sobě a je na hráčích, jak tyto konflikty vyřešit. Tyto konflikty by však měly zůstat pouze v mezích herního světa a nevnášet neshody mezi samotné hráče.

I přes to, že příběhové sekce a částí více zaměřené na roleplay nemají téměř žádnou strukturu, během souboje je struktura relativně pevná, samozřejmě s výjimkami. Před soubojem si všechny postavy i nepřátelé hodí na iniciativu, k výsledku si přičítají své vlastní bonusy a výsledné číslo určí pořadí kola, jde tedy o určení pořadí podle statistiky \chapterref{subsec:turns_stat_order}. Ve svém kole má poté hráč přesně určená pravidla, na to, jaké akce může provést. Obvykle se jedná o pohyb, akci (útok zbraní nebo seslání kouzla) a bonusovou akci (interakci s předmětem, vytasení zbraně nebo speciální kouzla), přičemž hráč si může vybrat v jakém pořadí tyto akce provede, pokud vůbec. 

Co se týče pohybu, herní pole je obvykle rozděleno na hexagony, i když přesný vzhled mapy záleží čistě na DM. Po herním poli se hráči mohou pohybovat ve svém kole, vzdálenost, kterou urazí je zapsána v jejich deníku postavy. Mají také možnost sprintovat, což jejich rychlost zdvojnásobí, za cenu spálení akce. Pravidlem také bývá, že akce by měly dávat z prostorového pohledu smysl, například že by se postavy neměly v boji přeskakovat ani střílet po nepřátelích, když jim stojí v cestě jejich spojenci. Opět ale záleží na rozhodnutí hráčů.

Samotné souboje jsou řešeny pomocí kostek. Každá zbraň má svůj vlastní druh kostky, který se hází, když se s ní útočí. Hráč si nejprve hodí kostkou, zda rána vůbec zasáhla, výsledek se porovná s obranou cíle, a pokud se útok povedl, cíl utrpí poškození. Kromě toho se také používají kostky k určení výsledku kouzla, které se sesílá. Výsledky těchto hodů se mohou dále upravovat pomocí bonusů, které postavy získávají z předmětů, nebo z jejich vlastních dovedností.

Konec hry je určený předem připraveným scénářem, který opět může být buď oficiální, nebo vytvořený DM. Hra končí, když hráči splní cíl, který je jim předložen, nebo když všichni hráči zemřou, i v takovém případě však je možnost vrátit se do hry jako jiná postava, pokud s tím ostatní hráči souhlasí. Výsledek hry je tedy určený splněním kooperativního cíle \chapterref{subsec:end_cooperative_goals}.

\subsection{Fyzické komponenty}
\label{subsec:dnd_components}

Flexibilita a otevřenost \dnd{} s sebou nese i osvobození od fyzických komponent, které by byly nutné k hraní. Hra jako taková ve své podstatě potřebuje jen tři komponenty -- pravidla, kostky a papír. I tak však existuje spousta dalších komponent, které mohou hru ozvláštnit.

\subsubsection*{Pravidla}
\label{subsubsec:dnd_comp_rules}

Pravidla \dnd{} jsou rozdělena na různé knihy, které se věnují jiným aspektům hry. Hlavními knihami jsou \textit{Player's Handbook}, která obsahuje pravidla pro hráče, \textit{Dungeon Master's Guide}, která pomáhá DM s vytvářením a řízením světa a \textit{Monster Manual}, která obsahuje předpřipravené příšery, se kterými se mohou hráči potýkat. Vývojáři však vydávají i další knihy, které mohou rozšiřovat herní svět, vyprávět nové příběhy, nebo přidávat nová pravidla, která mohou být využita v kampani.

\dnd{} má však i velmi aktivní komunitu, která vytváří vlastní obsah, a tím ještě více rozšiřuje možnosti hry. Opět může jít o nová pravidla, nové příšery, nebo i nové kampaně, které mohou být využity v rámci hry. Tento obsah může být dostupný zdarma nebo za poplatek, ať už v tištěné podobě, nebo v digitální formě.

\subsubsection*{Kostky}
\label{subsubsec:dnd_comp_dice}

\dnd{} je známé svými mnoha druhy kostek, které se během hry používají. Tou nejznámější, dalo by se říct až ikonickou, je dvacetistěnná kostka, která se během posledních let stala symbolem celé hry. Používá se nejčastěji, od určení iniciativy, přes útoky až po kontroly dovedností. Dalšími kostkami, které se používají, jsou čtyřstěnná, šestistěnná, osmistěnná, desetistěnná a dvanáctistěnná kostka, nejčastěji jako určení poškození od určité zbraně nebo kouzla. Každá sada kostek navíc obsahuje dvě desetistěnné kostky, z nichž jedna určuje desítky a druhá jednotky, což umožňuje hráčům házet procenta.

\subsubsection*{Deníky postav}
\label{subsubsec:dnd_comp_sheets}

I když je každá postava v družině jedinečná, všechny mají svůj deník, který obsahuje veškeré informace, které hráči potřebují k hraní, včetně statistik, dovedností, kouzel, předmětů a záznamů o postupu v kampani. V páté edici hry je list složen z několika částí. Všichni hráči mají na listu zapsané vlastnosti své postavy, klasicky se jedná o Sílu, Obratnost, Odolnost, Inteligenci, Moudrost a Charisma. Dále mají zapsané dovednosti, které se k těmto vlastnostem vážou, a které mohou během hry zlepšovat. Na listu je také místo pro záznamy o zdraví, obraně, iniciativě a jiných statistikách, které se během hry mění, a pro záznamy o zbraních, vybavení a zásobách, které s sebou postava nosí.

Každý z hráčů si však při tvorbě postavy volí rasu, povolání a zázemí, což jejich postavu dále ovlivňuje. I deníky postav na tyto změny reagují, hráči často využívají předpřipravené listy, které jsou personalizované pro jejich povolání, protože v něm jsou změny nejviditelnější. Mnoho hráčů si však vytváří své vlastní deníky tak, aby lépe vyhovovaly jejich potřebám.

\subsection{Podpůrné aplikace}
\label{subsec:dnd_apps}

Komplexita herního systému \dnd{} se přímo vybízí k využití podpůrných aplikací, které přichází jak v oficiální, tak i fanouškovské formě. Hlavní oficální aplikací je \textit{DnD Beyond}, která je zdarma, ale obsahuje i možnost dokoupení placeného obsahu, a slouží jako průvodce pravidly, zdroj pro vytváření postav a knihovna pro všechny dostupné příběhy a pravidla. Fanoušky vytvořené aplikace kopírují tyto funkcionality, ale zaměřují se také na jiné aspekty hry, od generátorů postav, přes knihovny příšer až po aplikace, které umožňují zaznamenávat celé kampaně a herní světy.

\subsection{Herní pravidla}
\label{subsec:dnd_rules}

Jak již bylo zmíněno výše, pravidla \dnd{} jsou rozdělena do několika knih, které se věnují různým aspektům hry. Hlavní pravidla, která se týkají hráčů, jsou obsažena v \textit{Player's Handbook}, která obsahuje všechny informace, které hráči potřebují k hraní. Nejprve krok po kroku popisuje, jak vytvořit postavu a následně se ponořuje do detailů ohledně rasy, povolání, zázemí a jiných rysů správné postavy. Dále jsou hráčům představeny základní zbraně, vybavení, jiné předměty a také kouzla, které jsou pak využita v následující sekci o pravidlech souboje.

\textit{Dungeon Master's Guide} je druhá kniha, která je určena především DM, a která se věnuje vytváření a řízení světa, ve kterém se hra odehrává. Jejím hlavním cílem je doprovázet vypravěče při přípravě hry, ať už se jedná o tvorbu světa, dobrodružství a příběhů v nich nebo nehráčských postav, které se v nich vyskytují. Přináší také cenné generátory náhodných událostí nebo předmětů pokladu, které mohou být využity v průběhu hry, a také rady, jak řešit problémy, které se mohou během hraní objevit.

Poslední z trojice základních pravidel je \textit{Monster Manual}, který obsahuje předpřipravené příšery, se kterými se mohou hráči potýkat. Každá příšera má svou vlastní sekci, ve které je popsán jak její příběh a místo v herním světě, tak její statistiky a schopnosti. Kniha je určena především pro DM, ale může být využita i hráči, kteří se o příšerách chtějí dozvědět více.

Během let však vzniklo mnoho dalších oficiálních placených knih, které rozšiřují možnosti hry. Může jít o nová pravidla (\textit{Xanathar's Guide to Everything} nebo \textit{Tasha's Cauldron of Everything}), nové příšery (\textit{Mordenkainen's Tome of Foes}), nebo i nové kampaně (\textit{Icewind Dale: Rime of the Frostmaiden}), které mohou být využity v rámci hry. Další rozšíření však vznikají i díky fanouškům, kteří vytvářejí vlastní obsah, který může být dostupný zdarma nebo za poplatek, ať už v tištěné podobě, nebo v digitální formě (\textit{Questonomicon}). Díky tomu se hráči nemusí bát, že jim dojde obsah, ze kterého by mohli pro své hry čerpat.

// todo dodat zdroje


%%% Gloomhaven %%%

\section{Gloomhaven}
\label{sec:gloomhaven}

\subsection{Herní systém}
\label{subsec:gh_gameplay}

\subsection{Fyzické komponenty}
\label{subsec:gh_components}

\subsection{Podpůrné aplikace}
\label{subsec:gh_apps}

\subsection{Herní pravidla}
\label{subsec:gh_rules}


%%% Comparison %%%

\section{Porovnání}
\label{sec:comparison}
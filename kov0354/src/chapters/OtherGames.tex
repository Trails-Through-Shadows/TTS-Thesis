\chapter{Analýza existujících her}
\label{chap:game_analysis}

Tato kapitola se zaměřuje na analýzu několika existujících deskových her, které poslouží jako inspirace pro vlastní návrh modelové hry. Rozebrány budou klady i zápory těchto her, co je dělá zajímavými, co je na nich dobré a co by naopak bylo třeba změnit. Analýza se bude zaměřovat především na herní principy, které byly uvedeny v předchozí kapitole \chapterref{chap:game_design}.


%%% Forgotten Waters %%%

\section{Na vlnách neznáma}
\label{sec:forgotten_waters}

\glsref{forgotten_waters} (anglicky \textit{Forgotten Waters}) je kooperativní desková hra pro 3 až 7 hráčů, kterou vydalo v roce 2020 vydavatelství Plaid Hat Games. Hra je zasazena do fantasy světa, kde hráči přebírají role pirátů, jejichž úkolem je spolupracovat, aby úspěšně splnili sen jejich kapitána. Každá z postav má však své vlastní cíle, které jsou reprezentovány vývojem postavy a jejími schopnostmi. Důležitou součástí hry je webová aplikace, která poskytuje hudební a zvukové efekty, včetně hlasového vyprávění, které hráče provází příběhem a vytváří tak atmosféru hry. \cite{forgotten_waters}


\subsection{Herní systém}
\label{subsec:fw_gameplay}

Z pohledu herní struktury tato hra jednoznačně spadá do kategorie kooperativních her \chapterref{subsec:structure_cooperative}, je však obohacena i o prvky scénářových \chapterref{subsec:structure_scenario} a legacy her \chapterref{subsec:structure_legacy}. Před samotnou hrou si hráči vyberou, pod kterým kapitánem budou sloužit jako posádka, což je v podstatě abstrakce nad pouhým vybráním herní kampaně, která obsahuje předpřipravený scénář, včetně herní mapy a příběhu, kterým budou hráči procházet. Samotná hra pak spočívá v cestování mezi lokacemi a potýkání se s událostmi, které tyto lokace nesou. Kampaň je rozdělena do několika částí, mezi kterými jsou hráči vybídnuti, aby si hru uložili a pokračovali v ní později. Jednotlivé části mohou trvat několik hodin, pokud však hráči chtějí, kampaň jde odehrát během jednoho dlouhého sezení.

Lokace se ve hře vybírají pomocí pohybu lodi po mapě tvořené hexagony. Podle scénáře jsou některé hexagony předem určené, jiné jsou prázdné a když na ně hráči stoupnou, umístí místo nich náhodně vylosovanou lokaci. Herní pole je tedy rozděleno dvoudimenzionálně a mění se dynamicky \chapterref{subsec:map_reveal} podle postupu posádky. Pohyb lodi je ovlivněn hodem hráče, který si vybral, že ji bude toto kolo řídit. Podle jeho úspěšnosti se loď posune u určitý počet polí, pohyb je tedy náhodný \chapterref{subsec:movement_roll_movement} s přidanými modifikátory.

V rámci lokace je hráčům představeno několik možností, které si může vybrat buď několik lidí, nebo jsou přístupné pouze prvnímu z nich \chapterref{subsec:actions_limited_actions}. Některé z nich jsou také povinné - musí si je vybrat alespoň jeden hráč. Tyto možnosti jsou reprezentovány jako vytištěný seznam akcí na stránce knihy, která je součástí balení. Výběr akcí probíhá v pořadí iniciativy, která je reprezentovaná stupnicí věhlasu, jde tedy pořadí podle jisté statistiky \chapterref{subsec:turns_stat_order}. Samotné vyhodnocování probíhá v pořadí od první akce po poslední tak, jak jsou na stránce vytištěny.

Co se týče konce hry, ten je určený předem připraveným scénářem. Každá část končí nějakou událostí, která se vyvolá po dokončení lokace. Po dokončení všech částí hry je dokončen i příběh a následuje příběhový epilog, vyhodnocený podle toho, jak se hráči během hry chovali. Jde tedy o příklad ukončení na základě splnění kooperativního cíle \chapterref{subsec:end_cooperative_goals}. Na závěr hry však ještě probíhá vyhodnocení příběhu jednotlivých postav, které v průběhu hry získávají body souhvězdí, které následně určují, jakým způsobem se příběh postavy vyvine. Jedná se tedy o ukázku konce hry na podle bodů vítězství \chapterref{subsec:end_victory_points}.


\subsection{Fyzické komponenty}
\label{subsec:fw_components}

\glsref{forgotten_waters} svůj herní systém podporuje několika fyzickými komponenty, které jsou součástí balení hry. Ty nejdůležitější z nich jsou přiblíženy níže.

\subsubsection*{Herní deska}
\label{subsubsec:fw_comp_board_and_book}

Herní deska je tvořena hexagonálním polem, které reprezentuje moře, po kterém se hráči pohybují. Na tuto prázdnou šablonu se umisťují hexagonální žetony navigace, které představují místa, kterými hráči proplouvají. Když hráči plují přes prázdné pole, náhodně vylosují žeton, který na toto pole umístí, což dodává hře pocit, že v ní opravdu prozkoumávají neznámé moře. Jednotlivé žetony navigace mají své vlastní identifikační číslo, které přes aplikaci odkáže hráče na příslušnou stránku v knize lokací.

Kniha lokací je příručka, která obsahuje všechny lokace, které mohou hráči navštívit. Každá lokace má svou vlastní stránku, na které je popsána, včetně možností, které na ní hráči mají. Na levé straně jsou vždy vytištěné jednotlivé akce, na které si hráči během kola postaví své figurky a dále s těmito akcemi pracují. Na pravé straně je vytištěný tematický obrázek, který hře dodává atmosféru.

\subsubsection*{Karty}
\label{subsubsec:fw_comp_cards}

Většina ve hře karet reprezentuje předměty, které mohou hráči získat a použít. Dělí se na poklady, které hráči získávají různými způsoby během hry, a příběhové karty, které se vztahují k příběhu postav. Karty přinášejí bonusy ke schopnostem, které postavy mohou využít, a také mohou nést složitější bonusy, které jsou ve zkratce popsané na kartě. Jedná se o jednoduchý způsob, jak hráčům umožnit jedinečný vývoj jejich postav, který je zároveň snadno zpřístupněný a přehledný.

\subsubsection*{Postavy a role}
\label{subsubsec:fw_comp_roles}

Jako posádka lodi si hráči před začátkem hry rozdělí role, o jejichž povinnosti se musí postarat. Spolu se svou rolí dostanou i list nebo desku, která slouží jako počítadlo pro různé statistiky, které budou hlídat. Jedná se o následující role:

\begin{itemize}
    \item \textbf{Lodní písař}: Stará se o lodní deník.
    \item \textbf{Kormidelník}: Hlídá věhlas, který určuje iniciativu při výběru akcí.
    \item \textbf{První důstojník}: Hlídá nespokojenost a počet členů posádky.
    \item \textbf{Bocman}: Kontroluje stav trupu - pokud se loď potopí, hra končí.
    \item \textbf{Bednář}: Hlídá lodní zásoby.
    \item \textbf{Dělostřelec}: Obsluhuje lodní děla, která se využívají především v soubojích.
    \item \textbf{Pozorovatel}: Hlídá průběh příběhu.
\end{itemize}

Pomocí takovéhoto rozdělení rolí mezi hráče je zajištěno, že se všichni hráči budou muset podílet na řízení lodi, a zároveň se také zamezí tomu, aby se některé činnosti opakovaly.

Samotné postavy také mají své vlastní deníky. Tyto deníky slouží jako záznamník pro vývoj postavy, který je reprezentován body souhvězdí, které postava získává během hry. Po konci příběhu se podle těchto bodů vyhodnotí, jaký osud postavu potkal. Kromě těchto příběhů a jejich potenciálu na získávání schopností různé úrovně se však postavy moc neliší.

\subsubsection*{Kostky}
\label{subsubsec:fw_comp_dice}

Pro vyhodnocování úspěšnosti akcí se ve hře používají dvanáctistěnné kostky, které se hází vždy, když je třeba určit výsledek zkoušky dovednosti. Když si hráč háže například na sílu, k výsledku hodu kostkou přičte svou vlastní hodnotu síly a případně bonusy, které mohl získat z karet. Výsledek pak porovná s obtížností úkolu, kterou určuje scénář.

Zpestření hodů zajišťují speciální žetony, které upravují výsledek hodu. Jedná se o žetony neštěstí, při kterých si hráč musí hodit znovu a počítat si horší výsledek, a žetony přehození, které naopak umožňují ze dvou výsledků počítat ten lepší. 

\subsubsection*{Figurky}
\label{subsubsec:fw_comp_figures}

\glsref{forgotten_waters} nepoužívá přímo figurky, ale volí levnější variantu - tzv. standees, ilustrace vytištěné na kartonu, které se postaví do plastových stojanů. Tyto standees reprezentují jak postavy v rámci lokace, tak i samotnou loď na herním plánu. Jednoduchost tohoto způsobu umožňuje snadnou reprezentaci postav a zároveň snižuje náklady na výrobu hry.


\subsection{Podpůrné aplikace}
\label{subsec:fw_apps}

Hlavní funkcionalitu, kterou aplikace přináší, je průvodce příběhem. Při vstupu si hráči zvolí kampaň, kterou chtějí hrát, načež jim aplikace přehraje úvodní dialog a pokud je to potřeba, krok po kroku jim ukáže, jak si mají herní desku a ostatní komponenty připravit. Dále aplikace slouží jako knihovna informací. Každý záznam, který hráčům ukazuje, má svůj třímístný kód, pomocí jehož hráči mohou na záznamy přistupovat. Tento přístup poskytuje vývojářům velkou flexibilitu v tom, odkud mohou hráči kódy získávat. Může jít o pouhé uvedení lokace (jak již bylo výše zmíněno, každá lokace má svůj odpovídající kód), krátkou anekdotu po získání nového předmětu nebo i o příběhový výsledek nějakého rozhodnutí. Jako vedlejší funkci aplikace také přináší hudební a zvukové efekty, které hru doplňují a vytváří kýženou atmosféru. \cite{fw_crossroads_app}

Nově také vznikla oficiální webová aplikace, která umožňuje hru hrát plně online. Na stránkách je možné si založit herní místnost, do které se mohou připojit ostatní hráči. Během hry jsou pak digitalizovány všechny komponenty, které jsou potřeba k hraní, včetně herní desky, knihy lokací, karet a počítadel pro jednotlivé role. \cite{fw_remote_app}

\subsection{Herní pravidla}
\label{subsec:fw_rules}

Pravidla hry \glsref{forgotten_waters} jsou přiložena jako součást balení, jsou však přístupná i v digitální formě na stránkách vydavatele. Hned ze začátku jsou hráči seznámeni s důležitostí webové aplikace a je jim předložen odkaz, přes který se na ni dostanou. Následně je ve zkratce uveden a shrnut veškerý herní materiál, který je v balení obsažen, a složitější komponenty jsou vyobrazeny nákresem s popisky klíčových částí.

Hlavní část pravidel je věnována popisu postupu na přípravu hry a průběhu jednotlivých kroků. Příprava je rozepsána do očíslovaných kroků a doplněna o ilustrace, které hráčům usnadňují orientaci. Následně je hráčům vysvětleno, jak je rozděleno kolo a jak fungují akce v něm. Dále jsou také přiblíženy jednotlivé postavy, jejich dovednosti a zkoušky, které se k nim vztahují. Na závěr jsou hráči seznámeni s tím, jak může hra skončit, ať už pozitivně či negativně, a jak se hra ukládá, pokud se rozhodnou ji hrát v oddělených sezeních. Jako zajímavý bonus na konci pravidel je přiložen generátor pirátských jmen, který hráčům pomocí dvou kostek umožní vygenerovat si jméno pro svou postavu.


%%% Dungeons & Dragons %%%

\section{Dungeons \& Dragons}
\label{sec:dungeons_and_dragons}

Obrovský vliv, který měla revoluční RPG hra \glsref{dnd} na vývoj stolních her, už byl zmíněn výše. Tato hra je však také zdrojem inspirace pro mnoho deskových her, které se snaží přenést některé z jejích prvků do stolního prostředí, a rozšířit tím svou potenciální hráčskou základnu. Tato sekce se zaměřuje na analýzu prvků, které byly převzaty z \dnd{} do jiných her, a na to, jak se tyto prvky projevily v novém kontextu.


\subsection{Herní systém}
\label{subsec:dnd_gameplay}


%%% Gloomhaven %%%

\section{Gloomhaven}
\label{sec:gloomhaven}


%%% Comparison %%%

\section{Porovnání}
\label{sec:comparison}
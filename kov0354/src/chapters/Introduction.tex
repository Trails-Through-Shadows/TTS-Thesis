\chapter{Úvod}

Deskové hry si i přes konkurenci těch počítačových udržují své místo jako oblíbená volnočasová aktivita spojující lidi v reálném světě. V kombinaci fyzických a digitálních prvků takovým způsobem, aby se oba světy vzájemně doplňovaly, vzniká potenciál pro nové zážitky, které ještě mnoho her na dnešním trhu nevyužívá. Tato bakalářská práce se zabývá vytvořením a dokumentací herního systému pro takovouto hybridní deskovou hru, která kombinuje fyzicky hratelné komponenty a virtuální prostředí.

První část práce se zaměřuje na popis historie deskových her, analýzu herních mechanizmů často používaných během jejich vývoje a představení několika konkrétních her, které se staly inspirací pro vytvoření nového herního systému. Druhá část se věnuje návrhu herního systému, který je založen na principu výpravné evoluční hry a obsahuje schéma herních modelů, popis herních mechanik a návrh fyzických komponent. Poslední část se zabývá implementací herního systému do digitální podoby a testováním jeho funkčnosti.

Výsledným produktem této práce je herní systém, který reflektuje principy evoluce a poskytuje hráčům bohaté a dynamické herní zážitky. Tento systém je určen pro hybridní využití fyzických prvků a digitálního prostředí a může být v budoucnu rozšířen o nové herní mechaniky nebo fyzické komponenty.

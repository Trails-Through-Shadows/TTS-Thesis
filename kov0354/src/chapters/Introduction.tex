\chapter{Úvod}

Přestože herní svět v~dnešní době dominují počítačové hry, deskové hry si stále udržují své místo jako oblíbená volnočasová aktivita, která spojuje lidi ve skutečném světě. V~kombinaci fyzických a~digitálních prvků takovým způsobem, aby se oba světy vzájemně doplňovaly, vzniká potenciál pro nové zážitky, které ještě mnoho her na dnešním trhu nevyužívá. Tato bakalářská práce se zabývá tvorbou a~dokumentací herního systému pro hybridní deskovou hru, která integruje fyzicky hratelné komponenty s~virtuálním prostředím.

V~úvodní části práce jsou zkoumány historické aspekty deskových her a~analyzovány klíčové herní mechanizmy, které ovlivnily jejich vývoj. Také jsou zde představeny tři deskové hry, které se staly inspirací pro konceptuální základy modelového herního systému. Druhá část se pak zaměřuje na samotný návrh herního systému, který je založen na principu evoluční hry s~důrazem na dynamický vývoj a~adaptaci herního prostředí. Je zde rozebráno schéma herních modelů, popis herních mechanik a~návrh fyzických komponent, všechny vytvořeny s~ohledem na kooperativní hratelnost a~obohacení herního zážitku. Na závěr je popsána implementace herního systému, která zahrnuje popis týmové spolupráce, vývojového procesu tvorby herního systému a~jiných aspektů jeho realizace. 

Výsledným produktem této práce je herní systém, který demonstruje potenciál hybridních deskových her a~poskytuje hráčům bohaté a~dynamické herní zážitky. Tento systém je určen pro hybridní využití fyzických prvků a~digitálního prostředí a~může být v~budoucnu rozšířen o~nové herní mechaniky nebo fyzické komponenty.

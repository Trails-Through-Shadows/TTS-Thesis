\chapter{Požadavky na modelovou hru}
\label{chap:requirements}

Před samotným návrhem herního systému je nutné určit požadavky, které by měl herní systém a hra jako taková splňovat. Tyto požadavky byly stanoveny na základě analýzy existujících řešení \chapterref{chap:game_analysis} a vlastních představ týmu a byly rozděleny do následujících kategorií.

\section{Struktura hry}
\label{sec:req_structure}

\begin{itemize}
    \item \textbf{Kooperace} -- 
        Hra by měla být kooperativní, tedy hráči by měli spolupracovat na dosažení cíle hry společně.
    \item \textbf{Příběh} -- 
        Hra by se měla držet centrálního příběhu, který hráče zaujme a bude je v průběhu hry motivovat.
    \item \textbf{Jedinečnost postav} -- 
        Hráči by měli být schopni vytvořit jedinečnou postavu, která se od ostatních liší nejen vzhledem, ale i herním stylem.

    \begin{itemize}
        \item \textbf{Unikátní akce} -- 
            Postavy by měly mít akce, které jsou pro ně specifické a mají velký vliv na pocit, který hráči z postavy mají.
        \item \textbf{Odlišné statistiky} -- 
            Statistiky jsou další oblastí, ve které by se postavy měly lišit.
        \item \textbf{Vývoj postav} -- 
            Postavy by měly mít možnost se vyvíjet, získávat nové schopnosti a zlepšovat své statistiky, ať už získáváním zkušeností nebo pomocí předmětů.
    \end{itemize}

    \item \textbf{Rozdělení na světovou a lokální mapu} -- 
        Herní svět bude zobrazen ve velké mapě v rámci webové aplikace, která bude obsahovat jednotlivé lokace. Ty budou mít svou vlastní mapu určenou k hraní na stole, která bude detailnější a bude sloužit pro souboje a interakci s prostředím.
    \item \textbf{Propojení s aplikací} -- 
        Hra bude propojena s webovou aplikací, která bude sloužit pro zjednodušení hraní hry a bude obsahovat informace, které by byly obtížné zobrazit ve fyzických komponentách. Je však nutné dbát na to, aby si hra zachovala svůj deskový charakter.
\end{itemize}

\section{Herní mechaniky}
\label{sec:req_mechanics}

\begin{itemize}
    \item \textbf{Karty a kostky} -- 
        Souboj by měl být založen na kartách, které budou hráči umožňovat provádět různé akce, a kostkách, které budou přinášet do výsledku náhodnost.
    \item \textbf{Příběh} -- 
        Příběh bude vyprávěn především prostřednictvím webové aplikace.
    \item \textbf{Inventář a předměty} -- 
        Předměty budou důležitou součástí hry díky jejich vlivu na vylepšení postavy. Inventář bude sloužit k jejich ukládání a bude omezený na určité typy předmětů.
    \item \textbf{Efekty} -- 
        Hra by měla obsahovat efekty, které mohou ovlivňovat hráčské postavy nebo nepřítele.
\end{itemize}

\section{Komponenty}
\label{sec:req_components}

\begin{itemize}
    \item \textbf{Kostky} -- 
        Kostky budou sloužit k určení výsledku útoků a jiných akcí, které hráči provádějí. Na stěnách kostky budou zaznamenány modifikátory ovlivňující výsledek a každý hráč bude mít svou vlastní kostku.
    \item \textbf{Karty} --
        Karty budou reprezentovat akce v rámci hry a budou obsahovat strukturované informace o tom, jak se daná akce provádí a jaké má následky.
    \item \textbf{Figurky} --
        Postavy, nepřátelé, summoni a překážky budou reprezentovány figurkami, z důvodu omezeného rozpočtu se však bude jednat o \textit{standees}.
    \item \textbf{Mapy lokací} --
        Lokace bude tvořena jednotlivými menšími částmi, které bude možné skládat dohromady a spojovat do většího celku. Mapa bude založena na hexagonech.
    \item \textbf{Pravidla} --
        Pravidla budou zapsána ve vytištěné příručce přiložené jako součást balení hry, ale budou také dostupná v elektronické podobě. Hráče provedou průběhem hry a jednotlivými mechanismy v přehledné a graficky přívětivé formě.
    \item \textbf{Přizpůsobení pro webovou aplikaci} --
        Webová aplikace bude neodlučitelnou součástí hry a bude vytvořena primárně pro desktopové zařízení s možností responzivního designu pro zobrazení mobilních zařízeních. Bude obsahovat informace o postavách, lokacích, předmětech a dalších herních prvcích, které by byly obtížné zobrazit ve fyzické podobě a bude dbáno na grafickou koherenci s fyzickými komponentami.
\end{itemize}

\section{Komerční aspekty}
\label{sec:req_commercial}

\begin{itemize}
    \item \textbf{Cílová skupina} --
        Hra bude určena pro hráče, kteří již mají zkušenosti s deskovými hrami podobného stylu, ale zároveň bude přístupná i pro nováčky, kteří se chtějí do světa deskových her dostat.
    \item \textbf{Jednoduchost výroby} --
        Hra bude navržena tak, aby byla jednoduchá na výrobu.
    \item \textbf{Cena hry a aplikace} --
        Cena hry bude nastavena tak, aby byla dostupná pro většinu hráčů, ale zároveň aby bylo možné z ní pokrýt náklady na výrobu a distribuci. Webová aplikace bude dostupná zdarma po zadání licenčního klíče fyzické hry.
    \item \textbf{Rozšíření} --
        Hra bude navržena tak, aby bylo možné v budoucnu vytvořit rozšíření, které by přidalo nové postavy, lokace a příběh.
\end{itemize}
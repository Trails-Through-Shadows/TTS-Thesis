\chapter{Analýza a návrh herního systému}
\label{chap:design_analysis}

Pro rozumný vývoj herního systému modelové hry je nutné nejprve provést analýzu a návrh, proto se tato kapitola tímto tématem zabývá. První část se zaměřuje na požadavky, které by měl herní systém splňovat, a na analýzu herního systému. Druhá část se pak věnuje samotnému návrhu herního systému.

Hybridní deskové hře, která je finálním produktem této práce, bylo určeno pracovní jméno \textit{Trails Through Shadows} (zkratkou \textit{TTS}). Dále je však zmiňována především pod názvem \textit{modelová hra}.

\section{Požadavky na herní systém}
\label{sec:requirements}

Před samotným návrhem herního systému je nutné stanovit požadavky, které by měl herní systém a hra jako taková splňovat. Tyto požadavky byly stanoveny na základě analýzy cílů, které by měla modelová hra splňovat, a na základě analýzy herního systému.

- Struktura
    - Kooperativní
    - Příběhová
    - Jedinečnost postav
        - Akce podle postavy
        - Jiné statistiky taky jako pohyb a iniciativa
        - Vývoj postav
    - Velká mapa s jednotlivými lokacemi
    - Propojení s aplikací
        - Ale pořád zachovat deskový charakter

- Mechaniky
    - Karty na akce, kostky na damage (souboj obecně)
    - Příběh
    - Inventář a předměty
    - Efekty

- Komponenty
    - Kostky
    - Karty
    - Figurky (hráči, nepřátelé, summoni, obstacles)
    - Mapa
    - Pravidla
    - Přizpůsobení pro webovou aplikaci

- Komerční aspekty
    - Cílovka
    - Jednoduchost výroby
    - Nacenění hry a aplikace
    - Rozšíření